\title{Formal approaches to number in Slavic and beyond}
% \subtitle{Add subtitle here if it exists}
\BackBody{The goal of this collective monograph is to explore the relationship between the cognitive notion of \textsc{number} and various grammatical devices expressing this concept in natural language with a special focus on Slavic. The book aims at investigating different morphosyntactic and semantic categories including plurality and number-marking, individuation and countability, cumulativity, distributivity and collectivity, numerals, numeral modifiers and classifiers, as well as other quantifiers. It gathers 19 contributions tackling the main themes from different theoretical and methodological perspectives in order to contribute to our understanding of cross-linguistic patterns both in Slavic and non-Slavic languages.}
\author{Mojmír Dočekal and Marcin Wągiel}
\renewcommand{\lsSeries}{osl}% use series acronym in lower case
\renewcommand{\lsSeriesNumber}{5}
\renewcommand{\lsISBNdigital}{978-3-96110-314-0}
\renewcommand{\lsISBNhardcover}{978-3-98554-010-5}
\BookDOI{10.5281/zenodo.5082006}
\typesetter{Berit Gehrke, Radek Šimík, Marcin Wągiel, Mojmír Dočekal}
\proofreader{Christopher Rance}

\renewcommand{\lsID}{316}

\renewcommand{\lsImpressumExtra}{GAČR grant number GA17-16111S\\Reviewer of the book: Jakub Dotlačil}


% title font on cover
\renewcommand{\lsCoverTitleFont}[1]{%
    \sffamily\addfontfeatures{Scale=MatchUppercase}%
    \fontsize{47pt}{16mm}\selectfont #1}
