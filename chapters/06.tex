\documentclass[output=paper]{langscibook} 

\author{Suzana Fong\affiliation{Massachusetts Institute of Technology}}

\title{The syntax of plural marking: The view from bare nouns in Wolof}  
\abstract{A cross-linguistically stable property of bare nominals is number neutrality: they do not imply any commitment to a singular or plural interpretation. In Wolof, however, BNs are singular when unmodified and a plural interpretation only becomes available when a nominal-internal plural feature occurs. The generalization is that BNs in Wolof are singular, unless plural morphology is exponed. I propose that, while both a singular and plural NumP are available in Wolof, only the former leads to a convergent derivation. This is caused by the stipulation that the plural Num must lower onto \textit{n}, combined with the assumption that BNs lack an \textit{n}P. Number morphology becomes available when a relative clause is merged with the BN. The licensing of a RC implies the addition of an \textit{n}P, which allows a plural Num to satisfy its lowering requirement. Some nominal modifiers, however, do not have number morphology and they do not require the projection of \textit{n}P. As such, the plural Num cannot satisfy its requirement.

\keywords{Wolof, bare nominal, number neutrality}}

\begin{document}
\SetupAffiliations{mark style=none}
\maketitle

\section{Introduction}
	
Wolof (Niger-Congo, Senegal) has a rich set of overt determiners (see \citealt{ttz2012wolofQuatif}).
		
\ea \label{fon:foOlNmNlBzKk}
    \ea \gll	Xale y-i lekk-na-\~{n}u gato b-i.\\
				child \textsc{cm.pl-def} eat-\textsc{na-3pl} cake \textsc{cm.sg-def}\\
		\glt    `The children ate the cake.'
    \ex \gll	Xadi gis-na a-b sàcc.\\
				Xadi see-\textsc{na.3sg} \textsc{indef-cm.sg} thief\\
		\glt    `Xadi saw a thief.'\label{fon:foOlNmNlBzKk2}
	\ex \gll	Awa jàpp-na a-y sàcc.\\
				Awa catch-\textsc{na.3sg} \textsc{indef-cm.pl} thief\\
		\glt    `Awa caught some thieves.'\\\label{fon:foOlNmNlBzKk3}
	\sn \hfill \citep[(2a/32a/33b); glosses adapted for uniformity]{ttz2012wolofQuatif}
    \z
\z
		
\noindent   The determiner contains a class marker (\textsc{cm}; see \citealt{loporcaroBabou16}) affix. The class marker also encodes number information (singular or plural): \textit{sàcc} `thief' remains constant in \REF{fon:foOlNmNlBzKk2} and \REF{fon:foOlNmNlBzKk3}. Whether the DP it heads is interpreted as singular or plural is correlated with the class marker used, \textit{b} and \textit{y}, respectively.
	
Wolof also has \textsc{bare nominals} (BNs).
	
	\ea
		\gll	Gis-na-a {ndonggo darra} senegalee.\\
			see-\textsc{na-1sg} student Senegalese\\
		\glt    `I saw a Senegalese student.'\label{fon:pLlfAwS}
	\z
	
\noindent   I assume that BNs are nominals that lack the morphology displayed by their overt counterparts like those in (\ref{fon:foOlNmNlBzKk}). BNs in Wolof lack a(n overt) determiner and the class marker attached to it. Because of the absence of a class marker, there is also no overt number morphology.
	
BNs in Wolof seem to be narrow scope indefinites. They can be licensed in an existential construction, which displays definiteness effects:
		
		\ea
			\ea[]{\gll	Am-na a-b / a-y xaj ci biti.\\
			have-\textsc{na.3sg} \textsc{indef-cm.sg} {} \textsc{indef-cm.pl} dog \textsc{prep} outside\\
			\glt    `There is/are a/some dog(s) outside.'}
			\ex[*]{\gll	Am-na xaj b-i ci biti.\\
			have-\textsc{na.3sg} dog \textsc{cm.sg-def} \textsc{prep} outside\\
			\glt Intended: `There is the dog outside.'}
			\ex[]{\gll	Am-na {xaj} ci dool b-i.\\
			have-\textsc{na.3sg} dog \textsc{prep} garden \textsc{cm.sg-def}\\
			\glt    `There is a dog in the garden.'}\label{fon:dogsgard}
			\z
		\z

\noindent   Furthermore, they seem to take narrow scope.
		
		\ea \gll	Mareem s\'{e}y-aat-na ak {f\'{e}cckat}.\\
		Mareem marry-\textsc{iter}-\textsc{na.3sg} \textsc{conj} dancer\\
		\glt    `Mareem married a dancer again.'\label{fon:aAtBAyzZBnN}
		\sn \ea[\ding{55}]{`Mareem married the same dancer several times (e.g. marriage, followed by divorce, followed by another marriage).'}\label{fon:aAtBAyzZBnN1}
			\ex[\ding{51}]{`Mareem has a very specific preference and she has married several, different dancers.'}\label{fon:aAtBAyzZBnN2}
			\z
		\z\label{fon:aAtBAyzZBnNOwlL}
		
\noindent Several, unrelated languages have BNs too. Among them is Mandarin.
		
    		\ea \gll	Zuotian wo mai le {shu}.\\
				yesterday I buy \textsc{asp} book\\
				\glt    `Yesterday, I bought one or more books.'
                \sn{\hfill (Mandarin; \citealt[(1)]{rullmannYou2006})}\label{fon:cHaIBNcn}
		\z
		
\noindent   As can be gleaned from the translation, the BN in (\ref{fon:cHaIBNcn}) has a number neutral interpretation, that is, it lacks a commitment to a singular or plural interpretation. This property is also known as ``general number'' \citep{corbettNumber}. 
		
Conversely, BNs in Wolof seem to be exclusively singular. This can be demonstrated by the fact that BNs cannot saturate a collective predicate (\ref{fon:bNnGaThHr}) or be the antecedent of plural discourse anaphora (\ref{fon:disKanFFf}).
		
		\ea[*]{\gll	J\`{a}ngalekat b-i dajeele-na {xale} ci bayaal b-i.\\
			teacher \textsc{cm.sg-def} gather-\textsc{na.3sg} child \textsc{prep} park \textsc{cm.sg-def}\\
			\glt Intended: `The teacher gathered child in the park.'}\label{fon:bNnGaThHr}
		\z
		
		\ea 
	        \gll    Gis-na-a {j\`{a}ngalekat}. Maymuna b\"{e}gg-na ko / \minsp{*} leen.\\
			see-\textsc{na-1sg} teacher Maymuna like-\textsc{na.3sg} \textsc{obj.3sg} {} {} \textsc{obj.3pl}\\
			\glt    `I saw teacher yesterday. Maymuna admires her.'\label{fon:disKanFFf}
		\z
		
\noindent   One may compare the Wolof data above with the behavior of BNs in Mandarin with respect to the same properties:
			
    \ea \gll Zuotian wo mai le {shu}. Wo ba ta / tamen dai hui jia le.\\
		yesterday I buy \textsc{asp} book. I \textsc{ba} it {} them bring back home \textsc{asp}\\
		\glt    `Yesterday, I bought one or more books. I brought it/them home.'
		\sn{\hfill (Mandarin; \citealt{rullmannYou2006})}
	\z
		
	\ea \gll	Laoshi zai gongyuan-li jihe-le {xuesheng}.\\
			teacher at park-in gather-\textsc{perf} student\\
			\glt    `The teacher gathered the students in the park.'
		\sn{\hfill (Mandarin; Fulang Chen, p.c.)}
	\z
		
In order to account for the singular (and not number neutral) interpretation of BNs in Wolof,  I will propose that the source of the singular interpretation of unmodified BNs in Wolof is nominal-internal. Compared to full nominals, BNs will be proposed to have a truncated structure. Specifically, they include only a Number Phrase (NumP) above the root. Wolof must have both a singular and a plural NumP. The NumP in BNs could in principle be plural too. But I stipulate that the plural Num must obligatorily lower onto \textit{n}. Because BNs lack a \textit{n}, the requirement that Num lower onto \textit{n} cannot be fulfilled. As such, the only convergent derivation is one where Num is singular. The correlation between the size of the structure and the number interpretation of a BN will be shown to be consistent with the effects that different modifiers may have on the number interpretation.

\section{BNs in Wolof are singular (when unmodified)}
\label{fon:sEk:BnNSgG}

In this section, we will examine data that suggest that BNs in Wolof are singular. We will first examine the behavior of full nominals to establish a baseline to compare BNs with.

First, (\ref{fon:dAjPlReKweRmTt}) demonstrates that \textit{dajeele} is a collective predicate and thus requires a plural object.
	
	\ea \gll	J\`{a}ngalekat b-i dajeele-na \minsp{*} a-b xale / a-y xale ci bayaal b-i.\\
			teacher \textsc{cm.sg-def} gather-\textsc{na.3sg} {} \textsc{cm.sg-indef} child {} \textsc{cm.pl-indef} child \textsc{prep} park \textsc{cm.sg-def}\\
			\glt `The teacher gathered some children in the park.'\label{fon:dAjPlReKweRmTt}
	\z

\noindent   (\ref{fon:bNnGaThHr}) above has already showed that a BN cannot saturate this predicate.
	
Second, a pronoun that refers back to a full nominal must match its number feature:
	
	\ea
		\ea \gll	Gis-na-a a-b j\`{a}ngalekat. Maymuna b\"{e}gg-na ko /\hspace{5pt} \minsp{*} leen.\\
					see-\textsc{na-1sg} \textsc{indef-cm.sg} teacher Maymuna like-\textsc{na.3sg} \textsc{obj.3sg} {} {} \textsc{obj.3pl}\\
					\glt   `I saw a teacher yesterday. Maymuna admires her.'
		\ex \gll	Gis-na-a a-y j\`{a}ngalekat. Maymuna b\"{e}gg-na \minsp{*} ko / leen.\\
					see-\textsc{na-1sg} \textsc{indef-cm.pl} teacher Maymuna like-\textsc{na.3sg} {} \textsc{obj.3sg} {} \textsc{obj.3pl}\\
					\glt   `I saw some teachers yesterday. Maymuna admires them.'
		\z
	\z

\noindent   We saw in (\ref{fon:disKanFFf}) above that, if a BN is the antecedent, discourse anaphora can only be singular.
	
Third, only a plural full nominal can be the antecedent of a reciprocal.
	
	\ea
	    \ea[*]{\gll	J\`{a}ngalekat b-i wanale-na a-b ndonggo{ }darra mu xam-ante.\\
				teacher \textsc{cm.sg-def} introduce-\textsc{na.3sg} \textsc{cm.sg-indef} student \textsc{3sg} know-\textsc{recip}\\
				\glt Intended: `The teacher introduced a student to each other.'}\label{fon:waNAleNoPeGaPe1}
	    \ex[]{\gll	J\`{a}ngalekat b-i wanale-na a-y ndonggo{ }darra \~{n}u xam-ante.\\
				teacher \textsc{cm.sg-def} introduce-\textsc{na.3sg} \textsc{cm.pl-indef} student \textsc{3pl} know-\textsc{recip}\\
				\glt `The teacher introduced some students to each other.'}\label{fon:waNAleNoPe1}
		\z
	\z

\noindent   If a BN is the antecedent, the resulting sentence is ungrammatical (\ref{fon:rEsIpRoKk}).

	\ea[*]{\gll	J\`{a}ngalekat b-i wanale-na {ndonggo{ }darra} mu / \~{n}u xam-ante.\\
		teacher \textsc{cm.sg-def} introduce-\textsc{na.3sg} student \textsc{3sg} {} \textsc{3pl} know-\textsc{recip}\\
		\glt Intended: `The teacher introduced student to each other.'}\label{fon:rEsIpRoKk}
	\z

\noindent	A similar effect can be seen with plural reflexives. As expected, a reflexive and its antecedent must have the same number features.

	

	\ea 
	    \ea[]{\gll	Kadeer sang-aloo-na xale y-i seen bopp.\\
				Kadeer wash-\textsc{caus}-\textsc{na.3sg} student \textsc{cm.pl-def} \textsc{poss.3pl} head\\
				\glt    `Kadeer made the children wash themselves.'}\label{fon:kOzARiVv3}
		\ex[]{\gll	Kadeer sang-aloo-na xale b-i bopp$=$am.\\
				Kadeer wash-\textsc{caus}-\textsc{na.3sg} student \textsc{cm.sg-def} head$=$\textsc{poss.3sg}\\
				\glt    `Kadeer made the child wash themselves.'}\label{fon:kOzARiVvGaPe1}
		\ex[*]{\gll	Kadeer sang-aloo-na xale b-i seen bopp.\\
				Kadeer wash-\textsc{caus}-\textsc{na.3sg} student \textsc{cm.sg-def} \textsc{poss.3pl} head\\
				\glt Intended: `Kadeer made the child wash themselves.'}\label{fon:kOzARiVvGaPe2}
		\z
	\z

\noindent   Following the pattern that we have seen so far, a BN cannot be the antecedent of a plural reflexive.
	
	\ea[*]{\gll	J\`{a}ngalekat b-i sang-aloo-na {ndonggo{ }darra} seen bopp.\\
		teacher \textsc{cm.sg-def} wash-\textsc{caus}-\textsc{na.3sg} student \textsc{poss.3pl} head\\
		\glt Intended: `The teacher made student wash themselves.'}\label{fon:kOzARiVv4}
	\z

\noindent   But it can be the antecedent of a singular reflexive. As such, (\ref{fon:kOzARiVv4})'s ill-formedness cannot be caused by the BN's inability to be an antecedent.
		
		\ea \gll	J\`{a}ngalekat b-i sang-aloo-na {ndonggo{ }darra} bopp$=$am.\\
			teacher \textsc{cm.sg-def} wash-\textsc{caus}-\textsc{na.3sg} student head$=$\textsc{poss.3sg}\\
			\glt `The teacher made some student wash himself/herself.'\label{fon:kOzARiVv5}
		\z
		
\noindent To summarize what we have seen so far, BNs in Wolof exhibit the same behavior as that showcased by their singular, full nominal counterparts. A generalization that can be drawn from these data is that BNs in Wolof are singular. This contrasts with what is usually considered to be a crosslinguistic stable property of BNs, namely, a number neutral interpretation \citep{dayal2011hindi}. The question that we must then ask is the following: how can we account for the exclusively singular interpretation (and not number neutral) interpretation of BNs in Wolof? Before proceeding to an analysis that tries to address this question, we will see data that indicate that the generalization arrived at above is too strong. More precisely, we will see that, if we add a modifier to the BN, if the modifier contains plural morphology, the BN can indeed have a plural interpretation. This is going to be the case of relative clauses, which display complementizer agreement in Wolof. In contrast, if the modifier does not contain any number exponent, a BN retains its exclusively singular interpretation.

\section{Adding a modifier: Relative clauses vs. plain modifiers}

\subsection{Relative clause}
\label{fon:sEk:ReLl}

In Wolof, a relative clause contains a class marker \citep{loporcaroBabou16} attached to the relative complementizer \textit{u} \citep{torrence2013clause}. The class marker of the relative clause and that of the determiner outside the relative clause must match.
		
    \ea
		\ea \gll	Samba tej-na palanteer \minsp{[} b-u tilim] b-i /\hspace{30pt} \minsp{*} y-i.\\
				Samba close-\textsc{na.3sg} window {} \textsc{cm.sg-comp} dirty \textsc{cm.sg-def} {} {} \textsc{cm.pl-def}\\
				\glt    `Samba closed the window that is dirty.'\label{fon:kOnkmT2}
		\ex \gll	Samba tej-na palanteer \minsp{[} y-u tilim] y-i /\hspace{35pt} \minsp{*} b-i.\\
				Samba close-\textsc{na.3sg} window {} \textsc{cm.pl-comp} dirty  \textsc{cm.pl-def} {} {} \textsc{cm.sg-def}\\
				\glt    `Samba closed the windows that are dirty.'\label{fon:kOnkmT3}
		\z
	\z
	
\noindent   BNs can be modified by either a relative clause with either a singular (\ref{fon:kOnkmT4}) or a plural (\ref{fon:kOnkmT5}) class marker.\footnote{At least in the Wolof dialect investigated in this paper, the relative complementizer \textit{-u} (and the class marker prefixed to it) can occur with overt determiners (of both the definite and indefinite varieties), which are placed outside of the relative clause. This is the reason why I consider (\ref{fon:kOnkmT2}) and (\ref{fon:kOnkmT3}) to be instances of BNs modified by a relative clause.}
	
	
		\ea
				\ea \gll	Samba tej-na {palanteer} \minsp{[} b-u tilim].\\
					Samba close-\textsc{na.3sg} window {} \textsc{cm.sg-comp} dirty\\
					\glt    `Samba closed some window that is dirty.'\label{fon:kOnkmT4}
				\ex \gll	Samba tej-na {palanteer} \minsp{[} y-u tilim].\\
					Samba close-\textsc{na.3sg} window {} \textsc{cm.pl-comp} dirty\\
					\glt    `Samba closed some windows that are dirty.'\label{fon:kOnkmT5}
        		\z
	    \z

	
\noindent What we saw in the previous section is that BNs are singular. We also saw that they behave like a singular full DP. We may ask then how they can be able to be modified by a relative clause with a plural class marker (\textit{y}, (\ref{fon:kOnkmT5})), while their singular full DP counterpart cannot (\ref{fon:kOnkmT2}). In fact, the behavior of BNs now resembles that of plural DPs (\ref{fon:kOnkmT3}). We may ask additionally if BNs modified by a plural relative clause may behave like full plural DPs in other aspects as well. In this section, we will see that the answer to this question is positive.

Specifically, the data below show us that a BN modified by a plural relative clause (i.e., a relative clause which contains a plural class marker like \textit{y} prefixed to the complementizer) behaves like its plural full nominal counterpart: the BN can now saturate a collective predicate, as well as act as the antecedent of a plural pronoun, reciprocal, and plural reflexive.
	
	\ea \label{fon:bNnGaThHrYoOOwl}
	    \ea[*]{\gll	J\`{a}ngalekat b-i dajeele-na {xale} \minsp{[} b-u Samba xam] ci bayaal b-i.\\
						teacher \textsc{cm.sg-def} gather-\textsc{na.3sg} child {} \textsc{cm.sg-comp} Samba know \textsc{prep} park \textsc{cm.sg-def}\\
						\glt    Intended: `The teacher gathered child who Samba knows in the park.'}\label{fon:bNnGaThHrBOo}
					\ex[]{\gll	J\`{a}ngalekat b-i dajeele-na {xale} \minsp{[} y-u Samba xam] ci bayaal b-i.\\
						teacher \textsc{cm.sg-def} gather-\textsc{na.3sg} child {} \textsc{cm.pl-comp} Samba know \textsc{prep} park \textsc{cm.sg-def}\\
						\glt    `The teacher gathered some children who Samba knows in the park.'}\label{fon:bNnGaThHrYoO}
			\z
		\z
		
		
		
		
			\ea
					\ea[]{\gll	Gis-na-a {j\`{a}ngalekat} \minsp{[} b-u Roxaya xam]. Maymuna b\"{e}gg-na ko / \minsp{*} leen.\\
						see-\textsc{na-1sg} teacher {} \textsc{cm.sg-comp} Roxaya know Maymuna like-\textsc{na.3sg} \textsc{obj.3sg} {} {} \textsc{obj.3pl}\\
						\glt    `I saw a teacher who Roxaya knows. Maymuna admires her.'}
					\ex[]{\gll	Gis-na-a {j\`{a}ngalekat} \minsp{[} y-u Roxaya xam]. Maymuna b\"{e}gg-na \minsp{*} ko / leen.\\
						see-\textsc{na-1sg} teacher {} \textsc{cm.pl-comp} Roxaya know Maymuna like-\textsc{na.3sg} {} \textsc{obj.3sg} {} \textsc{obj.3pl}\\
						\glt    `I saw some teachers who Roxaya knows. Maymuna admires them.'}
			\z
		\z
		
			\ea
					\ea[*]{\gll	J\`{a}ngalekat b-i wanale-na {ndonggo{ }darra}\hspace{2cm} \minsp{[} b-u Mareem xam] \~{n}u xam-ante.\\
						teacher \textsc{cm.sg-def} introduce-\textsc{na.3sg} student {} \textsc{cm.sg-comp} Mareem know \textsc{3pl} know-\textsc{recip}\\
						\glt    Intended: `The teacher introduced student that Mareem knows to each other.'}\label{fon:waNAleNoPe7}
					\ex[]{\gll	J\`{a}ngalekat b-i wanale-na {ndonggo{ }darra}\hspace{2cm} \minsp{[} y-u Mareem xam] \~{n}u xam-ante.\\
						teacher \textsc{cm.sg-def} introduce-\textsc{na.3sg} student {} \textsc{cm.pl-comp} Mareem know \textsc{3pl} know-\textsc{recip}\\
						\glt    `The teacher introduced student that Mareem knows to each other.'}\label{fon:waNAleNoPe6}
			\z
		\z
		
	\ea
	    \ea[*]{\gll	J\`{a}ngalekat b-i sang-oloo-na {ndonggo{ }darra}\hspace{1.8cm} \minsp{[} b-u njool] seen bopp.\\
						teacher \textsc{cm.sg-def} wash-\textsc{caus}-\textsc{na.3sg} student {} \textsc{cm.sg-comp} tall \textsc{poss.3pl} head\\
						\glt    Intended: `The teacher made student who is tall wash themselves.'}\label{fon:kOzARiVv6GaPpe}
					\ex[]{\gll	J\`{a}ngalekat b-i sang-oloo-na {ndonggo{ }darra}\hspace{1.8cm} \minsp{[} y-u njool] seen bopp.\\
						teacher \textsc{cm.sg-def} wash-\textsc{caus}-\textsc{na.3sg} student {} \textsc{cm.pl-comp} tall \textsc{poss.3pl} head\\
						\glt    `The teacher made some tall students wash themselves.'}\label{fon:kOzARiVv6}
			\z
		\z
		
\noindent In sum, in \sectref{fon:sEk:BnNSgG}, we had concluded that BNs in Wolof behave as if they were singular. In this section, however, we see that this generalization has to be relativized to unmodified BNs only, since BNs modified by a plural relative clause behave is if they were plural. In the next section, we will see that nominal modifiers that do not have the syntax of a relative clause do not have this effect on the number interpretation of BNs.

\subsection{Plain modifier}
\label{fon:sec:BreZ}

In Wolof, nominal modifiers usually have the syntax of relative clauses (e.g. \textit{tall} in (\ref{fon:kOzARiVv6})). Expressions for nationality, however, occur as plain modifiers (i.e., without the syntax of a relative clause.)
		
		\ea \gll	Mareem dajeele-na a-y woykat \dashuline{brezilien}.\\
				Mareem gather-\textsc{na.3sg} \textsc{indef-cm.pl} singer Brazilian\\
				\glt `Mareem gathered some Brazilian singers.'\label{fon:gEntToRrRSeAa3}
		\z
		
	
	
	\noindent In this section, we will examine the behavior of BNs when modified by a plain modifier. We will see that they retain the singular construal exhibitted by unmodified BNs (cf. \sectref{fon:sEk:BnNSgG}), contrasting with BNs modified by a plural relative clause (cf. \sectref{fon:sEk:ReLl}). More precisely, a BN combined with a plain modifier cannot saturate a collective predicate, nor can it be the antecedent of plural discourse anaphora, a reciprocal, or plural reflexive.
	
		
		
			\ea[*]{\gll	Roxaya dajeele-na {f\'{e}cckat} \dashuline{brezilien}.\\
				Roxaya gather-\textsc{na.3sg} dancer Brazilian\\
				\glt Intended: `Roxaya gathered Brazilian student.'}\label{fon:pLEjNBrZl}
		\z
		
		\ea \gll Gis na-a {woykat} \dashuline{brezilien}. Maymuna b\"{e}gg na ko / \minsp{*} leen.\\
				see \textsc{na-1sg} dancer Brazilian Maymuna like \textsc{na.3sg} \textsc{obj.3sg} {} {} \textsc{obj.3pl}\\
				\glt `I saw a Brazilian dancer. Maymuna admires her/.'
		\z
	
	\ea[*]{\gll	J\`{a}ngalekat b-i desin-ante-loo-na {ndonggo darra} \dashuline{brezilien}.\\
				teacher \textsc{cm.sg-def} draw-\textsc{recip-caus-na.3sg} student Brazilian\\
				\glt    Intended: `The teacher made student draw each other.'}\label{fon:bnPlayNReSiPrr}
	\z
	
	\ea[??]{\gll   J\`{a}ngalekat b-i nataal-oo-na {ndonggo darra} \dashuline{angale} seen bopp.\\
            teacher \textsc{cm.sg-def} draw-\textsc{caus-na.3sg} student English \textsc{poss.3pl} head\\
            \glt    Intended: `The teacher made English student draw themselves.'}\label{fon:bnPlayNReSiPrr444}
	\z
	
		
\noindent In view of the data examined so far, we may ask the following questions:


	\eanoraggedright \label{fon:qQq2}
	    \eanoraggedright Why does an unmodified BN behave as if it were singular, while a BN modified by a plural relative clause behaves as if it were plural?
				\ex{Why does adding a plain (i.e. number-less) nominal modifier not have the same effect?}
	\z
	\z

\section{Towards an analysis}
\label{fon:sec:analysis}

In this section, I will develop an analysis that attempts to address the questions in (\ref{fon:qQq2}). Before that though, I will consider alternative analyses.

\subsection{Other plausible analyses}

BNs in Wolof do display some of the telltale properties of \textsc{pseudo noun incorporation} (PNI; \citealt{massam2001pseudo,dayal2011hindi,baker2014pseudo}). First, they allow for noun modification, as seen in the two previous sections. Second, there cannot be a low adverb intervening between the verb and its affixes and the BN object.
		
			\ea
					\ea{\gll	J\`{a}ngalekat b-i j\`{a}ng-na \minsp{\{} cikaw\} taalif b-i \minsp{\{} cikaw\}.\\
						teacher \textsc{cm.sg-def} read-\textsc{na.3sg} {} loudly poem \textsc{cm.sg-def} {} loudly\\
						\glt    `The teacher read the poem loudly.'}\label{fon:laUdPoeMm1}
					\ex{\gll	J\`{a}ngalekat b-i j\`{a}ng-na \minsp{\{} \minsp{*} cikaw\} {taalif} \minsp{\{} cikaw\}.\\
						teacher \textsc{cm.sg-def} read-\textsc{na.3sg} {} {} loudly poem {} loudly\\
						\glt    `The teacher read a poem loudly.'}\label{fon:laUdPoeMm2}
			\z\label{fon:laUdPoeMm}
		\z

\noindent   A PNI analysis could thus be applicable. However, syntactic PNI analyses often capitalize on the inability of the BN to move \citep{massam2001pseudo}, their consequences to linearization \citep{baker2014pseudo}, or their licensing requirements (\citealt{levin2015licensing}). This does not seem sufficient to account for the singular \textit{interpretation} of Wolof BNs.
	
This brings us to \citeauthor{dayal2011hindi}'s (\citeyear{dayal2011hindi}) semantic analysis of PNI in Hindi. \citeauthor{dayal2011hindi} remarks that BNs in Hindi are not number-neutral, but rather singular. The author proposes that the plural interpretation arises as a byproduct of a pluractional operator that applies at the sentential level and which is introduced by aspect.
		
		
			\ea
					\ea[]{\gll anu-ne [ tiin ghanTe \uline{meN} ] / [ tiin ghanTe \uline{tak} ] {kitaab} paRhii.\\ 
						Anu-\textsc{erg} {} 3 hours in {} {} {} 3 hours for {} book read.\textsc{pfv}\\}\label{fon:dayLlasPkTt1}
						
						    \ea{`Anu read a book \uline{in} three hours.' \hfill ($=$ exactly one book)}
							\ex{`Anu read a book \uline{for} three hours.' \hfill ($=$ one  or more books)}
					    \z
				
					\ex[]{\gll anu-ne [ tiin ghanTe \uline{meN} ] / \minsp{*} [ tiin ghanTe \uline{tak} ] {kitaab} paRh Daalii.\\ 
						Anu-\textsc{erg} {} 3 hours in {} {} {} {} 3 hours for {} book read \textsc{compl.pfv}\\
						\glt    `Anu read a book \uline{in} three hours.'\hfill ($=$ exactly one book)}\label{fon:dayLlasPkTt2}
					\sn[]{\hfill    \citep[(32); adapted]{dayal2011hindi}}
			\z\label{fon:dayLlasPkTt3Ol}
		\z
		
\noindent   (\ref{fon:dayLlasPkTt1}) shows that the number interpretation of the BN \textit{kitaab} `book' depends on the telicity of the predicate. The temporal adverb \textit{tiin ghanTe meN} `in three hours' picks out the telic reading of the predicate. In that case, the BN has an exclusively singular interpretation. It is only when an atelic reading is singled out (in (\ref{fon:dayLlasPkTt1}), by using \textit{tiin ghanTe tak} `for three hours') that the number-neutral interpretation of the BN arises. To drive the point home, in (\ref{fon:dayLlasPkTt2}), the atelic reading is eliminated via the addition of the completive particle \textit{Daalii}. As expected from the pattern observed in (\ref{fon:dayLlasPkTt1}), only a singular interpretation is available. Or, more relevantly for \citeauthor{dayal2011hindi}'s claim, a number-neutral interpretation becomes impossible.
			
In brief, the data in (\ref{fon:dayLlasPkTt3Ol}) demonstrate that the number interpretation of BNs in Hindi is correlated with the aspectual properties of the overall sentence where it is embedded. In order to account for this pattern, \citeauthor{dayal2011hindi} proposes that BNs in Hindi are singular, but aspect may introduce a pluractional operator that applies to the event the BN is a part of. The iterative interpretation of the event has as a byproduct a number neutral interpretation of the otherwise singular object BN.

While I do not have the same type of data as (\ref{fon:dayLlasPkTt3Ol}), existing Wolof data suggest that aspect does not play the same role as it does in Hindi. Aspectual information remains constant across the data investigated here and yet the number interpretation is different. A sample of the data examined in the previous section is repeated here for convenience.
		
		\ea
				\ea[*]{\gll	J\`{a}ngalekat b-i dajeele-na {xale} ci bayaal b-i.\\
					teacher \textsc{cm.sg-def} gather-\textsc{na.3sg} child \textsc{prep} park \textsc{cm.sg-def}\\
					\glt    Intended: `The teacher gathered child in the park.'}
				\ex[]{\gll	J\`{a}ngalekat b-i dajeele-na {xale} [ y-u Samba xam ] ci bayaal b-i.\\
							teacher \textsc{cm.sg-def} gather-\textsc{na.3sg} child {} \textsc{cm.pl-comp} Samba know {} \textsc{prep} park \textsc{cm.sg-def}\\
							\glt    `The teacher gathered some children who Samba knows in the park.'}
				\ex[*]{\gll	Roxaya dajeele-na {f\'{e}cckat} brezilien.\\
					Roxaya gather-\textsc{na.3sg} dancer Brazilian\\
					\glt    Intended: `Roxaya gathered Brazilian dancer.'}
			\z
		\z
		
\noindent   What does vary in these data is the presence or absence of modifier and type of modifier, irrespective of aspect (which, to reiterate, remains the same across the examples). The analysis to be put forward will capitalize on this property.\footnote{Needless to say, a more complete set of Wolof data would require changes in the aspectual properties of the sentence, as in the Hindi data.}
	
\subsection{Proposal}

A takeaway from the discussion of plausible analyses is that it appears that, while sentential material does not have an effect on the number interpretation of BNs in Wolof (unlike what happens in Hindi), modifiers do seem to have an effect. However, different modifiers have different effects. Plural relative clauses may render a BN plural, but plain modifiers do not. Thus, it seems feasible that the source of the number interpretation in Wolof BNs is nominal-internal.
	
The first step in the analysis is the proposal of a structure for full nominals, as it will be the basis for the structure proposed for BNs. The underlying assumption here is that BNs are a truncated version of the full nominals in a given language \citep{massam2001pseudo}. (Linear order was not taken into account.)
	
	
	\pagebreak

\begin{figure}[h]
\centering
\begin{tikzpicture}[parent anchor=south,
			align=center,
			level distance=20pt,
			anchor=north,
			sibling distance=45pt,
			child anchor=north]
			\node	{DP}
			child {node (D) {D\\{\small	[CM:\textunderscore\textunderscore]}}}
			child {node {NumP}
				child {node {Num}}
				child {node {\textit{n}P}
					child {node (n) {\textit{n}\\{\small	[CM:$\beta$]}}}
					child {node {√\textsc{\textit{xale}}}}}}
			;
			\draw[dashed] (D.south) .. controls +(south:5em) and +(south:3em) .. (n.south);
			\end{tikzpicture}
\caption{Structure proposed for a full nominal}
\label{fon:bzKTrEeEinDef}
\end{figure}
	
%		\ea{\leavevmode\vadjust{\vspace{-\baselineskip}}\newline 
%			\begin{tikzpicture}[parent anchor=south,
%			align=center,
%			level distance=20pt,
%			anchor=north,
%			sibling distance=45pt,
%			child anchor=north]
%			\node	{DP}
%			child {node (D) {D\\{\small	[CM:\textunderscore\textunderscore]}}}
%			child {node {NumP}
%				child {node {Num}}
%				child {node {\textit{n}P}
%					child {node (n) {\textit{n}\\{\small	[CM:$\beta$]}}}
%					child {node {√\textsc{\textit{xale}}}}}}
%			;
%			\draw[dashed] (D.south) .. controls +(south:5em) and +(south:3em) .. (n.south);
%			\end{tikzpicture}}\label{fon:bzKTrEeEinDef}
%	\z
	
\noindent   Following \citet{kihm2005noun} and \citet{acquaviva2009roots}, I assume that idiosyncratic properties the Wolof class marker are represented at the categorizer \textit{n}. Inspired by \citeauthor{torrence2013clause}'s (\citeyear{torrence2013clause}) take on the class marker that appears on relative clauses (\sectref{fon:sEk:ReLl}) as an instance of complementizer agreement, I assume that the class marker that appears in the determiner is an instance of D--\textit{n} agreement.
		
I further stipulate that the feature $[\textsc{Plural}]$ (though not $[\textsc{Singular}]$) Num must lower onto \textit{n}. As mentioned, number in nouns is only encoded in the class marker. In the pairs of nouns in \tabref{fon:pLSgGGaAWaA}, the shape of the first consonant of the noun changes according to its number. I take this to be a case of root allomorphy.\footnote{We could in principle posit a morphological boundary between the first mutating consonant and the rest of the word (e.g. \textit{mb-aam} and \textit{b-aam}) and analyze the first segment as a number morpheme and the rest of the word as the root. However, such roots do not seem to occur elsewhere in the language.} However, it is commonly assumed that allomorphy obeys a strict locality condition. Here, I assume \citeauthor{bobaljik2012universals}'s (\citeyear{bobaljik2012universals}) formulation, according to which allomorphy cannot affect nodes across a maximal projection.
				
				\begin{table}
				\caption{Consonant mutation in SG/PL pairs \citep{loporcaroBabou16}}
				\label{fon:pLSgGGaAWaA}
				\begin{tabular}{llll}
				    \lsptoprule
						& Singular & Plural & Translation\\
						\midrule
						a. & mbaam mi & baam yi & `the donkey/-s'\\
						b. & mbagg mi & wagg yi & `shoulder/-s'\\
						c. & pepp mi & fepp yi & `grain/-s'\\
						d. & këf ki & yëf yi & `thing/-s'\\
						e. & bët bi & gët yi & `eye/-s'\\
						f. & loxo bi & yoxo yi & `hand/-s, arm/-s'\\
						g. & waa ji & gaa ñi & `guy/-s'\\
					\lspbottomrule
				\end{tabular}
				\end{table}

Given this condition, Num in \figref{fon:bzKTrEeEinDef} could not trigger allomorphy in the class in \textit{n} across the maximal projection \textit{n}P. In order to sidestep this issue, I stipulate Num must lower \citep{embick2001movement} onto \textit{n}, as see in \figref{fig:FoOlNomFGgg}.
			
\begin{figure}
    \centering
    \begin{tikzpicture}[parent anchor=south,
					align=center,
					level distance=20pt,
					anchor=north,
					sibling distance=45pt,
					child anchor=north]
					\node	{DP}
					child {node {D}}
					child {node {NumP}
						child {node (num1) {\sout{Num}}}
						child {node {\textit{n}P}
							child {node {\textit{n}}
								[sibling distance=30pt]
								child {node (num2) {Num\\{\small [Num:\textsc{pl}]}}}
								child {node {\textit{n}}}}
							child {node {√\textsc{\textit{xale}}}}}}
					;
					\draw[-latex] (num1.west) to [bend right=90] (num2.west);
					\end{tikzpicture}
    \caption{Structure for full nominal and Num to \textit{n} lowering}
    \label{fig:FoOlNomFGgg}
\end{figure}
			
I further assume that `what you see is what you get': all things equal, methodological concerns should prevent one from positing null, purely abstract nodes. I will thus try to propose a structure of BNs in Wolof that is based on the structure proposed for full nominals (\figref{fon:bzKTrEeEinDef}), but without projections that do not have morphological support. The bare minimum component of the structure is the root, otherwise we cannot capture the basic meaning of the BN. Moving on to \textit{n}P, given the proposal above that Wolof class markers are the exponent of the categorizer \textit{n} and the `what you see is what you get' assumption, because there is no class marker in BNs, I assume they do not project an \textit{n}P. A desideratum is that we model the singular (not number-neutral) interpretation of BNs in Wolof. Following \citet{ritter1991} and \citet{harbour2011valence}, I assume that the only interpretable $[\mbox{Number}]$ feature is the one placed in NumP. DP may have unvalued $\varphi$-features (\citealt{harbour2011valence} and references therein), including $[\mbox{Number}]$. These features are, nonetheless, assumed to be purely syntactic (they participate in agreement with DP-external probes); they play no role at LF. I propose thus that BNs have a NumP projection. Finally, I will remain agnostic as to whether BNs have a silent DP projection or if they lack a DP layer altogether. As far as I can tell, the presence or absence of such a DP plays no role in the present analysis. For convenience, I omit the representation of a DP layer in the diagrams to follow.

Hence, we arrive at structure in \figref{fon:sTrkTBnbN}:
		
\begin{figure}
    \centering
    \begin{tikzpicture}[parent anchor=south,
				align=center,
				level distance=20pt,
				anchor=north,
				sibling distance=45pt,
				child anchor=north]
				\node	{NumP}
				child {node {Num}}
				child {node {√\textsc{\textit{xale}}}}
				;
				\end{tikzpicture}
    \caption{Truncated structure proposed for BNs in Wolof}
    \label{fon:sTrkTBnbN}
\end{figure}
		
A comment is in order on previous literature on the syntax of number neutrality. \citet{rullmannYou2006} and \citet{kramer2017general} investigate BNs in Mandarin and Amharic, respectively. In both languages, BNs are number neutral. \citeauthor{rullmannYou2006} and \citeauthor{kramer2017general} capture this semantic property by proposing that BNs lack NumP. A common assumption is that entities of type $e$ denote singleton sets (atoms) and their sums; what number does is restrict that denotation to only singleton sets (singular) or pluralities (plural). Under this view, number neutrality in BNs emerges as a consequence of the absence of a restriction that picks out just atoms or pluralities. Because BNs in Wolof are exclusively singular, the same bare syntactic structure will not work. Adopting the rather common assumptions mentioned above about number, a structure like that in \figref{fon:sTrkTBnbN} may gain further traction: it contains a bare minimum of structure; the functional layer that it does contain is able to restrict the number interpretation of the nominal.

However, \figref{fon:sTrkTBnbN} alone is consistent with a singular or plural restriction. This overgenerates, as BNs in Wolof are exclusively singular (when unmodified).
	
\subsubsection{Singular interpretation of unmodified BN}
\label{fon:seK:anAlzS}


	To recall, BNs in Wolof are singular, even though BNs in other languages are number neutral. The addition of different types of nominal modifiers has, correspondingly, different effects. If we add a modifier with a plural class marker, the BN behaves as if it were plural. A relative clause is this type of modifier. In contrast, if the nominal modifier lacks number morphology, the BN is still singular. Plain adjectives that name nationalities are this type of modifier.
			
Wolof clearly has full nominals that have a plural interpretation (\textit{xale y-i} `the children' in (\ref{fon:foOlNmNlBzKk})). Assuming that the only interpretable instance of $[\mbox{Number}]$ is in NumP, it must be the case that Wolof has a plural Num. All things equal, this instance of Num should be available for BNs as well. However, under the stipulation that plural Num must lower to \textit{n}, the derivation that builds \figref{fon:dErIvVBnSgPl1} fails because this requirement cannot be fulfilled. (\ref{fon:foOlNmNlBzKk}) also shows that Wolof should have a singular Num available too, which should also be available in building a BN. By stipulation, a singular Num does not have a lowering requirement to fulfill. As such, the derivation that builds \figref{fon:dErIvVBnSgPl2} can converge.

\begin{figure}[h]
\RawFloats
\centering
\begin{minipage}[b]{0.49\textwidth}
\centering
\begin{tikzpicture}[parent anchor=south,
					align=center,
					level distance=20pt,
					anchor=north,
					sibling distance=45pt,
					child anchor=north]
					\node	{NumP}
					child {node (num1) {Num\\\textsc{[pl]}}}
					child {node (num2) {√\\/\textsc{\textit{xale}}/}}
					;
					\draw[-latex] (num1.south) to [bend right=90] node [anchor=center] {\ding{55}} (num2.south);
					\end{tikzpicture}
\caption{Plural Num cannot lower to \textit{n} in BN}
\label{fon:dErIvVBnSgPl1}
\end{minipage}
\begin{minipage}[b]{0.49\textwidth}
\centering
\begin{tikzpicture}[parent anchor=south,
					align=center,
					level distance=20pt,
					anchor=north,
					sibling distance=45pt,
					child anchor=north]
					\node	{NumP}
					child {node (num1) {Num\\\textsc{[sg]}}}
					child {node (num2) {√\\/\textsc{\textit{xale}}/}}
					;
					\end{tikzpicture}
\caption{No lowering requirement}
\label{fon:dErIvVBnSgPl2}
\end{minipage}
\end{figure}

			
% \begin{multicols}{2}
%    \ea[*]{\leavevmode\vadjust{\vspace{-\baselineskip}}\newline 
%					\begin{tikzpicture}[parent anchor=south,
%					align=center,
%					level distance=20pt,
%					anchor=north,
%					sibling distance=45pt,
%					child anchor=north]
%					\node	{NumP}
%					child {node (num1) {Num\\\textsc{[pl]}}}
%					child {node (num2) {√\\/\textsc{\textit{xale}}/}}
%					;
%					\draw[-latex] (num1.south) to [bend right=90] node [anchor=center] {\ding{55}} (num2.south);
%					\end{tikzpicture}}\label{fon:dErIvVBnSgPl1}
%			\z
%		
%		\columnbreak
%		
%	\ea{\leavevmode\vadjust{\vspace{-\baselineskip}}\newline 
%					\begin{tikzpicture}[parent anchor=south,
%					align=center,
%					level distance=20pt,
%					anchor=north,
%					sibling distance=45pt,
%					child anchor=north]
%					\node	{NumP}
%					child {node (num1) {Num\\\textsc{[sg]}}}
%					child {node (num2) {√\\/\textsc{\textit{xale}}/}}
%					;
%					\end{tikzpicture}}\label{fon:dErIvVBnSgPl2}
%			\z
% \end{multicols}
			
We are now in the position to answer the following question: why are unmodified BNs in Wolof interpreted in the singular? The reason is that this is the only possible convergent derivation (\figref{fon:dErIvVBnSgPl2}).

\subsubsection{Adding a nominal modifier}

To recall, if a plural relative clause is added to the BN, it can have a plural interpretation. Here, I introduce an auxiliary assumption: relative clauses require a bigger, more complex nominal structure.\footnote{I am grateful to an anonymous LAGB 2019 reviewer for this suggestion. I assume that the projection or not of an \textit{n}P layer does not affect the bareness of the BN. It is shown in \citet{fong2020} that BNs in Wolof behave uniformly whether or not they are modified by a relative clause. For instance, they are obligatorily narrow scope indefinites and cannot occur in the subject position of a finite clause, regardless of the presence of a relative clause.} A common assumption is that relative clauses are adjoined to NP, even in different relative clause analyses. Translated into the distributed morphology terms assumed here, this means that relative clauses are adjoined to \textit{n}P \citep{havenhill2016relative}.
 
I proposed that BNs in Wolof lack an \textit{n}P projection due to the lack of a class marker. As such, the presence of a relative clause adjoined to a BN in sentences like (\ref{fon:bNnGaThHrYoO}) implies the projection of an \textit{n}P -- otherwise, the relative could not have been adjoined. The structure for the BN in a sentence like (\ref{fon:bNnGaThHrYoO}), must thus include an \textit{n}P in order to accommodate the relative clause, as shown in \figref{fig:bnModfHeLKlAwZ}. I follow \citet{torrence2013clause} in assuming a raising analysis is appropriate for relative clauses in Wolof.
		
\begin{figure}
    \centering
    \begin{tikzpicture}[parent anchor=south,
				align=center,
				level distance=20pt,
				anchor=north,
				sibling distance=75pt,
				child anchor=north]
				\node	{NumP}
				child {node (num1) {\sout{Num}}}
				child {node {\textit{n}P}
					child {node {\textit{n}P}
						[sibling distance=35pt]
						child {node (num2) {Num\\{\small	[\textsc{pl}]}}}
						child {node {\textit{n}}
							child {node {\textit{n}}}
							child {node {√}}}}
					child {node {CP}
					[sibling distance=20pt]
					child {coordinate (trileft)}
					child {node {\small	(relative clause)} edge from parent[draw=none]}
					child {coordinate (triright)}}}
				;
				\draw[-latex] (num1.west) to [bend right=90] (num2.west);
				\draw (trileft) -- (triright);
				\end{tikzpicture}
    \caption{Complex structure for BNs modified by a relative clause}
    \label{fig:bnModfHeLKlAwZ}
\end{figure}

\noindent   As a byproduct of the projection of \textit{n}P, a plural Num can also be introduced in the derivation, as its lowering requirement can now be fulfilled.	

Conversely, why does a plain modifier not have the same effect? A way to account for the difference between full relative clauses and plain modifiers would be to assume that the latter do not need a more complex projection to adjoin to a nominal. Specifically, a \textit{n}P projection would not be required for an adjective like \textit{brezilien} `Brazilian' to occur. A BN thus modified can be diagrammed as in \figref{fon:pLEjNBrZlTrE}.

\begin{figure}[h]
\RawFloats
\centering
\begin{minipage}[b]{0.49\textwidth}
\centering
\begin{tikzpicture}[parent anchor=south,
				align=center,
				level distance=20pt,
				anchor=north,
				sibling distance=45pt,
				child anchor=north]
				\node	{NumP}
				child {node {Num}}
				child {node {√}
					child {node {√}}
					child {node {\textit{a}P}}}
				;
				\end{tikzpicture}
        \caption{BN modified by plain modifier}
        \label{fon:pLEjNBrZlTrE}
\end{minipage}
\begin{minipage}[b]{0.49\textwidth}
\centering
\begin{tikzpicture}[parent anchor=south,
				align=center,
				level distance=20pt,
				anchor=north,
				sibling distance=45pt,
				child anchor=north]
				\node	{NumP}
				child {node (num1) {Num\\\textsc{[pl]}}}
				child {node {√}
					child {node (num2) {√}}
					child {node {\textit{a}P}}}
				;
				\draw[->] (num1.south) .. controls +(south:3em) and +(south:3em) .. node [anchor=center] {\ding{55}} (num2.south);
				\end{tikzpicture}
        \caption{No Num to \textit{n} lowering}
        \label{fig:BrexApNoLowWw}
\end{minipage}
\end{figure}

		
\noindent   The absence of a plural reading is reduced to the same reason why unmodified BNs are exclusively singular: a plural NumP is in principle available in the language, but the derivation crashes because the plural Num cannot have its lowering requirement satisfied. This is schematized in \figref{fig:BrexApNoLowWw}.
	
	The analysis put forward gives rise to a prediction. A crucial ingredient in the analysis is the proposal that relative clauses and plain modifiers attach at different levels of the nominal structure, thus requiring different amounts of structure to be projected. Relative clauses require an \textit{n}P, while plain modifiers require a smaller, simpler structure, being attachable to the root. A common assumption is that the nominal spine has a hierarchical structure, with the \textit{n}P above the root. The prediction thus is that there can be a relative clause outside a plain modifier, since the former adjoins to a layer (\textit{n}P) that includes the layer where the latter is adjoined to (the root). Conversely, the reverse order should not be possible, since the relative clause at \textit{n}P should ``close off'' the domain where the plain modifier was supposed to be adjoined. The prediction is borne out by facts:
	
	\ea
	    \ea[]{\gll	Gis-na-a {ndonggo darra} \dashuline{brezilien} [\textsubscript{RC} b-u Samba xam].\\
				see-\textsc{na-1sg} student Brazilian {} \textsc{cm.sg-comp} Samba know\\
				\glt `I saw a Brazilian student who Samba knows.'}
			\ex[*]{\gll	Gis-na-a {ndonggo darra} [\textsubscript{RC} b-u Samba xam] \dashuline{brezilien}.\\
				see-\textsc{na-1sg} student {} \textsc{cm.sg-comp} Samba know Brazilian\\
				\glt  Intended: `I saw a Brazilian student who Samba knows.'}
	    \z
	\z

\section{Concluding remarks}

The goal of the present paper was to answer the following questions:

	
	
		\eanoraggedright \label{fon:qQq2444}
	    \eanoraggedright{Why does an unmodified BN behave as if it were singular, while a BN modified by a plural relative clause behaves as if it were plural?}
				\ex{Why does adding a plain (i.e. number-less) nominal modifier not have the same effect?}
	\z
	\z

\noindent   While both a singular and plural NumP are available in Wolof, only the former leads to a convergent derivation. This is caused by the stipulation that the plural Num must lower onto \textit{n}, combined with the assumption that BNs lack an \textit{n}P. The licensing of a relative clause implies the addition of an \textit{n}P, which in turn allows a plural Num to satisfy its lowering requirement. Plain modifiers, on the other hand, do not require a more complex nominal structure. In particular, \textit{n}P is not projected, so the plural Num cannot satisfy its requirement, just as in unmodified BNs.

As implied in \sectref{fon:sec:analysis}, a number of stipulations are made. Needless to say, further motivation must be provided to support these claims or, alternatively, the analysis should replace them with less stipulative components. Furthermore, aspect data must be elicited, in order to fully rule out an analysis like the one that \citet{dayal2011hindi} proposes for BNs in Hindi.

\section*{Abbreviations}

\begin{tabularx}{.5\textwidth}{@{}lX@{}}
\textsc{caus} & causative\\
\textsc{cm} & class marker\\
\textsc{comp} & complementizer\\
\textsc{def} & definite\\
\textsc{impf} & imperfective\\
\textsc{na} & \textit{na}, a sentential particle\\
\textsc{obj} & object\\
\end{tabularx}
\begin{tabularx}{.5\textwidth}{@{}lX@{}}
\textsc{pl} & plural\\
\textsc{poss} & possessive\\  
\textsc{prep} & preposition\\
\textsc{recip} & reciprocal\\
\textsc{refl} & reflexive\\
\textsc{sg} & singular\\
&\\
\end{tabularx}



\section*{Acknowledgements}
Many thanks L. Tour\'{e} for teaching me their language. This work would not be possible without them. Thank you also to P. Tang for her help. For discussion and criticism, I am also grateful to D. Fox, M. Hackl, S. Iatridou, M. Martinovi\'{c}, D. Pesetsky, N. Richards, R. Schwartzchild, and G. Thoms. Thank you also to F. Chen for sharing her Mandarin judgments with me and for useful comments.

{\sloppy\printbibliography[heading=subbibliography,notkeyword=this]}

\end{document}
