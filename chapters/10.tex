\documentclass[output=paper]{langscibook} 

\author{Nina Haslinger\affiliation{Georg-August-Universität Göttingen} and
Eva Rosina\affiliation{Universität Wien} and
Magdalena Roszkowski\affiliation{Central European University} and
Viola Schmitt\affiliation{Humboldt-Universität zu Berlin} and
Valerie Wurm\affiliation{Humboldt-Universität zu Berlin}}
\title{Cumulation cross-linguistically}  
\abstract{Semantic theories of cumulativity vary in several respects, including (i) whether cumulativity is limited to lexical predicates and (ii) whether there are cumulation operators in the object language. We address the cross-linguistic predictions of different settings of these two parameters and evaluate them in light of a preliminary set of data from 22 languages, largely collected from native-speaker linguists. We submit that cumulative readings of non-lexical predicates are available cross-linguistically. We then address the question whether there are overt morphemes that behave like the cumulation operators **, ***, etc. Our data only give a partial answer, since there are different ways of integrating such operators into the grammar. No language in our sample had overt markers that were required for a cumulative reading, but absent in case of a distributive reading. Assuming that the LFs of distributive readings do not have to contain such cumulation operators, our data set does not provide evidence for their existence. 

\keywords{plurals, cumulativity, cumulation operators, semantic typology}}

\begin{document}
\SetupAffiliations{output in groups = true}
\maketitle

\section{Introduction}\label{has-sch:sec:1}

English sentences containing two or more plural-denoting expressions -- like \textit{Abe and Bert}, \textit{(the) two cats} etc.~-- have a particular form of ``weak'' truth conditions (\citealt{Kroch:1974, Langendoen:1978, Scha:1981, Krifka:1986} a.o.). For instance, \REF{has-sch:1a} is true in scenario \REF{has-sch:1b}, where each boy fed only one of the cats.

\ea \label{has-sch:1}
\ea\label{has-sch:1a} The boys fed the two cats. 
\ex\label{has-sch:1b} \textsc{scenario}: Abe fed cat Ivo. Bert fed cat Joe. 
\z\z

\noindent Such truth conditions are known as \textsc{cumulativity}:\footnote{Some of the literature also applies the term cumulativity to a property of one-place predicates: the property of being closed under sum. We will not adopt this usage here: Throughout the paper, we take cumulativity to be a semantic relation between two or more plural expressions.} Properties of the individuals making up a plurality ``add up'' to properties of the entire plurality (\citealt{Link:1983, Krifka:1986, Sternefeld:1998} a.o.).\footnote{With non-upward-monotonic plural quantifiers like \textit{exactly two cats}, the cumulative reading is not necessarily weaker than the distributive one. This will become crucial in \sectref{has-sch:sec:3.3}.} While \REF{has-sch:1a} does not state that \sib{fed the two cats} holds of each boy, this property does hold of the plurality \sib{the boys} because the cats fed by the individual boys ``add up'' to two.

This paper addresses the question what the semantic mechanism behind these cumulative truth conditions is. Most of the existing literature concentrates on complex cases of cumulativity in English and German (e.g.,~\citealt{Schein:1993,Beck:2000a,Champollion:2010a,Schmitt:2019}). But the different accounts also make quite simple typological predictions that have received less attention. We will present data relevant to two typological issues on which the existing analyses arguably make different predictions: (i)~whether there is morphosyntactic evidence for the presence of \textsc{cumulation operators} and (ii)~whether cumulative readings of syntactically complex predicates are cross-linguistically common.

The paper is structured as follows: \sectref{has-sch:sec:2} introduces some theories of cumulativity and two dimensions along which they differ. \sectref{has-sch:sec:3} presents preliminary cross-linguistic data relevant to these parameters and discusses one of the few previous publications known to us that address predictions of theories of cumulativity in an understudied language, namely \citet{Beck:2012}.\footnote{We thank a reviewer for mentioning \citet{Henderson:2012} as another theoretical work discussing cumulativity and distributivity in an underrepresented language (see \sectref{has-sch:sec:3.4}).} \sectref{has-sch:sec:4} explores which theoretical picture the cross-linguistic situation suggests.

\section{Different types of theories of cumulativity}\label{has-sch:sec:2}


We start with a brief sketch of different ways of deriving the weak truth conditions of cumulative sentences (a partially similar overview is given in \citealt{Champollion:2015b}). One point of variation concerns the semantic primitives they require. While some accounts \citep{Scha:1981,Krifka:1986,Beck:2000a,Champollion:2010a} model cumulativity as a property of relations between individuals -- like \sib{fed} in \REF{has-sch:1a} -- or of higher-type plural objects based on individuals \citep{Schmitt:2019}, others derive it from the properties of thematic-role relations between individuals and events, so that it is inherently tied to event semantics (e.g.,~\citealt{Schein:1993, Landman:2000, Kratzer:2003, Ferreira:2005, Zweig:2008, Zweig:2009}). Our discussion here, however, will focus on two other parameters structuring the theoretical landscape. Our first parameter is whether cumulativity is always a property of \textit{lexical}  predicates of individuals:

\ea\label{has-sch:par1} \textsc{Parameter 1:} Does the theory permit non-lexical cumulative relations?\z

\noindent For illustration, consider first the paraphrase of sentences like \REF{has-sch:1a} in \REF{has-sch:4a}. 
 $\leq_a$ is the atomic-part relation.\footnote{Unless indicated otherwise, our discussion employs basic notions from plural semantics.  We assume a set $A \subseteq D_e$ of atomic individuals, a binary operation $+$ on $D_e$  (the sum operation mentioned above) and a function $f: (\mathcal{P}(A)\ \backslash\ \{\emptyset\}) \to D_e$ such that: 1) $f(\{x\}) = x$ for any $x \in A$ and 2) $f$ is an isomorphism between the structures $(\mathcal{P}(A)\ \backslash\ \{\emptyset\}, \cup)$ and $(D_e, +)$. We thus have a one-to-one correspondence between plural individuals and nonempty sets of atomic individuals. See \citet{Link:1983} and \citet{Champollion:2016} for a more detailed discussion.} 
 
 \ea\label{has-sch:4a} The boys fed the two cats. \\
`Every $x \leq_a$ \sib{the boys} fed at least one $y \leq_a$ \sib{the two cats} and every $y \leq_a$ \sib{the two cats} was fed by at least one $x \leq_a$ \sib{the boys}.'
\z
 
\noindent Such cases can be accounted for via meaning postulates on lexical predicates like \textit{feed} (see~\citealt{Scha:1981, Krifka:1986}). But \citet{Beck:2000a} show that a similar paraphrase exists for cases like \REF{has-sch:4b}, where \textit{the boys} and \textit{the two cats} are not co-arguments of a lexical predicate. The cumulation mechanism thus seems to target the relation $[\lambda x.\lambda y.y\ \text{wants to feed}\ x]$, which is not expressed by a surface constituent in \REF{has-sch:4b}.

\ea\label{has-sch:4b} The boys want [to feed the two cats]. \\
`For every $x \leq_a$ \sib{the boys}, there is at least one $y \leq_a$ \sib{the two cats} that $x$ wants to feed, and for every $y \leq_a$ \sib{the two cats}, there is at least one $x \leq_a$ \sib{the boys} that wants to feed $y$.'
\z

\noindent The second parameter is whether cumulativity is contributed by operators in the syntactic representation of cumulative sentences:

\ea\label{has-sch:par2} \textsc{Parameter 2:} Does the theory assume object-language cumulation operators? \z

\noindent This boils down to the question whether there is a silent morpheme (or set of silent morphemes) responsible for cumulation:\footnote{A reviewer asks why we use ``morpheme'' rather than ``operator''. Our choice relates to our assumption that operators present at LF are visible to morphology, addressed in \sectref{has-sch:sec:2.2}.} in \REF{has-sch:4a}, cumulativity of \textit{feed} could be due either to its lexical meaning or to a silent cumulation operator attaching to the lexical head \textit{feed} in the syntax. In the non-lexical case \REF{has-sch:4b}, this operator would have to attach to a derived LF constituent denoting the relation $[\lambda x.\lambda y.y\ \text{wants to feed}\ x]$ \citep{Beck:2000a}. To derive \REF{has-sch:4b} without such operators, cumulativity would have to be built directly into the rules for function-argument composition, as in \citet{Schmitt:2019} or in the event-based tradition (see \sectref{has-sch:sec:2.4} for a discussion of both these systems). These parameters yield four logical possibilities (to our knowledge only three of them have been explored), which differ in their typological consequences.

\subsection{No non-lexical cumulative relations, no cumulation operators}\label{has-sch:sec:2.1}

The assumption underlying most early work on cumulativity (e.g.,~\citealt{Scha:1981, Krifka:1986}) is that cumulativity is a property of relation-denoting lexical items and thus reflects the lexical meanings of predicates taking more than one argument. The extensions of lexical items denoting binary relations are assumed to be closed under a \textsc{pointwise sum} operation which, for any set of pairs in the relation, sums up all the first components and simultaneously all the second components.\footnote{Sentences with more than two plurals can also have weak truth conditions similar to those of \REF{has-sch:1a}. The theories sketched below differ wrt.~whether they predict different formal reflexes of cumulativity for binary predicates, ternary predicates etc. Since this interesting issue is beyond the scope of this work, we focus on cases with two plurals like \REF{has-sch:1a}.} This closure condition is illustrated for \textit{feed} in \REF{has-sch:ruule} (where `$\bigplus (S)$' stands for the sum of all elements in $S$).

\ea\label{has-sch:ruule} For all $S, S' \subseteq D_{e}$ such that for every $x' \in S$ there is a $y' \in S'$ s.t. \sib{feed}$(x')(y') = 1$ and for every $y' \in S'$ there is an $x' \in S$ such that \sib{feed}$(x')(y') = 1$, \sib{feed}$(\bigplus(S))(\bigplus(S'))= 1$. \z

\noindent It follows that if \sib{feed} is true of the pair ${\langle a, i\rangle}$ and the pair ${\langle b, j\rangle}$, it is also true of the ``pointwise sum'' of these pairs, ${\langle a+b, i+j\rangle}$. In general, the extension of \textit{feed} contains all pairs of individuals that we can form by simultaneously adding up feeders and their feedees. \REF{has-sch:6} gives a sample extension that meets this condition.

\ea\label{has-sch:6} \sib{feed} = $\{\langle a, i\rangle, \langle b, j\rangle, \langle b, k\rangle, \langle a+b, i+j\rangle, \langle a+b, i+k\rangle, \langle b, j+k\rangle, \langle a+b, i+j+k\rangle\}$ \z

\noindent In scenario \REF{has-sch:1b}, \sib{feed}$(\textsc{ivo})(\textsc{abe}) = 1$ and \sib{feed}$(\textsc{joe})(\textsc{bert}) = 1$, so we must also have \sib{feed}$(\textsc{ivo}+\textsc{joe})(\textsc{abe}+\textsc{bert}) = 1$, which correctly predicts that \REF{has-sch:1a} is true, assuming a structure where no additional operators are present.

\subsection{No non-lexical cumulative relations, cumulation operators}\label{has-sch:sec:2.2}

In \REF{has-sch:ruule}, the closure condition is encoded as a meaning postulate constraining possible extensions of \textit{feed}. But cumulative truth conditions could also be derived from a lexical predicate true of only those pairs where the feeding relation holds ``primitively'', as in \REF{has-sch:2a}, if it then is affixed with an operator performing closure under pointwise sum. \REF{has-sch:2b} defines such an operator, **, for binary predicates. 


\ea
\ea \label{has-sch:2a} \sib{feed} = $\{\langle a, i\rangle, \langle b, j\rangle, \langle b, k\rangle\}$ \hfill \sib{** feed} = \REF{has-sch:6}
\ex \label{has-sch:2b} For any $P \in  D_{\stb{e, \stb{e, t}}},$ \sib{**}$(P)$ is the smallest relation $R$ such that (i) for all $x, y \in D_{e}$, if $P(x)(y)$, then $R(x)(y)$  and (ii) for all $S, S' \subseteq D_{e}$ such that for every $x' \in S$ there is a $y' \in S'$ such that $R(x')(y')$ and for every $y' \in S'$ there is an $x' \in S$ such that $R(x')(y') $, $R(\bigplus(S))(\bigplus(S'))$.
\z\z

\noindent While this analysis follows the operator-less approach in taking cumulativity to reflect a property of binary predicates, this property is encoded in a separate expression attaching to the predicate, not in the predicate's lexical entry. If ** is constrained to apply to lexical predicates only,  we then expect to find cumulative readings in the same configurations in which the purely lexical analysis from \sectref{has-sch:sec:2.1} predicts them. But there is one respect in which predictions diverge: the operator-based approach leads us to expect that the ** operator should have overt counterparts in the morphology of at least some languages.\footnote{A reviewer asks whether, if it were the case that we found morphological reflexes of cumulativity, they could be (semantically vacuous) syntactic agreement markers which indicate that the lexical predicate is cumulative, rather than realizations of **. If such agreement existed, we would indeed expect it to play a role in the morphology of at least some languages. But in order to test whether a morphological marker associated with cumulativity is a realization of ** or an agreement marker on a lexically cumulative predicate, we would arguably need configurations where ** applies to something other than the lexical predicate, i.e.~cumulation of complex predicates (see \sectref{has-sch:sec:2.3}). So within a theory in which only lexical predicates can be cumulated, we cannot distinguish these two hypotheses.} The fact that it can be spelled out as zero in English would be purely accidental. On the other hand, if the operator-less theory (\sectref{has-sch:sec:2.1}) had cross-linguistic validity, we would not expect other languages to have overt morphemes marking cumulativity.\footnote{As the terms 
``cumulativity'' and ``cumulation operators'' are not used in a uniform way in the literature, we should clarify that we are only concerned with cumulative relations between two or more plurals. The term ``cumulativity'' is often also applied to a property of unary predicates: being closed under sum. Consequently, the operator \REF{has-sch:onestar}, which closes a set under sum, is called a cumulation operator by several authors, e.g.,~\citet{Sternefeld:1998}.

\ea \label{has-sch:onestar}\sib{*}$(P)$ is the smallest set $S$ such that $P \subset S$ and for any $S' \subseteq S, \bigplus(S') \in S$.\z

\noindent We will not address the question if there are morphosyntactic counterparts of *, except to note that there are several plausible candidates for them, like nominal plural morphology \citep{Sternefeld:1998} or pluractional morphology in an event-based semantics (see \sectref{has-sch:sec:3.4}).}

Let us clarify why, given the operator-based approach, we would predict the operator to be visible in some languages. In line with much work on syntactic and semantic typology (see, e.g.,~\citealt{Matthewson:2001, Bobaljik:2012}), we make two general assumptions  that our entire discussion here is based on: first, we assume that operators present at the syntactic level that is visible to semantics are also visible to the morphological component of the language, which means that we expect a correlation between LF complexity and morpho-syntactic complexity. While the other option -- that LF operations are not visible to the morphological system -- is not ruled out \textit{per se}, it would seem to render the whole body of work that tries to probe LF complexity via morpho-syntactic markedness potentially vacuous (and would  raise the question of how else to account for the typological gaps reported by \citealt{Bobaljik:2012} or also our own work). We discuss this issue at length in \citet{Flor:2017}. 

Our second assumption is that morphemes visible to the syntax should occur overtly in at least some languages -- which is to say that we assume that there are no morphemes whose phonological exponent is null obligatorily, in all languages. This assumption is based on what could be considered ``reasons of economy'': we don't want to postulate material for which we find no grammatical indication.\footnote{A reviewer mentions indices (as used in \citealt{Heim:1998}) as an element of LF syntax that is obligatorily silent, i.e. does not have any phonological representation. However, first of all, this particular assumption about indices has been subjected to substantial criticism (see,~e.g.,~\citealt{Jacobson:1999}). Second, there is linguistic work that aims to find overt reflections of indices (and other ``logical variables'') and claims that they are in fact found in sign languages like the American Sign Language (ASL; see \citealt{Schlenker:2018} for an overview). The objective of such research is analogous to that of this paper: to look for morphosyntactic evidence for material postulated to be present at LF.}



\subsection{Non-lexical cumulative relations, cumulation operators}\label{has-sch:sec:2.3}

The main reason why several authors posit cumulation operators for English relates to Parameter 1 -- non-lexical cumulation, as described in \REF{has-sch:par1}. Under both theories discussed so far, cumulative truth conditions arise only if the plural expressions are co-arguments of a lexical predicate. In English, there are counterexamples to this claim \citep{Beck:2000a}. Consider \REF{has-sch:7}:

\ea\label{has-sch:7}
\ea\label{has-sch:7a} The two boys wanted to feed the two cats. \\  \hfill (adapted from \citealt{Beck:2000a}) \label{has-sch:3a}
\ex\label{has-sch:7b} \textsc{scenario:} Abe wanted to feed Ivo. Bert wanted to feed Joe. \label{has-sch:3b}
\ex\label{has-sch:7c} required relation: $\lambda x_{e}. \lambda y_{e}. y$ wanted to feed $x$
\ex\label{has-sch:7d} LF: [[the two boys] [[the two cats] [** [2 [1 [$t_{1}$ wanted to feed $t_{2}$ ]]]]]] \z\z

\noindent \REF{has-sch:7a} has cumulative truth conditions of the kind paraphrased in \REF{has-sch:4b} --   so \REF{has-sch:7a} is true in scenario \REF{has-sch:7b} -- but the cumulative relation needed to derive this, \REF{has-sch:7c}, is not expressed by a lexical item or even a surface constituent. \citet{Beck:2000a} propose that in such cases, covert ``tucking in''  movement derives an LF constituent denoting this relation, which is then affixed with the ** operator from \REF{has-sch:2b}. So \citet{Beck:2000a} and the approach in \sectref{has-sch:sec:2.2} both use cumulation operators, but differ w.r.t.~their status: for \citet{Beck:2000a}, they are not part of a lexical decomposition of certain predicates, but can apply to any relational expression derivable by syntactic processes.\footnote{In principle, both lexical and syntactic cumulation could be available cross-linguistically. If so, we would expect there to be languages in which non-lexical cumulation requires a certain marker, while lexical cumulation does not. Moreover,
languages could then also differ in how they encode cumulativity, i.e.~there might be some languages which are restricted to lexical cumulation.  While neither of these possibilities can be  ruled out by the data sets we present in \sectref{has-sch:sec:3.1} and \sectref{has-sch:sec:3.3} below, these data do not provide support for either of them. In particular, our  data on non-lexical cumulation  in \sectref{has-sch:sec:3.1}  do not provide evidence that some languages lack non-lexical cumulation or associate it with special morphosyntactic marking.}

What are the typological predictions of this theory? First, we would not expect languages where cumulativity is restricted to lexical predicates. Second, as the theory relies on cumulation operators, we might expect to find overt morphemes expressing ** in some languages. The latter prediction is not entirely obvious: if ** is merged after covert movement of the plurals, as \REF{has-sch:7d} suggests, its insertion should  have no effect on the PF side. But given our underlying assumption that morpho-syntactic markedness patterns are informative about LF complexity, laid out in \sectref{has-sch:sec:2.2}, it would be undesirable to posit an operator that \textit{cannot} be merged in the overt part of the derivation and thus never has morphological effects.\footnote{There are also other ways of distinguishing between a theory where the operator is always silent and the operator-less theories discussed in \sectref{has-sch:sec:2.4}. \citet{Schmitt:2019} argues that operator-based approaches cannot derive the right truth-conditions for cases like \REF{has-sch:gene}. In this example, it seems that the predicate conjunction and \sib{the two dogs} can both receive a cumulative reading relative to \sib{the two boys}, although \textit{the two dogs} is contained within the predicate conjunction. A cross-linguistic look at cases like \REF{has-sch:gene} would therefore be relevant.

\ea \label{has-sch:gene} The two boys made Gene [$_{P}$ feed the two dogs][ and [$_{Q}$ brush the hamster].
\z
	
\noindent Since \citet{Beck:2000a} derive complex cumulative relations via covert movement, it seems that island constraints on cumulativity could provide a further way of disentangling these two theories. Yet, \citet{Schmitt:2019} notes an operator-based approach would not necessarily predict island effects, so the absence of such effects would be compatible with both theories.\label{has-sch:fn-flattening}} Some languages should then overtly realize  ** if the relational expression it modifies is a constituent at both PF and LF.
Further, alternative implementations would lead us to expect such marking even if the modified expression is only an LF constituent, as in  \REF{has-sch:7d}: the operator could be merged before covert movement occurs, stranding the indices below it, or else we could appeal to ``post-cyclic'' merge of overt material  \citep{Fox:1999}. 

\subsection{Non-lexical cumulative relations, no cumulation operators}\label{has-sch:sec:2.4}

The fourth type of analysis is also motivated by non-lexical cases of cumulativity like \REF{has-sch:7a}, but differs more fundamentally from the lexical approaches. Cumulativity is not due to any particular constituent of cumulative sentences, but built into the basic mechanism that combines lexical predicates with their arguments. This allows these systems to account for non-lexical cumulation while interpreting all plurals \textit{in situ}. In this section, we outline two theories of this kind -- the \textsc{plural projection} system from \citet{Schmitt:2019} and \citet{Haslinger:2018a} and a class of theories under which cumulativity is a property of thematic-role relations (\citealt{Schein:1993,Landman:1996,Landman:2000,Kratzer:2003,Kratzer:2008,Ferreira:2005,Zweig:2008,Zweig:2009}).\footnote{A reviewer asks whether we take \citet{Sternefeld:1998} to be another theory of this type. Sternefeld uses the notion of 
``semantic glue'' -- operators that may be inserted more or less freely at LF, and would thus not influence surface syntax. Yet, he suggests that the pluralization operator for \textit{unary} predicates, *, plays ``a double role, namely as the semantic interpretation of plural nominal morphology on the one hand, and as freely insertible glue elsewhere in the system, on the other'' \citep[314, fn.~7]{Sternefeld:1998}. Since his theory does not rule out a similar ``double role'' for **, we consider it to be a theory with syntactic cumulation operators.}

\subsubsection{Plural projection}

The plural projection framework relies on the nonstandard ontological assumption that all semantic domains contain pluralities: there are not only pluralities of individuals, but also pluralities of predicates or propositions. We then have semantically plural expressions associated with any type $a$. Any such plural expression denotes a set of expressions whose elements are pluralities of type $a$, rather than a single plurality of type $a$, for reasons clarified below.\footnote{\citet{Haslinger:2018a} introduce a special type $a^*$ of `plural sets' with elements of type $a$, which is technically distinct from type $\stb{a, t}$, but has a domain with the same structure (up to isomorphism) as type $\stb{a, t}$. We suppress this distinction in the main text since it is not crucial to our purposes in this paper.} 
For example, \textit{the two cats} denotes a set containing the sum of the two cats \REF{has-sch:9a}. Since pluralities are then available throughout the type system, semantic plurality can be treated as a property that, by default, ``projects'' from a node to its mother: Standard plurals like \textit{Abe and Bert} or \textit{the cats} denote sets of pluralities -- but so do larger expressions containing them, like \textit{fed the two cats}, which denotes a set containing the sum of two properties (in our scenario, feeding Ivo and feeding Joe) \REF{has-sch:9c}. Similarly, the VP in \REF{has-sch:9d} denotes a set containing the sum of two properties -- the property of wanting to feed Ivo and that of wanting to feed Joe.

\ea
\ea\label{has-sch:9a} \sib{the boys} = $\{\textsc{abe}+\textsc{bert}\}$, \sib{the two cats} = $\{\textsc{ivo}+\textsc{joe}\}$
\ex\label{has-sch:9c} \sib{fed the two cats} = $\{(\lambda x.\textsc{fed}(\textsc{ivo})(x))+(\lambda x.\textsc{fed}(\textsc{joe})(x))\}$ 
\ex\label{has-sch:9d} \sib{want to feed the two cats} = $\{(\lambda x.\textsc{want}(\textsc{feed}(\textsc{ivo})(x))(x))+(\lambda x.\textsc{want}(\textsc{feed}(\textsc{joe})(x))(x))\}$ \z
\z

\noindent The top row of \figref{has-sch:fig:pp} illustrates the general principle behind this ``projection'' mechanism: to combine a non-plural functor with a plural argument, we apply it to each atomic part of the argument and sum up the results. The case where the functor, but not the argument is plural is similar. Cumulative sentences always involve configurations where a set of pluralities of a functional type combines with a set of pluralities of a matching argument type. The weak semantics associated with cumulativity results from the behavior of the projection rule for such cases. The mother node will denote the set of value pluralities that can be formed by picking a functor plurality and an argument plurality, applying atomic function parts to atomic argument parts in such a way that each atomic part of the function and each atomic part of the argument is used at least once, and summing up the results. (See \citealt{Haslinger:2018a} for a fully compositional definition of this rule, and \citealt{Haslinger:2018b} for a discussion of its relation to the **-operator.) The plural set derived in the bottom row of \figref{has-sch:fig:pp} contains $f(a)+g(b)$ as this can be derived using each of the function parts $f$ and $g$, and each of the argument parts $a$ and $b$, but it cannot contain, e.g.,~$g(a)+g(b)$.

\begin{figure}[h]
\centering
    \begin{forest}
    for tree={s sep=1cm, inner sep=0, l=0}
    [{$\{f(a)+f(b)\}$} [$\{f\}$] [$\{a+b\}$] ] 
    \end{forest}
	\begin{forest}
    [{$\{f(a)+g(a)\}$} [$\{f+g\}$] [$\{a\}$] ] 
    \end{forest}

\vspace*{3mm}

    \begin{forest}
    for tree={s sep=1cm, inner sep=0, l=0}
    [{$\{f(a)+g(b), g(a)+f(b), g(a)+g(b)+f(b), f(a)+f(b)+g(b), \ldots\}$} [$\{f+g\}$]  [$\{a+b\}$] ]
    \end{forest}
    \caption{An abstract illustration of the plural projection rule}
    \label{has-sch:fig:pp}
\end{figure}

Applying this principle to the functor set in \REF{has-sch:9d} and the argument set \sib{the boys} from \REF{has-sch:9a}, we derive the denotation in \REF{has-sch:10} for our non-lexical cumulation example \REF{has-sch:7a}. This denotation is a set of pluralities of propositions. A truth definition maps such a set to true iff at least one of its elements consists exclusively of true atoms. This yields the truth conditions paraphrased in \REF{has-sch:4b} for this sentence.

\ea \label{has-sch:10} $\{\textsc{want}(\textsc{feed}(\textsc{ivo})(\textsc{abe}))(\textsc{abe})+\textsc{want}(\textsc{feed}(\textsc{joe})(\textsc{bert}))(\textsc{bert}),$\\$ \textsc{want}(\textsc{feed}(\textsc{ivo})(\textsc{bert}))(\textsc{bert})+\textsc{want}(\textsc{feed}(\textsc{joe})(\textsc{abe}))(\textsc{abe}), $\\$ \textsc{want}(\textsc{feed}(\textsc{ivo})(\textsc{abe}))(\textsc{abe})+\textsc{want}(\textsc{feed}(\textsc{ivo})(\textsc{bert}))(\textsc{bert})+\textsc{want}(\textsc{feed}(\textsc{joe})(\textsc{abe}))(\textsc{abe}),  \textsc{want}(\textsc{feed}(\textsc{ivo})(\textsc{bert}))(\textsc{bert})+\textsc{want}(\textsc{feed}(\textsc{ivo})(\textsc{abe}))(\textsc{abe})+\textsc{want}(\textsc{feed}(\textsc{joe})(\textsc{bert}))(\textsc{bert}), \ldots\}$ \z  

\noindent For our purposes, the core property of this system is that the weak truth conditions symptomatic of cumulativity are derived without cumulation operators.\footnote{A reviewer mentions cumulative readings of sentences with modified numerals like \textit{exactly/less than four boys} as a data point in favor of the operator approach. We disagree:  Such data are problematic for \textit{any} approach to cumulativity  (see, e.g.,~\citealt{Krifka:1999, Landman:2000, Brasoveanu:2013}), as each theory needs additional assumptions to account for them. \citet{Buccola:2016} provide such an expansion for the operator approach.  For an analysis of quantificational plural expressions (and the interaction between plurals and quantifiers) within the projection approach see \citet{Haslinger:2018a, Haslinger:2020a} (the latter paper discusses modified numerals).}  So if it  were cross-linguistically valid, we should not find overt morphemes marking cumulativity. We also would not expect grammars to formally distinguish lexical and non-lexical cases of cumulativity, or to prohibit non-lexical cases. Finally, \citet{Beck:2000a} argue that the formation of non-lexical cumulative relations is subject to independently motivated syntactic constraints, which would favor the syntactic operator approach (but see \Cref{has-sch:fn-flattening} above and \citealt{Schmitt:2019}) -- an empirical issue that has not been studied cross-linguistically.\footnote{\citet{Schmitt:2019} claims that the formation of non-lexical cumulative relations is \textit{not} subject to the constraints usually observed for covert movement: She argues that the examples for which \citet{Beck:2000a} claim a cumulative reading to be absent -- and which would involve island-violating covert movement -- permit this reading once more context is added.  \citet{Schmitt:2019} doesn't consider this a definitive argument against the operator approach, however.}

\subsubsection{Event-based analyses}

There is a second class of theories that accounts for non-lexical cumulation without applying the ** operator to complex predicates (see, e.g.,~\citealt{Schein:1993,Landman:1996,Landman:2000,Kratzer:2003,Kratzer:2008, Ferreira:2005,Zweig:2008,Zweig:2009}). These theories crucially rely on a neo-Davidsonian semantics in which verbs simply denote sets of events (cf.~\citealt{Carlson:1984}) as in \REF{has-sch:ev1a}, and combine with their arguments via thematic-role relations. If so, \textit{see} denotes a set of ``primitive'' seeing events, which is then closed under sum as in \REF{has-sch:ev2a} to yield a set of possibly plural seeing events. To compose with this verb meaning, each argument must be mapped to a predicate of events. This mapping is achieved by thematic-role predicates, such as AG in \REF{has-sch:ev3a}, that attach to arguments in the syntax. For instance, \sib{AG} maps the sum \textsc{abe}+\textsc{bert} to the set of all events $e$ that Abe and Bert cumulatively stand in the agent relation to. For a predicate like \textit{see} that arguably cannot apply collectively, this means $e$ can be decomposed into subevents such that each of Abe and Bert is the agent of some subevent, and each subevent has Abe or Bert as its agent. 

\ea
\ea \label{has-sch:ev1a} \sib{see} = $\{e, e'\}$
\ex \label{has-sch:ev2a} \sib{* see} = $\{e, e', e+e'\}$
\ex \label{has-sch:ev3a} \sib{AG} is the smallest relation $R$ such that (i) for all $x\in D_{e}$ and all events $e$, if $x$ is the agent of $e$, then $R(x)(e)$  and (ii) for all $S \subseteq D_{e}$ and all sets $E$ of events such that for every $x \in S$ there is an $e \in E$ such that $R(x)(e)$ and for every $e \in E$ there is an $x \in S$ such that $R(x)(e) $, $R(\bigplus(S))(\bigplus(E))$.
 \z
\z

\noindent Crucially, if thematic-role relations are defined as in \REF{has-sch:ev3a}, they are cumulative relations. The theoretical interest of this idea lies in the fact that it provides an account of non-lexical cumulativity that requires neither ** operators attaching to complex constituents, nor a composition rule specific to plurality. To see this, consider the LF a cumulative sentence with infinitival embedding would have under this theory \REF{has-sch:see-lf}. We use \textit{see} here since the intensionality of \textit{want} gives rise to complications (see \sectref{has-sch:sec:4}).

\ea \label{has-sch:see-lf}  [[AG [Ada and Bea]] [$_C$ [* saw] [TH [$_B$ [AG [two women]] [$_A$ * sell [TH [drugs]]]]]]]\z

\noindent The verb meaning in the embedded clause combines intersectively with its object, which also denotes a predicate of events once TH has applied; thus, the node labeled $A$ will denote a predicate true of all (possibly plural) selling events with drugs as the cumulative theme. This combines, again intersectively, with the embedded-clause subject, yielding the set of all selling events with two women as the cumulative agent and some drugs as the cumulative theme. To give an example, if $e$ in \REF{has-sch:ev1a} is an event of Claire selling drugs and $e'$ is an event of Dora selling drugs, $e+e'$ will satisfy the predicate expressed by $B$.

To combine this with the matrix predicate, we need to assume that the theme of a seeing event may be another event. The matrix VP labeled $C$ will then denote the set of all (possibly plural) seeing events with some event satisfying $B$ as their cumulative theme. Crucially, this set would contain, for instance, the sum of an event of Ada seeing Claire sell drugs and an event of Bea seeing Dora sell drugs, since the cumulative theme of this plural event is $e+e'$. Adding the agent argument and applying an existential event quantifier, we get the truth conditions in \REF{has-sch:see-formula} (relative to a world $w$), which correspond to a cumulative reading.

\ea \label{has-sch:see-formula} $\lambda e'.$\sib{*}$\textsc{see}(w)(e') \land \text{\sib{AG}}(\textsc{ada}+\textsc{bea})(e') \land \exists e[$\sib{*}$\textsc{sell}(w)(e) \land \text{\sib{TH}}(e)(e') \land \exists x[\textsc{women}(w)(x) \land |x| = 2 \land \text{\sib{AG}}(x)(e) \land \exists y[\textsc{drugs}(w)(y) \land \text{\sib{TH}}(y)(e)]]] $
\z

\noindent In sum, in such theories, cumulation between two individual arguments is always mediated by an event argument. The locus of cumulativity is the thematic-role relations relating individuals to events, or events to other events.

What are the typological predictions of this system? Each of the relevant compositional steps yields a one-place predicate of events. There is therefore no need to account for cumulative truth conditions in terms of lexically cumulated predicates; the only lexically cumulative predicates are the thematic-role relations. But unlike the ** operator, these thematic-role predicates are assumed to be present whenever an argument of an event predicate is introduced, regardless of whether the argument is singular or plural and whether its relation to the other individual arguments is cumulative. While a theory of this type would therefore lead us to expect overt counterparts of the thematic-role predicates, it would not predict the existence of overt morphology specific to cumulativity. Its predictions concerning overt morphology and non-lexical cumulativity therefore coincide with those of the plural projection account. Potential differences between the two operator-less non-lexical accounts are discussed in \sectref{has-sch:sec:4} below.\footnote{A reviewer notes that one could have a system where cumulative thematic-role relations like \REF{has-sch:ev3a} are derived from ``primitive'' thematic-role relations via a syntactically represented ** operator. One would then, by our logic, expect to find overt marking of this ** operator. However, this differs from the prediction of the operator-based account in that we would expect this marking on \textit{any} plural argument, regardless of whether there are other plurals in the sentence and whether the sentence as a whole has a cumulative reading. In effect, at least for DP/NP arguments, this marking would have the distribution of plural morphology. Such a system would therefore still not predict that we find morphemes specific to cumulativity.}

\subsection{Summary}

We sketched four approaches to cumulative truth conditions based on the two parameters in \tabref{has-sch:table} below.

\begin{table}[h!]
\caption{Four types of cumulation approaches}
\label{has-sch:table}
\centering
\begin{tabularx}{\textwidth}{lXX}  
  \lsptoprule
  & $-$ non-lexical relations & $+$ non-lexical relations \\  \midrule
 $+$ ** operator & / & \citet{Sternefeld:1998}\\ 
 & & \citet{Beck:2000a}\\\tablevspace
 $-$ ** operator & \citet{Scha:1981}, \citet{Krifka:1986} a.o. 
 & \citet{Landman:1996,Landman:2000}, \\
 & & \citet{Schein:1993},\\
 & & \citet{Kratzer:2003,Kratzer:2008} a.o.;\\
 & & \citet{Schmitt:2019}, \\
 & & \citet{Haslinger:2018a} \\
  \lspbottomrule
\end{tabularx}
\end{table}

The first two are inadequate for English as they limit cumulativity to lexicalized relations. But it remains to be seen if they might be adequate for other languages, i.e.~if the availability of non-lexical cumulation varies across languages. The latter two approaches permit non-lexical cumulativity, but differ in how they encode it: a cumulation operator in the syntax or a plural-sensitive composition mechanism. Typological questions relevant to the choice between them include whether ** is realized overtly in some languages. 

\section{Cross-linguistic predictions}\label{has-sch:sec:3}

We now discuss the cross-linguistic predictions of the different potential settings of Parameter 1, given in \REF{has-sch:par1} (Is there non-lexical cumulation?) and Parameter 2, given in \ref{has-sch:par2} (Are there object-language cumulation operators?). We  will draw on data from the literature and preliminary results from two cross-linguistic data samples we are compiling.


\subsection{Q 1: Does non-lexical cumulation exist cross-linguistically?} \label{has-sch:sec:3.1}

We saw above that English exhibits cases of non-lexical cumulation. This is predicted by theories that model cumulativity as a freely  available
 syntactic operation -- possibly \textit{modulo} syntactic constraints (\sectref{has-sch:sec:2.3}) or via composition rules (\sectref{has-sch:sec:2.4}), but not by theories in which cumulativity is due to meaning postulates on lexical predicates (\sectref{has-sch:sec:2.1}) or additional operators that exclusively modify lexical predicates (\sectref{has-sch:sec:2.2}). We are currently collecting a cross-linguistic data set to test whether English is exceptional in this respect and thus probe the scope of the theories in question. The preliminary data set (here: Sample 1) contains seven languages from three major language families (Indo-European, Uralic, Japanese): 
Dutch, German, Hungarian, Japanese, Polish, Punjabi, Bosnian/Croatian/Montenegrin/\-Serbian (henceforth BCMS). 
Via a written questionnaire, we asked consultants to construct certain types of sentences in their language and judge their adequacy in certain scenarios.\footnote{The preliminary character of our results stems from the fact that, so far, these are based on one or two speakers per  language (with the exception of German, for which we consulted several speakers) with all of our consultants except one being linguists. The questionnaire (which includes the instructions to those consultants who were linguists) is accessible via \url{https://sites.google.com/view/the-typology-of-cumulativity/questionnaires}.} Some of the examples targeted non-lexical cumulativity: Consultants were asked to identify correlates of \REF{has-sch:questi-b}--\REF{has-sch:questi-d} in their languages and judge their truth value in cumulative scenarios of the kind shown in \REF{has-sch:sceni}.

\ea\label{has-sch:questi}
\ea \label{has-sch:questi-b}  Ada and Bea tried to arrest two criminals.
\ex \label{has-sch:questi-c}  Ada and Bea saw two women sell drugs.
\ex \label{has-sch:questi-d}  Ada and Bea believe that two criminals are threatening Gene. 
\z\z

\ea\label{has-sch:sceni}
\ea \textsc{scenario}: Ada tried to arrest criminal 1, Bea tried to arrest criminal 2.\label{has-sch:sceni-b} 
\ex \textsc{scenario}: Ada saw woman 1 sell drugs. Bea saw woman 2 sell drugs. \label{has-sch:sceni-c} 
\ex  \textsc{scenario}: Ada believes criminal 1 is threatening Gene. Bea believes that criminal 2 is threatening Gene. \label{has-sch:sceni-d} 
\z\z

\noindent The core result is that all seven languages permit non-lexical cumulativity. More precisely, they all permit it for sentences corresponding to \REF{has-sch:questi-b} and  \REF{has-sch:questi-c}.\footnote{One of our consultants for Dutch disliked a cumulative reading for the Dutch correlate of \REF{has-sch:questi-c} with an infinitival complement, but accepted it with a finite complement. This is surprising given the lower acceptability of cumulation across \textit{believe} in some languages, but orthogonal to our initial question. Further, one example we gave with seemingly lexical cumulation in English -- a sentence with \textit{feed} like \REF{has-sch:1a} -- was translated with complex predicates with causative morphology in Punjabi and Japanese. The sentences were judged true in a ``cumulative'' scenario, which provides additional evidence for the availability of non-lexical cumulation.} 
For instance,  \REF{has-sch:se1} from BCMS and \REF{has-sch:hu1} from Hungarian are judged true in scenario \REF{has-sch:sceni-b}, hence both sentences have a cumulative reading.\footnote{The categorial status of \textit{da} in \REF{has-sch:se1} is controversial (see \citealt{Todorovic:2020} a.o.).}

\ea\label{has-sch:se1} \gll Ju\v{c}e su Ada i Bea poku\v{s}ale da uhapse dva kriminalca.\\
yesterday \textsc{aux.3pl} Ada.\textsc{nom} and Bea.\textsc{nom} try.\textsc{pf.ptcp.pl.fem} \textsc{prt} arrest.\textsc{pf.npst.3pl} two.\textsc{masc} criminal.\textsc{pauc} \\
\glt `Yesterday, Ada and Bea tried to arrest two criminals.' \phantom{.}\hfill (BCMS)
\z 

\ea\label{has-sch:hu1} \gll Ada \'{e}s Bea tegnap megpr\'{o}b\'{a}lt letarz\'{o}ztatni k\'{e}t b\H{u}n\"{o}z\H{o}t. \\
Ada and Bea yesterday \textsc{prt}.try.\textsc{pst}.\textsc{3sg} arrest.\textsc{inf} two criminal.\textsc{acc} \\
\glt 'Yesterday, Ada and Bea tried to arrest two criminals.' \phantom{.}\hfill (Hungarian) 
\z

\noindent As the relation that must hold cumulatively, [$\lambda x.\lambda y. y$ tried to arrest $x$], was not expressed by a single lexical item in either language, we have evidence for non-lexical cumulation. The other languages in the sample behaved analogously. The only major point of variation concerned examples corresponding to \REF{has-sch:questi-d}: the cumulative reading was available in German for many (but not all) speakers, Punjabi and BCMS but not in Polish and Hungarian, and the judgements for Dutch and Japanese were unclear. 

Irrespective of the judgments for examples involving correlates of \textit{believe}, the data involving correlates of \textit{see} and \textit{try} sufficiently support the conclusion that non-lexical cumulation is possible in all languages in our sample, so we submit Generalization 1. Yet, given the small size of our sample, further research must determine whether any languages systematically block non-lexical cumulation.

\ea \textsc{Generalization 1:} Non-lexical cumulation, although potentially subject to further restrictions, exists across languages. \z

\noindent The variation concerning cumulativity with \textit{believe} is an interesting point for further study, especially as we also find variation \textit{within} languages, for instance in German. A potentially relevant observation is that in some of the languages under discussion, belief ascriptions involve a finite complement, whereas the other predicates embed infinitives. (We omit a more detailed  data presentation, as the restrictions on non-lexical cumulative readings are not our main concern here and including all the data would exceed the scope of this paper.) While there is certainly no direct correlation between finiteness and lower acceptability of the cumulative reading, one could speculate that cumulative readings are available more easily for complements with a smaller left periphery, assuming a theory where both finite and non-finite complements can come in different ``sizes'' \citep{Wurmbrand:2015,Todorovic:2020}. This would be in line with ``syntactic'' theories of cumulation like \citet{Beck:2000a}. Alternatively, attitude predicates might block cumulativity semantically or pragmatically.\footnote{A semantic explanation would have to rely on a lexical semantics of attitudes that differs from the one traditionally assumed and interacts with cumulativity in a non-trivial way. A pragmatic account would have to appeal to the interaction of  general pragmatic constraints on the availability of cumulative readings with the semantics of attitude predicates. Accordingly, the different potential explanations would attribute the inter-speaker variation to different sources (syntactic constraints vs.~lexical meanings of attitude verbs vs.~pragmatic constraints on cumulativity).} We briefly return to the theoretical relevance of cumulation across attitude predicates in \sectref{has-sch:sec:4}.

\subsection{Cumulation and distributivity operators in the grammar}\label{has-sch:sec:3.2}

We saw above that English provides no morpho-syntactic evidence for cumulation operators. This is not \textit{per se} a problem for theories assuming such operators: one would not expect them to be overt in \textit{all} languages. Yet, one would expect to find morpho-syntactic correlates of these operators in \textit{some} languages, while the composition-based approaches in \sectref{has-sch:sec:2.4} do not make this prediction. 
Since cumulation operators could interact with other plural-sensitive semantic phenomena, like distributivity, in different ways, it is not always clear how to identify their overt counterparts in a given language. Let us illustrate the different options in English. English sentences with multiple plurals are often ambiguous between cumulative and distributive readings: under its cumulative reading, \REF{has-sch:11} is true in scenario \REF{has-sch:11a}, but (at least with \textit{exactly}) false in the distributive scenario \REF{has-sch:11b}. For the distributive reading, the situation is reversed.

\ea \label{has-sch:11} Abe and Bert fed (exactly) two cats.
\ea \label{has-sch:11a} \textsc{cumulative scenario:} Abe fed cat Ivo. Bert fed cat Joe.
\ex \label{has-sch:11b} \textsc{distributive scenario:} Abe fed cats Ivo and Joe. Bert fed cats Kai and Leo. \z
\z

\noindent The distributive and the cumulative construal are usually assumed to correspond to distinct LFs. The existence of elements that disambiguate the sentence towards one of these construals (e.g.,~predicate modifiers like English \textit{each} or \textit{between them}, DP-level items like distributive numerals) further confirms that grammar is sensitive to the distinction.\footnote{While we take \textit{between them} to be an element that is ``parasitic'' on a cumulative reading that is derived by independent means, a reviewer points out that it could also be analyzed as a realization of **. It is beyond the scope of this paper to settle this issue (or the analogous issue for \textit{together}), but there is some evidence that \textit{between them} does not have the exact distribution assumed for the **-operator. For instance, \textit{between them} seems to be limited to sentences where at least one plural involves a numeral/cardinal/universal expression: all of the sentences in \REF{has-sch:num} can have a cumulative reading, but only \REF{has-sch:numa} permits \textit{between them} (under the relevant reading). 

\ea \label{has-sch:num}
\ea \label{has-sch:numa} Those boys ate ten sausages (between them).
\ex Those boys ate the sausages ($\#$ between them).
\ex Those boys saw the dogs ($\#$ between them).
\z\z

\noindent We thank Tim Stowell (p.c.) for these judgments.} This raises the question whether one of the readings is ``more primitive'': is the cumulative reading built ``on top of'' the distributive reading or \textit{vice versa}? From the perspective of a theory with cumulation operators, the different possible answers to this question entail different predictions about the distribution of these operators and of their potential overt realizations.

As a starting point, consider the LF in \REF{has-sch:12a} for the cumulative reading of \REF{has-sch:11} (see \sectref{has-sch:sec:2.2} for the semantics of the ** operator). Assuming that indices can range over plural as well as atomic individuals, \REF{has-sch:12a} is true iff there is a plurality of two cats that stands in the relation \sib{** fed} to the sum of Abe and Bert.

\ea 
\ea \label{has-sch:12a} {[Abe and Bert [2 [two cats [1 [t$_2$ [**fed t$_1$] ] ] ] ] ]} 
\ex \label{has-sch:12b} \sib{two cats} = $\lambda P_{\stb{e, t}}.\exists x_e[\textsc{cats}(x) \land |x| = 2 \land P(x)]$ \z\z

\noindent In principle, we could start with a structure with a distributive interpretation and derive the cumulative reading by adding ** to it (and performing the syntactic operations needed to form the right relation). As an illustration of this class of analyses (here: Class I), take the potential lexical meaning for \textit{fed} in \REF{has-sch:13b}.


\ea \label{has-sch:13b} \sib{fed} = $\lambda x_e.\lambda y_e.\forall y' \leq_a y.\forall x' \leq_a x.\textsc{fed}(x')(y')$ \z

\noindent So far, we have tacitly assumed that \sib{fed} cannot be true of plural arguments unless affixed with **. But  \sib{fed}  in \REF{has-sch:13b} takes two potentially plural arguments $x$, $y$ and requires that each atomic part of $y$ must have fed each atomic part of $x$ -- a distributive relation. Given \REF{has-sch:13b}, the LF in \REF{has-sch:13a} would yield the distributive reading, but the cumulative reading would require the more complex LF \REF{has-sch:12a}.\footnote{Just as we omit any discussion of collectivity, we also ignore cases (brought up by a reviewer) where some sub-pluralities of the agent and/or theme acted collectively (see e.g, \citealt{Does:1992, Landman:2000, Vaillette:2001, Champollion:2017}). A serious investigation the predictions of the different theories for such examples would exceed the scope of the present paper by far.}


\ea \label{has-sch:13a} {[Abe and Bert [2 [two cats [1 [t$_2$ [fed t$_1$] ] ] ] ] ]}  \z

\noindent We should point out that (as noted by a reviewer) in the case of distributivity, there is a general consensus that a purely lexical account is insufficient and distributivity operators must be represented in the syntax (see, e.g.,~\citealt{Champollion:2015b}). Thus, the lexical item \textit{fed} in \REF{has-sch:13b} should be viewed as a shorthand for a complex structure including a distributivity operator. We suppress these details here to focus on the crucial prediction of Class I analyses: they lead us to expect languages that require special morphology for a cumulative reading of a sentence like \REF{has-sch:11}, while removing this morphology would yield a distributive reading. To derive this prediction, we rely on the assumption that theories with cumulation operators would lead us to expect languages where they have an \textit{obligatory} non-zero spell-out. This is because the operator-based theory would otherwise leave a generalization unexplained, namely that the zero spell-out is universally available. In contrast, an operator-less theory leads us to expect that cumulativity is never obligatorily marked.\footnote{In particular, since there seem to be languages where distributivity is marked overtly obligatorily (even in the sample discussed in \sectref{has-sch:sec:3.3} below; see \citealt{Flor:2017b, Flor:2017}), it would be surprising if cumulation operators behaved differently.}

The second class of analyses (Class II) assumes that lexical predicates like \textit{fed} cannot hold of plural arguments unless a ``pluralizing'' operator is added. There could then be two distinct kinds of such operators, yielding cumulative and distributive readings, respectively. Thus, the distributive reading could have an LF like \REF{has-sch:14b}, where \textsc{d} has the denotation in \REF{has-sch:14a}, applying to a unary predicate and a plurality and requiring the predicate to hold of each atomic part of the plurality. 

\ea 
\ea \label{has-sch:14a} \sib{\textsc{d}} = $\lambda P_{\stb{e, t}}.\lambda x_e.\forall x'[x' \leq_a x \rightarrow P(x')]$
\ex \label{has-sch:14b} {[Abe and Bert [\textsc{d} [2 [two cats [\textsc{d} [1 [t$_2$ [fed t$_1$] ] ] ] ] ] ] ]} \z\z

\noindent As \REF{has-sch:14b} lacks ** and the LF for the cumulative reading lacks \textsc{d}, no morphosyntactic containment relation between the two readings is predicted: languages that overtly express both ** and \textsc{d} would have different markers for the distributive and the cumulative reading that are in complementary distribution, and any sentence with plural arguments would contain one of the markers.

The third kind of system (Class III) would be one where predicates always need to be pluralized via ** (or analogous operators for higher arities) before combining with plural arguments, and \textsc{d} can only apply `on top of' cumulation operators, so that the distributive reading always corresponds to a more complex LF. A suitable LF for the distributive reading of \REF{has-sch:11} is given in \REF{has-sch:15}. Note that, since the task of making the lexical predicate \textit{fed} compatible with plural arguments is now performed by **, we need only one occurrence of \textsc{d}, unlike in \REF{has-sch:14b}.

\ea \label{has-sch:15} {[Abe and Bert [\textsc{d} [2 [two cats [1 [t$_2$ [**fed t$_1$] ] ] ] ] ] ]} \z

\noindent In Class III systems, both readings of a sentence with two plural arguments require a cumulation operator. What does this mean for our question how to identify overt realizations of such operators? Given a system of Class I or II, we could identify such overt realizations by comparing plural sentences with a cumulative reading and those restricted to a distributive reading. But in a Class III system, this is impossible, as cumulation operators would show up in both types of sentences. Instead, we would have to compare sentences with at least one plural argument to those completely lacking plural arguments. This was not the focus of the cross-linguistic study we will now discuss,
which concentrated on morphosyntactic contrasts correlating with the distributive/non-distributive distinction. Foreshadowing, while our results don't support operator-based theories of Class I and II,  they do not affect operator-based theories of Class III.


\subsection{Q 2: Is there evidence for object language cumulation operators?}\label{has-sch:sec:3.3}

We now turn to the question whether cumulation operators are overtly realized in a way compatible with a Class I or Class II analysis of the distributive reading -- i.e., an analysis where the distributive reading does not involve such operators. We will draw on Sample 1 as well as what we call  Sample 2, which stems from an open-ended survey of native-speaker linguists we initiated on the online platform TerraLing \citep{terraling}.
 Sample 2 currently contains 19 languages, four of which are also in Sample 1, from 7 major language families.\footnote{As this was a survey on many topics and we only have partial results for many languages, we only count the languages where consultants answered the query whether sentences analogous to \REF{has-sch:11} show obligatory morphosyntactic marking of the cumulative or the distributive reading, external to the conjunction. These are: Basaá (Niger-Congo/Bantu), Dagara [Burkina] (Niger-Congo/Gur), Dutch (Indo-European/Germanic), Estonian (Uralic), German (IE/Germanic), Greek (IE), Guangzhou Cantonese (Sino-Tibetan/Chinese), Igbo (Niger-Congo), Iranian Persian (IE/Indo-Iranian), Iraqi Arabic (Afro-Asiatic/Semitic), Italian (IE/Romance), Korean (Koreanic), Nones (IE/Romance), Norwegian (IE/Germanic), Polish (IE/Slavic), BCMS (IE/Slavic; referenced as `Serbo-Croatian' in the TerraLing group), Sicilian (IE/Romance), Turkish (Turkic), Wuhu Chinese (Sino-Tibetan/Chinese). The consultants are native-speaker linguists except for the following languages, where we interviewed non-linguist native speakers: Estonian, Iranian Persian and Iraqi Arabic.} 

This survey focused  on sentences where conjunctions of individual-denoting expressions -- specifically proper names -- combine with simple predicates containing a numeral as in \REF{has-sch:11} (= \REF{has-sch:110}) or a measure phrase.

\ea \label{has-sch:110} Abe and Bert fed (exactly) two cats.
\z

\noindent Consultants were again asked to construct relevant examples 
and judge their truth value in scenarios we provided. The precise questionnaire, including examples and contexts, can be found in our TerraLing group \citep{conjunction}.

In contrast to Sample 1, we did not ask for non-lexical cumulative predicates. The initial goal was to determine whether the cumulative reading -- on which \REF{has-sch:11} is true in scenario \REF{has-sch:11a} -- is cross-linguistically more ``primitive'' than the distributive reading -- on which \REF{has-sch:11} is true in \REF{has-sch:11b} -- or \textit{vice versa}. Simplifying slightly, we thus asked consultants to check whether correlates of \REF{has-sch:11} required additional morphology to make the cumulative reading available (i.e.~the counterpart of \REF{has-sch:11} is only true in scenario  \REF{has-sch:11b}, and extra morphology is needed to make it true in scenario  \REF{has-sch:11a}). Similarly, they had to check whether correlates of \REF{has-sch:11} required additional morphology for the distributive reading (i.e.~the counterpart of \REF{has-sch:11} is only true in scenario \REF{has-sch:11a}, and extra morphology is needed to make it true in scenario \REF{has-sch:11b}). Consultants were asked to use numeral modifiers like \textit{exactly} if possible, to ensure that there is no entailment relation between the two readings (with an `at least' reading of the numeral, the distributive reading of \REF{has-sch:11} entails the cumulative one). In our questionnaire about non-lexical cumulation (Sample 1), we also asked if either of the readings required extra morphemes, but used the linguistic context instead of modifiers to force an exact reading of the numeral.

Our result was that no language in either sample required extra marking for the cumulative reading -- but some languages in both samples required overt marking to make the distributive reading available. (Languages for which such judgments were reported both with numeral-modified indefinites and with measure phrases, suggesting a consistent pattern, include Basaá, Greek and Turkish.) So we found no morpho-syntactic evidence that cumulation operators can turn a structure limited to a distributive reading into one with a cumulative reading -- if so, we would expect ``purely distributive'' structures that obtain a cumulative reading if extra morphology is added. We take this to support Generalization 2:

\ea \textsc{Generalization 2:} Cross-linguistically, in sentences with a conjunctive subject and a numeral or measure phrase
 in the predicate, there is no morphological evidence for cumulation operators, assuming that these operators are absent in distributive sentences. \z

\subsection{Pluractional markers as cumulation operators?}\label{has-sch:sec:3.4}

To summarize, we did not find overt expressions with the behavior predicted for a cumulation operator by analyses in which \textit{distributive} readings  do \textit{not} require such an operator. But our survey data have no bearing on Class III analyses, where distributive readings have strictly more complex LFs with an additional distributivity operator ``on top'' of the cumulation operator. \citeposst{Beck:2012} interesting study of the pluractional system in Konso (Afro-Asiatic/Cushitic) addresses potential morphosyntactic evidence for a system of this kind.  To conclude our survey, we will summarize this work and explain why we consider the consequences of the Konso data for our questions inconclusive, pending further study.

Konso distinguishes between singulative and pluractional verbs. The semantic correlate of this contrast is a distinction between predicates true of events with multiple individuable subevents (pluractional)
and predicates true  only of events lacking individuable subevents (singulative). \citet{Ongaye:2017} discuss various secondary inferences triggered by the singulative and the pluractional, which we gloss over here. Lexical verb roots are classified as singulative or pluractional in an unpredictable way, but  two derivational processes  affect pluractionality: a process that applies to a pluractional root and forms a derived singulative, and a reduplication process that forms derived pluractionals. According to \citet{Ongaye:2017}, only the latter is fully productive.

How does this relate to cumulation operators? As \REF{has-sch:16} shows, the distribution of the pluractional is closely tied to semantic plurality in that, if a verb takes a plural argument, it must bear pluractional marking.\footnote{We cite the data from \citet{Beck:2012} as her original source, an unpublished talk handout by Ongaye Oda Orkaydo, was unavailable to us. For clarity, the glosses for the nominal suffixes were adapted following \citet{Ongaye:2013}. We write ʔ instead of Beck's ? for the glottal stop. According to \citet{Ongaye:2013}, Konso has what he calls \textsc{plural gender}; this marker, glossed as \textsc{p}, is not fully correlated with semantic plurality. Note also that \REF{has-sch:16a} can have an iterative interpretation (see below), but we follow Beck's translation.}

\ea\label{has-sch:16}
\ea\label{has-sch:16a}\gll harreeta-sik kaharta-siʔ\hspace{5.4cm} \minsp{\{} i=did-diit-t-i {/} i=diit-t-i\} \\
donkey-\textsc{def.masc/fem} ewe-\textsc{def.masc/fem} {} 3=\textsc{redp}-kick[\textsc{sg}]-\textsc{3sg.fem}-\textsc{pf} {} 3=kick[\textsc{sg}]-\textsc{3sg.fem}-\textsc{pf} {} \\ \hfill 
\glt `The donkey (has) kicked the ewe.' \hfill \citep[Konso; ][(14a), (17a)]{Beck:2012}%\footnote{This example can have an iterative interpretation (see below), but we follow Beck's translation.}
\ex\label{has-sch:16b} \gll harreeta-sik kaharraa-siniʔ \minsp{\{} i=did-diit-t-i  {/} \minsp{*} i=diit-t-i\}\\
donkey-\textsc{def.masc/fem} ewes-\textsc{def.p} {} 3=\textsc{redp}-kick[\textsc{sg}]-\textsc{3sg.fem}-\textsc{pf} {} {} 3=kick[\textsc{sg}]-\textsc{3sg.fem}-\textsc{pf}  {} \\ \hfill
\glt `The donkey (has) kicked the ewes.' \hfill \citep[Konso; ][(14b), (17c)]{Beck:2012}
\ex\label{has-sch:16c} \gll harreewwaa-sinik kaharraa-siniʔ i=did-diit-i-n \\ 
donkeys-\textsc{def.p} ewes-\textsc{def.p} 3=\textsc{redp}-kick[\textsc{sg}]-\textsc{pf}-\textsc{pl} \\ \hfill
\glt `The donkeys (have) kicked the ewes.' \hfill \citep[Konso; ][(14d)]{Beck:2012} \z\z

\noindent  In \REF{has-sch:16a}, with two singular arguments, both the singulative and the derived pluractional (formed via reduplication) can be used. With pluractional marking, the sentence conveys that the ewe was kicked many times, i.e., it has a so-called ``iterative'' interpretation, while the singulative conveys there was only one kicking. Crucially, if one of the arguments is plural, the singulative is bad \REF{has-sch:16b}. This arguably follows from the event-based paraphrase given above, since an event in which several sheep are kicked has individuable subevents. Multiple plural arguments, as in \REF{has-sch:16c}, also require the pluractional.

Given this restriction on plural arguments, Beck suggests pluractional verbs denote cumulative predicates, while singulative verbs denote predicates requiring atomic arguments. If so, the reduplication process  in \REF{has-sch:16} provides a fully productive way of deriving a cumulative predicate from a predicate prohibiting plural arguments. If the semantic correlate of this reduplication were **  (or its counterpart for predicates of higher arity), the pattern in \REF{has-sch:16} would follow.

But there are two reasons why, although the data discussed by \citet{Beck:2012} are all compatible with an operator-based account of cumulation, the presence of overt pluractional morphology in her data is not a clear-cut argument for such a theory over a non-lexical, composition-based theory. First, \citet{Beck:2012} points out that her source, \citet{Ongaye:2010}, gives a paraphrase for \REF{has-sch:16c}  suggesting a distributive reading. The question whether a cumulative reading is also available is left open, and is also not resolved in the more recent study of Konso pluractionals in \citet{Ongaye:2017}. So a clearer picture of how the language marks distributivity would be needed to evaluate the analytical options discussed above and in \citet{Beck:2012}. Second, assuming that the cumulative reading is available, the sensitivity of the pluractional to event structure yields new analytical options that do not involve cumulation operators.\footnote{As the pluractional is compatible with singular arguments \REF{has-sch:16a} and, in this case, adds the implication that there were multiple kicking events, its semantics cannot appeal exclusively to the semantic number of the verb's type $e$ arguments. \citet{Ongaye:2017} provide an independent argument that the pluractional is sensitive to event structure: some verbs can be in the singulative with a plural argument, but only if the latter has a collective reading.} To illustrate this, we briefly return to the different ways of integrating cumulativity into event semantics.

On one approach, discussed in \citet{Beck:2012}, transitive verbs have an extra argument position for an event. Thus, \textit{kick} denotes a relation between two individual arguments and an event argument, as in \REF{has-sch:17a}. The cumulation operator ***, which is a generalization of ** to three-place relations (see \citealt{Sternefeld:1998, Vaillette:2001}) then closes this relation under pointwise sum \REF{has-sch:17b}. We could analyze the LF syntax of both \REF{has-sch:16c} and its English counterpart along the lines of \REF{has-sch:17c} (ignoring the question whether the plurals undergo LF movement). Structure \REF{has-sch:17c} denotes a predicate true of all events that are events of the donkeys cumulatively kicking the ewes. Beck suggests that reduplication in \REF{has-sch:16} could spell out an operator similar to ***, which would derive the data pattern.

\ea \label{has-sch:17}
\ea \label{has-sch:17a}\sib{kick} = $\{\langle a, b, e\rangle, \langle c, d, e'\rangle\}$
\ex \label{has-sch:17b}\sib{***kick} = $\{\langle a, b, e\rangle, \langle c, d, e'\rangle, \langle a+c, b+d, e+e'\rangle\}$ 
\ex \label{has-sch:17c} [ [the donkeys] [ [ ***kicked ] [the ewes] ] ] \z\z

\noindent Yet, as we saw in \sectref{has-sch:sec:2.4} above, the literature provides another approach to cumulativity in event semantics -- the thematic-role approach. On this theory, \REF{has-sch:16c} and its English counterpart would have an LF along the lines of \REF{has-sch:18c}.

\ea \label{has-sch:18c} [ [AG [the donkeys] ] [ [ *kicked ] [TH [the ewes] ] ] ]
\z

\noindent If the pluralized verb in \REF{has-sch:18c} combines with its arguments intersectively, we obtain the set of all kicking events $e$ with the following property: the donkeys cumulatively stand in the agent relation to $e$, and the ewes in the theme relation, also cumulatively. Cumulativity arises from the semantics of the thematic-role predicates. But since the * operator is required to get events with more than one atomic part, a cumulative reading would still be unavailable without it.

So even if the cumulative reading is available in Konso, there is an analysis of the pluractional that does not identify it with a cumulation operator (in the sense of ``cumulation'' we have been using throughout this paper): it could spell out the event-pluralization operator *. The consequences for the question whether overt counterparts of operators like ** or *** exist then depend on the choice between the operator-based analysis in \REF{has-sch:17} and the thematic-role analysis in \REF{has-sch:18c}.\footnote{\citet{Henderson:2012} provides an analysis of pluractionality in Kaqchikel that relies on cumulation operators: Kaqchikel morphologically marks two different types of pluractionality, which Henderson analyzes as taking scope above and below the cumulation operator, respectively. Since Henderson doesn't identify either of the two pluractional morphemes with the cumulation operator, his data do not directly contradict our conclusion that there is no \textit{morphological} evidence for cumulation operators. That said, it is unclear to us at this point whether the operator-less theories can derive his data set. Since we only became aware of his work at a very late stage of the work reported here, we must leave this issue to future research.}


\section{Cross-linguistic data and theories of cumulativity}\label{has-sch:sec:4}

In summary, we can draw two conclusions: (i) \citeposst{Beck:2000a} main finding for English -- that we find cumulative readings for relations that don't correspond to lexical elements or even surface constituents -- generalizes to several typologically diverse languages. (ii) There is no compelling \textit{positive} evidence for object language cumulation operators (although, depending on our assumptions about their distribution, they might still exist). The question we want to address now is which theories of cumulativity best account for the results.

Result (i) provides evidence for a theory that permits non-lexical cumulation, and our restricted data set did not turn up any evidence that languages vary in this respect, although a larger sample would be needed to settle this question. Result (ii) could be derived from any theory that does not rely on a syntactically represented ** operator. Thus, the theories that account for both generalizations are the two composition-based ones -- the plural projection approach and the thematic-role approach. A theory using cumulation operators would of course be compatible with both results at the observational level, in the sense that none of the individual data points in our samples falsify this approach. However, if our generalization (ii) turns out to reflect a real typological gap, a composition-based approach to cumulativity would correctly predict this gap, while an operator-based approach would have to treat it as coincidental.

This raises the question how one could decide between the two composition-based theories -- the thematic-role and the plural projection account. At some level of abstraction, the two theories are similar: both encode cumulation in the mechanism combining predicates with their arguments. However, the thematic-role account encodes a semantic constraint on cumulation that does not hold in the plural projection system. To see this, let us introduce a relation of \textsc{event-connectedness} informally characterized as follows. An individual-denoting definite or indefinite $x$ is \textsc{event-connected} to an event predicate $P$ in a given LF iff one of the following conditions holds: (i) $x$ is linked to the event argument of $P$ by a thematic-role relation. (ii) $x$ is event-connected to some predicate $Q$, and there is a thematic-role relation linking particular $P$-events to particular $Q$-events.

Let us now consider \REF{has-sch:questi-bb} again -- the LF a cumulative sentence with infinitival embedding would have under the thematic-role account. In \REF{has-sch:questi-bb}, \textit{Ada and Bea} is event-connected to *\textit{saw}, and \textit{two women} and \textit{drugs} are event-connected to *\textit{sell}. But since \REF{has-sch:questi-bb} also provides a thematic-role relation linking particular seeing events to particular selling events -- it requires there to be a seeing event whose theme is a selling event -- \textit{two women} and \textit{drugs} are also event-connected to *\textit{saw} and \textit{Ada and Bea} is event-connected to *\textit{sell}.

\ea \label{has-sch:questi-bb}  [[AG [Ada and Bea]] [[*\textit{saw}] [TH [[AG [two women]] *sell [TH [drugs]]]]]]
\z

\noindent The thematic-role approach to cumulation then makes the following prediction: Two distinct individual-type plural definites or indefinites $x$ and $y$ can cumulate only if there is a predicate that both $x$ and $y$ are event-connected to. This does not prevent \textit{Ada and Bea} from cumulating with \textit{the two women} in \REF{has-sch:questi-bb}, since both of these arguments are event-connected to *\textit{saw}.

The plural projection approach also permits cumulation in examples of this kind, but the predictions of the two theories diverge in other cases. While the plural projection system allows lexical items that block cumulativity (see, e.g., \citealt{Haslinger:2018a} on \textit{every}), it does not take this blocking to be inherently related to particular semantic types. It therefore permits cumulation between individual-denoting expressions that are not event-connected. The most prominent such case are examples where an intensional predicate, like \textit{believe} in \REF{has-sch:questi-dd}, intervenes between the two plurals. If we generalize the traditional possible-worlds semantics for \textit{believe} \citep{Hintikka:1969} to a neo-Davidsonian semantics, the theme arguments of \textit{believe} in a configuration like \REF{has-sch:questi-dd} are not particular threatening events, but propositions that specify the content of the belief \REF{has-sch:believe-formula}.\footnote{We think that our argument also extends to most analyses on which the \textsc{theme} of \textit{believe} is not a proposition (e.g.,~\citealt{Kratzer:2006,Moulton:2009,Moulton:2015,Hacquard:2006,Hacquard:2010}). These analyses assume primitive entities that carry propositional content, and assume that the \textsc{theme} of \textit{believe} is such an entity rather than a proposition. However, the cited works use operators in the embedded clause that map a proposition to a set or property of such content-bearing entities. Thus, the embedded clause has a proposition-denoting subconstituent. Consequently, an individual-denoting argument within this subconstituent -- e.g.,~\textit{two criminals} in \REF{has-sch:questi-dd} -- cannot be event-connected to arguments in the matrix clause (like \textit{Ada and Bea} in \REF{has-sch:questi-dd}), even if the content-bearing entities are events. This is because there is no thematic-role relation that relates particular threatening events to the belief states or other content-bearing entities quantified over in the main clause. Neither are they related by a chain of thematic-role relations. Therefore, a cumulative reading of sentences like \REF{has-sch:questi-dd} would still remain outside the scope of the thematic-role approach.
% We are unsure if this point extends to \citet{Moltmann:2019}, brought up by a reviewer, which also uses content-bearing entities as arguments of \textit{believe}, but employs a notion of content that is not built on possible worlds: truthmaker semantics. Entities such as belief states are related to a set of `truthmaking' and a set of `falsemaking' situations, rather than a proposition defined in terms of worlds. While truthmaker semantics might have interesting applications to cumulativity, we think that to account for \REF{has-sch:questi-dd}, such a system would have to extend the thematic-role approach in a non-trivial way, as Moltmann's definition (24) still relates content-bearing objects to the entire sets truthmakers and falsemakers, not to particular situations.
} If so, the arguments of \textit{threaten} in \REF{has-sch:questi-dd} are not event-connected to \textit{believe}. In sum, if a cumulative relation between \textit{two criminals} and \textit{Ada and Bea} is available in \REF{has-sch:questi-dd} (see \citealt{Pasternak:2018} and \citealt{Schmitt:2020} for further discussion of such readings), this relation is not straightforwardly captured by the thematic-role approach.

\ea
\ea \label{has-sch:questi-dd}  Ada and Bea believe that two criminals are threatening Gene.
\ex  \label{has-sch:questi-aa} Ada and Bea tried to arrest two criminals.
\z\z

\ea
\label{has-sch:believe-formula}$\lambda e.$\sib{*}$\textsc{believe}(w)(e) \land \text{\sib{AG}}(\textsc{ada}+\textsc{bea})(e) \land \text{\sib{TH}}(\lambda w'.\exists e'.$\sib{*}$\textsc{threaten}(w')(e') \land \exists x(\textsc{criminals}(w')(x) \land \text{\sib{AG}}(x)(e') \land \text{\sib{TH}}(\textsc{gene})(e')))(e)$
\z

\noindent Let us now return to our data set. \sectref{has-sch:has-sch:sec:3.1} showed that the cumulative reading for the correlate of \REF{has-sch:questi-dd} was unavailable in some of the languages in our sample -- while it was available in the other non-lexical configurations we tested. Further, in English, these cumulative readings are available for some speakers, but there is inter-speaker variation especially wrt.~\REF{has-sch:questi-dd}, where the cumulative reading is not universally accepted. So does this finding unambiguously support event-based analyses over the plural projection account? We don't think so -- in fact, we believe that none of the data addressed here sufficiently  distinguish between the theories. First, recall that while the correlates of \REF{has-sch:questi-dd} lacked a cumulative reading in some of the languages, they \textit{did} exhibit such a reading in other languages. So while the plural projection account  must explain the lack of the cumulative reading in the first set of languages  -- by appealing to independent syntactic or pragmatic factors blocking cumulativity -- event-based analyses must explain its presence in the second set, possibly by assuming language-specific additional operations underlying this reading. Further, the predictions of  event-based analyses depend on the semantics of the embedding configuration: it is not obvious whether \textit{try} in \REF{has-sch:questi-aa} can have a particular, actual event as its \textsc{theme} argument or whether its \textsc{theme} is irreducibly of a higher type, e.g.,~a property of events. If the \textsc{theme}s of \textit{try} are particular events, both theories under discussion correctly permit cumulation. If they are not, \textit{Ada and Bea} in \REF{has-sch:questi-aa} is not event-connected to \textit{two criminals} and event-based analyses would incorrectly block a cumulative reading.

To distinguish between the two theories, we would therefore need a more detailed data set, controlling not only for the semantic type of the complements in each language, but also for their syntax and for pragmatic factors that might block the cumulative construal. This, however, must be left to future research.

\section*{Abbreviations}
\begin{tabularx}{.45\textwidth}{lQ}
\textsc{acc} & accusative \\
\textsc{aux} & auxiliary \\
\textsc{def} & definite \\
\textsc{fem} & feminine \\
\textsc{inf} & infinitive \\
\textsc{masc} & masculine \\
\textsc{nom} & nominative \\
\textsc{npst} & non-past tense \\
\textsc{p} & plural gender agreement \\
\end{tabularx}
\begin{tabularx}{.45\textwidth}{lQ}
\textsc{pauc} & paucal \\
\textsc{pf} & perfective aspect \\
\textsc{pl} & plural / pluractional \\
\textsc{prt} & particle \\
\textsc{pst} & past tense \\
\textsc{ptcp} & participle \\
\textsc{redp} & reduplication \\
\textsc{sg} & singular / singulative \\
&\\
\end{tabularx}

\section*{Acknowledgements}
We want to thank Sigrid Beck, Enrico Flor, Jovana Gajić, Gurmeet Kaur, Hilda Koopman, Levente Madarász, Csaba Pléh, Tim Stowell, Yasutada Sudo, Hedde Zeijlstra and everyone who contributed to our TerraLing questionnaire study. We also thank two anonymous reviewers for their detailed comments and criticism. This work was supported by the Austrian Science Fund (FWF) projects P-29240 \textit{Conjunction and disjunction from a typological perspective} (all authors) and P-32939 \textit{The typology of cumulativity} (Rosina, Schmitt, Wurm).


{\sloppy\printbibliography[heading=subbibliography,notkeyword=this]}

\end{document}
