\documentclass[output=paper]{langscibook} 
\ChapterDOI{10.5281/zenodo.5082448}

\author{Mojmír Dočekal\affiliation{Masaryk University} and Marcin Wągiel\affiliation{Masaryk University}}
\title{Preface} 

\abstract{} %newest class

%\abstract{
%Abstract goes here and should not have more than 150 words.
%\keywords{countability, non-countable nouns, coercion, abstract nouns }}


\begin{document}
\SetupAffiliations{mark style=none}
\maketitle

\noindent This collective monograph is one of the outcomes of the research project \textit{Formal Approaches to Number in Slavic} (GA17-16111S; \url{https://sites.google.com/view/number-in-slavic/home}) funded by the Czech Science Foundation (GAČR) and carried out at the Department of Linguistics and Baltic Languages at Masaryk University in Brno in cooperation with researchers from the Center for Language and Cognition at the University of Groningen, the Department of German Studies at the University of Vienna, and the Center for Experimental Research on Natural Language at the University of Wrocław. The project examined the ways in which number, as a cognitive category, as well as various numerical operations are incorporated into grammars of Slavic in comparison with other languages.

Early versions of many of the contributions making up this book were first presented as papers at the Number, Numerals and Plurality workshop organized at the 12th conference on Syntax, Phonology and Language Analysis (SinFonIJA~12), which was held at Masaryk University in Brno on September 12--14, 2019 (the program of the conference can be found online: \url{https://sites.google.com/phil.muni.cz/sinfonija12/program}). The workshop aimed at a maximum of theoretical diversity and broad empirical coverage, features that we hope are maintained in this book. Encouraged by the success of the workshop and the quality of the papers presented, we invited selected authors as well as other researchers to address four coherent topics within the study of number in natural language: (i)~plurality, number and countability, (ii)~collectivity, distributivity and cumulativity, (iii)~numerals and classifiers, and (iv)~other quantifiers. The proposed collective monograph gathers peer-reviewed contributions exploring those themes both in Slavic and non-Slavic languages. Each of the chapters completed the two-round double-blind review process in which every paper was evaluated and commented on by two reviewers.

This book would not have been possible without our extremely helpful reviewers: Boban Arsenijević, Joanna Błaszczak, Lisa Bylinina, Pavel Caha, Lucas Champollion, Luka Crnič, Flóra Lili Donáti, Kurt Erbach, Suzana Fong, Jovana Gajić, Ljudmila Geist, Scott Grimm, Piotr Gulgowski, Andreas Haida, Nina Haslinger, Dorota Klimek-Jankowska, Heidi Klockmann, Ivona Kučerová, Caitlin Meyer, Olav Mueller-Reichau, Rick Nouwen, Roumyana Pancheva, Lilla Pintér, Wiktor Pskit, Magdalena Roszkowski, Viola Schmitt, Yasutada Sudo, Balázs Surányi, Peter Sutton, Yuta Tatsumi, Barbara, Tomaszewicz-Özakın, Tue Trinh, Hanna de Vries, Kata Wohlmuth and Eytan Zweig. The whole book was reviewed by Jakub Dotlačil. Many thanks for your reviews! Furthermore, we would also like to most sincerely thank the OSL handling editors Berit Gehrke and Radek Šimík for their continuous and extensive help and support in making this book happen. We are also grateful to Chris Rance for proofreading the English text. Finally, we wish to acknowledge the technical support of the entire Language Science Press editorial team as well as the help of everyone else who contributed by type-setting and proofreading parts of the contents of this book. We hope that the readers will find it interesting and inspiring. This book is dedicated to the memory of Joanna Błaszczak who passed away shortly before its publication. She will be missed.\bigskip\\

\hfill Mojmír Dočekal \& Marcin Wągiel

\hfill Brno, 8 July 2021

{\sloppy\printbibliography[heading=subbibliography,notkeyword=this]}
\end{document}
