\documentclass[output=paper]{langscibook} 

\author{Andreas Haida\affiliation{The Hebrew University of Jerusalem} and  Tue Trinh\affiliation{Leibniz-Zentrum Allgemeine Sprachwissenschaft}}
\title{Splitting atoms in natural language}  
\abstract{The classic Fregean analysis of numerical statements runs into problems with sentences containing non-integers such as \textit{John read 2.5 novels}, since it takes such statements to specify the cardinality of a set which by definition must be a natural number. We propose a semantics for numeral phrases which allows us to count mereological subparts of objects in such a way as to predict several robust linguistic intuitions about these sentences. We also identify a number of open questions which the proposal fails to address and hence must be left to future research.

\keywords{numerals, measurement, density, scales, implicatures}}

\begin{document}
\SetupAffiliations{output in groups = true}
\maketitle

\section{A new semantics for numeral phrases}

\subsection{Problems with the Fregean analysis}

Standard analyses of numerical statements have roots in \citet{frege1884grundlagen} and take these to be, essentially, predications of second order properties to concepts, that is specifications of cardinalities. Thus, the sentence 

\ea 
John read 3 novels.
\label{hai-tri:3novels}
\z

\noindent is considered to be a claim about the set of novels that John read, namely that it has three members. The truth condition of \REF{hai-tri:3novels} is taken to be either \REF{hai-tri:eq3} or \REF{hai-tri:geq3}, depending on whether the `exact' or the `at least' meaning is assumed to be basic for numerals.\footnote{For arguments that numerals have the `at least' meaning as basic, see \citet{horn1972semantic,fintelheim1997classnotes,fintelfox2002classnotes,fox2007class}, among others. For arguments that numerals have the `exact' meaning as basic, see \citet{geurts2006take,breheny2008new}, among others. Note that the choice between these two views does not affect what we say in this chapter, as will be clear presently.}

\ea
\ea $|\{x \mid \text{$x$ is a novel $\wedge$ John read $x$}\}| = 3$ 
\label{hai-tri:eq3}
\ex $|\{x \mid \text{$x$ is a novel $\wedge$ John read $x$}\}| \geq 3$ 
\label{hai-tri:geq3}
\z
\z

\noindent Let us now consider \REF{hai-tri:2.5-1}, which we take to be an expression that is accepted as a well-formed sentence of English. 

\ea 
John read 2.5 novels. 
\label{hai-tri:2.5-1}
\z

\noindent Extending the traditional analysis of numerical statements to this sentence yields absurdity: \REF{hai-tri:eq2.5} is a contradiction, and \REF{hai-tri:geq2.5} is logically equivalent to \REF{hai-tri:geq3}.

\ea 
\ea $|\{x \mid \text{$x$ is a novel that John read}\}| = 2.5$ 
\label{hai-tri:eq2.5}
\ex $|\{x \mid \text{$x$ is a novel that John read}\}| \geq 2.5$
\label{hai-tri:geq2.5}
\z
\z

\noindent It is obvious that \REF{hai-tri:2.5-1} is neither contradictory nor equivalent to \REF{hai-tri:3novels}. Suppose, for example, that John read \textit{ Brothers Karamazov}, \textit{ Crime and Punishment}, one-half of \textit{ Demons}, and nothing else.\footnote{John is a Dostoyevsky enthusiast.} In this context, \REF{hai-tri:2.5-1} is true and \REF{hai-tri:3novels} false. The fact that \REF{hai-tri:2.5-1} can be true shows that it is not contradictory, and the fact that it can be true while \REF{hai-tri:3novels} is false shows that the two sentences are not equivalent.

We believe there is no sense in which we can ``extend'' Frege's theory to include non-integers: the number of objects which fall under a concept must be a whole number. For Frege, the concept of a ``concept'' entails, as a matter of logic, that it has sharp boundary: ``[...] so wird ein unscharf definirter Begriff mit Unrecht Begriff genannt [...] Ein beliebiger Gegenstand $\Delta$ f\"{a}llt entweder unter den Begriff $\Phi$, oder er f\"{a}llt nicht unter ihn: tertium non datur'' \citep[][\S 56]{frege1893grundgesetze}.\footnote{In English: ``[...] it is therefore wrong to call a vaguely defined concept a concept [...] For any object, either it falls under the concept or it doesn't: tertium non datur'' (our translation).} In fact, Frege considers the reals to be of a different metaphysical category from the naturals, and even made the distinction notationally explicit, writing ``2'' for the real number two and ``\st{2}'' for the natural number two \citep{snyder2016counting, snyder&shapiro2016frege}. 

At this point, an issue concerning the type of expressions we are investigating should be addressed. In \citet{salmon1997}, phrases such as  $\textit{2}\,\nicefrac{\textit{1}}{\textit{2}}$ \textit{oranges}, which the author pronounces as `two and one-half oranges', are discussed. Here we are dealing with expressions like \textit{2.5 novels} which are pronounced, we suppose, as `two point five novels'. We do not intend to suggest that the two types of expressions should receive the same analysis. An anonymous reviewer raises the question of whether some of our judgements might be an artefact of this pronunciation, i.e. of pronouncing \textit{2.5} as `two point five' instead of `two and a half', for example. This issue, we must admit, goes beyond the scope of our chapter. We would note, however, that independently of how the issue is settled empirically, the fact that it is raised might be symptomatic of a worry which, as we surmised from various discussions, is shared by a number of colleagues. The worry is that we are not investigating ``natural language'', but instead, are ruminating on some sort of conventional discourse which has been manufactured for the special purpose of making conversation in mathematics more expedient. A question which we have heard more than once is ``what about languages spoken by communities which have no mathematics at this level?'' We believe the worry is unfounded. It is true that we have to learn how to write and pronounce decimals, but the linguistic judgements involving these expressions which we present and try to account for below do not come about by way of instruction. In fact, these intuitions should be surprising given the definitions we learn in school. As for the question about languages without expressions for decimals, we would say that our study is similar in kind to one of, say, the Vietnamese pronominal system which can express many distinctions that are not lexically encoded in English. Speech communities may differ, due to historical accidents, in how they lexicalize conceptual space, i.e. in what they can say, but this is of course no reason for assuming that research into language particular phenomena does not inform our understanding of what they \textit{ could} say, i.e. of universal grammar.


\subsection{The proposal} 

This chapter proposes an analysis of numeral phrases which can account for intuitions about such sentences as \REF{hai-tri:2.5-1}. First, we will assume the logical form of \textit{John read 2.5 novels} to be the structure shown in Figure \Ref{hai-tri:2.5-2}, where {\sc some} and {\sc many} are covert \citep[cf.][]{Hackl:2000}.\footnote{Although we reference \citet{Hackl:2000}, we should note that existential quantification, i.e. the meaning of {\sc some}, is included in the definition of Hackl's {\sc many}. We thank an anonymous reviewer for reminding us to mention this difference.}

\begin{figure}[h!]
\begin{forest} [$\alpha$ [DP [ {\sc some} ] [NumP, nice empty nodes [ [ 2.5 ] [ {\sc many} ] ] [ novels ] ] ] [$\beta$, calign=fixed angles [ $\lambda_x$ ] [S [ John ] [VP [ read ] [ \hspace{3pt}$t_x$ ] ] ] ] ]
\end{forest}
\caption{The logical form of \REF{hai-tri:2.5-1}}
\label{hai-tri:2.5-2}
\end{figure}

\noindent Our proposal will consist in formulating a semantics for {\sc many}, leaving other elements in Figure \Ref{hai-tri:2.5-2} with their standard meaning.\footnote{In particular, we assume that the covert {\sc some} has the same meaning as its overt counterpart, which is $\eval{\textsc{some}}{} = \eval{\texttt{some}}{} = [\lambda P \in \mathcal{D}_{\langle e,t \rangle}\;.\;[\lambda Q \in \mathcal{D}_{\langle e,t \rangle}\;.\;\exists x\;.\;P(x) = Q(x) = 1]]$.} This semantics presupposes the fairly standard view of the domain of individuals, ${\mathcal D}_e$, as a set partially ordered by the part-of relation $\sqsubseteq$ to which we add $\emptyset$ as the least element \citep[cf.][]{Link:1983, Landman:1989, Schwarzschild:1996, bylinina&nouwen2018zero}.\footnote{We do not assume that $\emptyset$ is an element of ${\mathcal D}_e$ itself. Neither do we exclude this possibility.}
The join operation $\sqcup$ and the meet operation $\sqcap$ on $\langle{\mathcal D}_e\cup\{\emptyset\},\sqsubseteq\rangle$ are given the usual definitions below, where  $\iota$ represents, following standard practice, the function mapping a singleton set to its unique element.

\ea 
\ea $x\sqcup y :=\iota\{z\mid x\sqsubseteq z\wedge y\sqsubseteq z\wedge\forall z'(x\sqsubseteq z'\wedge y\sqsubseteq z'\rightarrow z\sqsubseteq z')\}$
\ex $x\sqcap y :=\iota\{z\mid z\sqsubseteq x\wedge z\sqsubseteq y\wedge\forall z'(z'\sqsubseteq x\wedge z'\sqsubseteq y\rightarrow z'\sqsubseteq z)\}$
\z
\z
 
\noindent We assume that plural nouns denote cumulative predicates, i.e. subsets of $\mathcal{D}_e$ which are closed under $\sqcup$ \citep[cf.][]{Krifka:1989, Chierchia:1998, Krifka:2003, sauerland&al2005plural, spector2007aspects, zweig2009number, Chierchia:2010}. For each predicate $A$, the set of $A$ atoms, $A_{at}$, is defined as

\ea 
$A_{at} := \{x \in A \mid \neg\exists y\;.\;y \sqsubset x \wedge y \in A\}$. 
\label{hai-tri:atdef}
\z

\noindent To illustrate, let $b$ and $c$ be the two novels \textit{ Brothers Karamazov} and \textit{ Crime and Punishment}, respectively. The individual $b \sqcup c$ has proper parts that are novels, hence $b \sqcup c$ will not be in $\eval{\texttt{novels}}{}_{at}$. In contrast, neither $b$ nor $c$ has proper parts that are novels, hence both of these individuals are in $\eval{\texttt{novels}}{}_{at}$. In other words, $\eval{\texttt{novels}}{}_{at}$ contains things that we can point at and say `that is a novel'. The semantics we propose for {\sc many} is \REF{hai-tri:many}, where $d$ ranges over degrees.

\ea 
$\eval{\mbox{\sc many}}{}(d)(A) = [\lambda x\in{\mathcal D}_e\;.\;\mu_A(x) \ge d]$  
\label{hai-tri:many}
\z

\noindent We then predict that \textit{John read 2.5 novels} is true iff there exists an individual $x$ such that $\mu_{\eval{\texttt{novels}}{}}(x) \ge 2.5$ and John read $x$. The term $\mu_{\eval{\texttt{novels}}{}}(x)$ represents `how many novels are in $x$', so to speak. We want to be able to count novels in such a way that proper subparts of novels, which are not novels, also contribute to the count. To this end, we propose to explicate the measure function $\mu_A$ as follows.\footnote{\citet{salmon1997} tentatively suggests to analyze ``$2\,\nicefrac{1}{2}$'' by means of the quantifier `$2.5$' in a logical form like `$2.5x(\mbox{$x$ is an $F$ that is $G$})$'. This quantifier is characterized as a `mixed-number quantifier', operating on pluralities, where the quantity of a plurality is measured in such a way that whole $F$s count as one and ``a part of a whole $F$ counts for part of a whole number.'' Our proposal can be seen as an order-theoretic specification of such a quantifier.}

\ea \label{hai-tri:mu-A}
$\mu_A(x)=
\left\{\begin{array}{ll}
%\left\{\begin{array}{ll}
\mu_A(y)+1 & \mbox{if $a\sqsubset x$, $y\sqcup a=x$, and $y\sqcap a=\emptyset$ for some $a \in A_{at}$}\\
\mu_a(x)   & \mbox{if $x\sqsubseteq a$ for some $a \in A_{at}$}\\
\#  & \mbox{otherwise}
\end{array}\right.$
\z

\noindent Thus, each $A$ atom which is a subpart of $x$ will add 1 to $\mu_A(x)$. If $x$ is an $A$ atom or a subpart of an $A$ atom, $\mu_A(x)$ will be $\mu_a(x)$, which represents `how much of the $A$ atom $a$ is in $x$', so to speak. The measure function $\mu_a$ is explicated as follows.
     
\ea \label{hai-tri:mu-a}
For each $a \in A_{at}$, 
\ea $\mu_a$ is a surjection from $\{x\in{\mathcal D}_e\mid x\sqsubseteq a\}$ to $(0,1] \cap\mathbb{Q}$
\ex $\mu_a(x\sqcup y) = \mu_a(x)+\mu_a(y)$ for all $x,y\in\mbox{dom}(\mu_a)$ such that $x\sqcap y=\emptyset$ \label{hai-tri:mu-a-b}
\ex $\mu_a(a)=1$
\z
\z

\noindent This definition allows us to use any positive rational numbers smaller or equal to 1 to measure parts of an atom, with 1 being the measure of the whole atom. Furthermore, it guarantees that the measurement of parts of an atom is additive: if $x$ and $y$ are non-overlapping parts of an atom, their mereological sum $x\sqcup y$ measures the arithmetic sum of the measurements of $x$ and $y$. Thus, two chapters, chapters 1 and 2, of a novel cannot be added to two chapters, chapters 2 and 3, of the same novel to give four chapters of that novel because of the overlap.

Two points should be noted about the definition in \REF{hai-tri:mu-A}. First, it follows from it that $\mu_A(x)$ is undefined (for all $x$) if $A_{at}$ is empty. An anonymous reviewer raises the concern that this definition might exclude the denotation of count nouns like \textit{fence} from being measured by $\mu$, the problem being that fences are homogeneous entities. That is, the concern is that $\eval{\texttt{fence}}{}_{at} = \emptyset$ and, consequently, that $\mu_{\eval{\texttt{fence}}{}}(x) = \#$. We hypothesize that measuring this type of noun requires contextual restriction: if ${\mathcal C}$ is a syntactic variable and $\eval{\texttt{fence}_{\mathcal C}}{g} = \eval{\texttt{fence}}{g} \cap g({\mathcal C})$, then $\eval{\texttt{fence}_{\mathcal C}}{g}_{at} \ne \emptyset$ iff $\eval{\texttt{fence}}{g} \cap g({\mathcal C})_{at} \ne \emptyset$; consequently, $\mu_{\eval{\texttt{fence}_{\mathcal C}}{}}(x)$ is defined if (and only if) $\eval{\texttt{fence}}{g} \cap g({\mathcal C})_{at} \ne \emptyset$ (for certain $x$). Thus, we surmise that sentences like \textit{Ann passed by 3 fences} or \textit{Ann painted 3.5 fences} presuppose a context in which fences aren't homogeneous entities but maximal stretches of fence, such as the whole stretch of a fence around a property or along a border. Thus, we agree with \citet{wagiel2018thesis} that counting can involve a notion of `maximality'. However, we put forth the hypothesis that maximality only comes into play through contextual restriction, in the absence of atoms in the unrestricted extension of a noun. 

Second, note that overlap is dealt with twice in our definitions, viz. in the first clause of \REF{hai-tri:mu-A}, to prevent atoms from being counted more than once, and in \REF{hai-tri:mu-a-b}, to do the same for subatomic parts. This is in line with the claim that subatomic quantification is subject to the same constraints as quantification over wholes \citep{wagiel2018thesis, wagiel2019SuB}. However, we are not committed to all aspects of W\k{a}giel's theory. Specifically, we see reason to reject his claim that counting (of atoms and subatomic parts) requires `topological integrity.' It seems to us that the sentence \textit{John owns 2 cars} can be much more readily accepted as true if John owns (nothing but) a whole car and a car that is sitting disassembled in various places in his garage than the sentence \textit{John owns two cups} if he owns (nothing but) a whole cup and the shards of a shattered cup. While some notion of `integrity' might play into this contrast, we believe that the way this notion enters is by affecting, dependent on context, what is considered a possible extension of the nouns \textit{car} and \textit{cup} in the actual world. A more thorough comparison of our proposal to W\k{a}giel's theory is beyond the bounds of this chapter but we believe that the two proposals are largely compatible.

Before we discuss some predictions of our proposal, it should be said that the need for non-integral counting in natural language has been recognized. \citet{kennedy2015de-fregean}, for example, says the following about \#, the measure function which maps objects to number: ``Note that \# is not, strictly speaking, a cardinality function, but rather gives a measure of the size of the (plural) individual argument of the noun in ``natural units'' based on the sense of the noun [...]. If this object is formed entirely of atoms, then \# returns a value that is equivalent to a cardinality. But if this object contains parts of atoms, then \# returns an appropriate fractional or decimal measure [...]'' \citep[][footnote 1]{kennedy2015de-fregean}. However, this is all Kennedy says about the matter. In particular, he does not explicate what he means by ``appropriate'', and is not concerned with the data that we present below. The notion of ``natural units'' refered to by Kennedy in the quote above is due to \citet{Krifka:1989}, who proposes a function, NU, which maps a predicate $P$ and an object $x$ to the number of natural units of $P$ in $x$. Like Kennedy, Krifka does not consider the data presented in the next section, and neither does he provide a definition of NU which is explicit enough to relate to them. In fact, Krifka stipulates that NU is an `extensive measure function', on the model of such expressions as \textit{litter of}, which means he actually makes the wrong prediction for the data point presented in \sectref{hai-tri:sec:pred2}. below. Specifically, Krifka will predict that \REF{hai-tri:0.5+0.25} must be contradictory as \REF{hai-tri:1+1} is. Thus, what we are doing here is essentially improving upon Kennedy and Krifka, with the improvement being explication in the former and explication as well as correction in the latter case.

\section{Some predictions of the proposal}\label{hai-tri:sec:predictions}

This section presents some intuitions about numerical statements which are predicted by our semantics for {\sc many}. The list is not intended to be exhaustive.%\vspace{-18pt}

\subsection{First prediction} We predict the observation made at the beginning of this chapter, namely that \REF{hai-tri:2.5-3} is neither contradictory nor equivalent to \REF{hai-tri:3-3}.

\ea
\ea John read 2.5 novels. 
\label{hai-tri:2.5-3}
\ex John read 3 novels. 
\label{hai-tri:3-3}
\z
\z

\noindent This is because $\mu_{\eval{\texttt{novels}}{}}(x) \ge 2.5$ is neither contradictory nor equivalent to $\mu_{\eval{\texttt{novels}}{}}(x) \ge 3$. To see that $\mu_{\eval{\texttt{novels}}{}}(x) \ge 2.5$ is not contradictory, let $b$, $c$, and $d$ be, again, the three novels \textit{ Brothers Karamazov}, \textit{ Crime and Punishment}, and \textit{ Demons}, respectively, and let $d'$ be a subpart of \textit{ Demons} which measures one-half of this novel, so that $\mu_{\eval{\texttt{novels}}{}}(d') = \mu_d(d') = 0.5$. Then, $\mu_{\eval{\texttt{novels}}{}}(b\sqcup c \sqcup d') = \mu_{\eval{\texttt{novels}}{}}(c \sqcup d') + 1 = \mu_{\eval{\texttt{novels}}{}}(d') + 1 + 1 = \mu_{d}(d') + 1 + 1 = 0.5 + 1 + 1 = 2.5$. The non-equivalence follows from the logical truth that $2.5 < 3$ and the fact that there is an $x$ such that $\mu_{\eval{\texttt{novels}}{}}(x)=2.5$ (as shown above).%\vspace{-18pt}

\subsection{Second prediction} \label{hai-tri:sec:pred2}

We predict that \REF{hai-tri:1+1} is a contradiction but \REF{hai-tri:0.5+0.25} is not.

\ea 
\ea[\#]{John read 1 Dostoyevsky novel yesterday, and 1 Tolstoy novel today, but he did not read 2 Russian novels in the last two days.} 
\label{hai-tri:1+1}
\ex[]{John read 0.5 Dostoyevsky novels yesterday, and 0.25 Tolstoy novels today, but he did not read 0.75 Russian novels in the last two days.}
\label{hai-tri:0.5+0.25}
\z
\z

\noindent The first conjunct of \REF{hai-tri:1+1} requires two different novels, say $b$ and $c$, to have been read by John.\footnote{Here and below, we refer to the conjuncts of \textit{but}.} As $\mu_{\eval{\texttt{novels}}{}}(b \sqcup c) = 2$, the second conjunct of \REF{hai-tri:1+1} contradicts the first. On the other hand, suppose John read a subpart of $b$, call it $b'$, yesterday and read a subpart of $c$, call it $c'$, today, and suppose that $b'$ measures one-half of $b$ and $c'$ measures one-quarter of $c$, i.e. $\mu_{\eval{\texttt{novels}}{}}(b') = \mu_b(b') = 0.5$ and $\mu_{\eval{\texttt{novels}}{}}(c') = \mu_c(c') = 0.25$. Then the first conjunct of \REF{hai-tri:0.5+0.25} is true. However, $b'$ and $c'$, put together, do not make up something which has a subpart that is a novel, or something which is a subpart of a novel. In other words, there is no $a \in \eval{\texttt{novels}}{}_{at}$ such that $a \sqsubset b' \sqcup c'$ or $b' \sqcup c' \sqsubseteq a$, which means $\mu_{\eval{\texttt{novels}}{}}(b' \sqcup c') = \#$, which means $\mu_{\eval{\texttt{novels}}{}}(b' \sqcup c') \not\geq 0.75$, which means the second conjunct of \REF{hai-tri:0.5+0.25} is true.

Note that our prediction in this case differs from that of \citet{liebesman2016counting}, who would predict that \textit{John read 0.75 novels} is true in the described context, since Liebesman's proposal, according to our understanding, would allow subparts of different novels to be added, as long as the sum is smaller than 1. Furthermore, judgments might be different for an example like \REF{hai-tri:orange}, which seems to have a contradictory reading.

\ea[\#]{John ate 0.5 oranges yesterday, and 0.25 oranges today, but he did not eat 0.75 oranges (or more) in the last two days.}\label{hai-tri:orange}
\z

\noindent We believe that the difference between \REF{hai-tri:0.5+0.25} and \REF{hai-tri:orange} comes down to the fact that \textit{orange} can be more easily coerced to a mass interpretation than \textit{novel} (cf. \textit{The smoothie contains orange} vs. \#\textit{The shredder bin contains novel}). To accommodate the contradictory reading of \REF{hai-tri:orange}, we tentatively assume that $\eval{\texttt{oranges}}{}$ can be contextually extended by sums of subparts of different oranges.

\subsection{Third prediction} 

We predict that \REF{hai-tri:tautology} is a tautology.

\ea If John read 0.75 novels, and Mary read the rest of the same novel that John was reading, then Mary read 0.25 novels. 
\label{hai-tri:tautology}
\z

\noindent Suppose John read a portion of $b$, call it $b'$, which measures three-fourths of $b$, so that $\mu_{\eval{\texttt{novels}}{}}(b') = \mu_b(b') = 0.75$.  Suppose, furthermore, that Mary read the rest of $b$, call it $b''$, which is all of that part of $b$ which John did not read. Then the antecedent is true. Now by hypothesis, $b' \sqcup b'' = b$, and $b \in \eval{\texttt{novels}}{}_{at}$. This means $\mu_b(b' \sqcup b'') = \mu_b(b) = 1$. Since $b'$ and $b''$ do not overlap, i.e. $b' \sqcap b'' = \emptyset$, we have $\mu_b(b' \sqcup b'') = \mu_b(b') + \mu_b(b'') = 1$. And because $\mu_b(b') = 0.75$, we have $\mu_b(b'') = 1 - 0.75 = 0.25$, hence $\mu_{\eval{\texttt{novels}}{}}(b'') = 0.25$, which means the consequent is true.%\vspace{-18pt}

\subsection{Fourth prediction} 

We predict that \REF{hai-tri:notcontradiction} is not a contradiction.

\ea John read 0.5 novels, and Mary read 0.25 of the same novel that John was reading, but John and Mary together did not read 0.75 novels. 
\label{hai-tri:notcontradiction}
\z

\noindent Suppose John read $b'$ which measures 0.5 of $b$, and Mary read $b''$ which measures 0.25 of $b$. Thus, $\mu_{b}(b') = 0.5$ and $\mu_{b}(b'') = 0.25$. The first conjunct is then true. Now let $b'$ and $b''$ overlap, so that $b' \sqcap b'' \neq \emptyset$. Furthermore, let $o$ be $b'\sqcap b''$ and $d'$ and $d''$ the non-overlapping parts of $b'$ and $b''$, respectively. Thus, $b'= d'\sqcup o$, $b''= d''\sqcup o$, and $b'\sqcup b''=d'\sqcup d''\sqcup o$. This means $\mu_{b}(b'\sqcup b'')=\mu_{b}(d'\sqcup d''\sqcup o)=\mu_{b}(d')+\mu_{b}(d'')+\mu_{b}(o)<\mu_{b}(d')+\mu_{b}(o)+\mu_{b}(d'')+\mu_{b}(o)=\mu_{b}(d'\sqcup o)+\mu_{b}(d''\sqcup o)=\mu_{b}(b')+\mu_{b}(b'')=0.5+0.25=0.75$, which means $\mu_{b}(b'\sqcup b'') < 0.75$, which hence means the second conjunct is true. %\vspace{-18pt}

\subsection{Fifth prediction} \label{hai-tri:sec:pred5}

We predict that \REF{hai-tri:0.5novels} is coherent, but \REF{hai-tri:0.5quantities} is not.\footnote{Note that the word \textit{quantity} in \REF{hai-tri:0.5quantities} is not intended to mean `200 pages', or `3000 words', or any contextually specified quantity of literature. The intended meaning of \textit{quantity} here is the lexical and context-independent one.}

\ea
\ea[]{John read (exactly) 0.5 novels.}
\label{hai-tri:0.5novels}
\ex[\#]{John read (exactly) 0.5 quantities of literature.}
\label{hai-tri:0.5quantities}
\z
\z

\noindent That \REF{hai-tri:0.5novels} is coherent is, by now, obvious. It will be true if John read, say, half of \textit{ Anna Karenina}. What makes \REF{hai-tri:0.5quantities} incoherent, then, must lie in the semantics of \textit{quantities of literature}, henceforth {\ttfamily qol} for short. According to the semantics we proposed for {\sc many}, \REF{hai-tri:0.5quantities} entails the existence of an individual $x$ such that $\mu_{\eval{\texttt{qol}}{}}(x) = 0.5$, which entails the existence of some $a \in \eval{\texttt{qol}}{}_{at}$ such that $x \sqsubseteq a$. Given that any subpart of a quantity of literature is itself a quantity of literature, we have $\eval{\texttt{qol}}{}_{at} = \{x \in \eval{\texttt{qol}}{} \mid \neg\exists y \sqsubset x \wedge y \in \eval{\texttt{qol}}{}\} = \emptyset$. Thus, there is no $a \in \eval{\texttt{qol}}{}_{at}$, which means there is no $x$ such that $\mu_{\eval{\texttt{qol}}{}}(x) = 0.5$, which means \REF{hai-tri:0.5quantities} is false. Furthermore, it is analytically false, which is to say false by virtue of the meaning of the word \textit{quantity}. This, we hypothesize, is the reason for its being perceived as deviant. We will come back to this point in the last section.%\vspace{-18pt}

\subsection{Sixth prediction} \label{hai-tri:sec:pred6}

We predict \REF{hai-tri:nogap}, which we claim to be a fact about natural language. 

\ea
There is no numerical gap in the scale which underlies measurement in natural language.
\label{hai-tri:nogap}
\z

\noindent What \REF{hai-tri:nogap} is intended to say, illustrated by a concrete example, is that to the extent \textit{John read 2.5 novels} is meaningful, \textit{John read 2.55 novels} is too, as well as \textit{John read 2.555 novels}, or any member of $\{\texttt{John read n novels} \mid \eval{\texttt{n}}{} \in \mathbb{Q}^+\}$.\footnote{Where $\mathbb{Q}^+$ are the positive rationals. Thus, \REF{hai-tri:nogap} should really be qualified with the phrase ``as far as rational numbers are concerned'', as pointed out by an anonymous reviewer, who raises the issue of irrational numbers. We refer the reader to \sectref{hai-tri:sec:reals} for more discussion on this point. Here we would only note that by ``meaningful'', we mean the sentence has non-trivial truth condition, and licenses inferences, as shown for \textit{John read 2.5 novels} in the last section.} This follows from the fact that 0.5, as well as 0.55, as well as 0.555, as well as any other rational number in $(0,1] \cap\mathbb{Q}$, are all in the range of $\mu_a$, for any $a \in \eval{\texttt{novels}}{}_{at}$. This fact, in turn, follows from the fact that $\mu_a$ is, by stipulation, a function onto $(0,1] \cap\mathbb{Q}$. Note, importantly, that we cannot guarantee \REF{hai-tri:nogap} by stipulating, merely, that the set of degrees underlying measurement in natural language is dense. To see that density alone does not exclude gaps, consider the set in \REF{hai-tri:dense+gap}.

\ea  
$S :=$ $\mathbb{Q}^+ \backslash \{x \in \mathbb{Q} \mid 3 < x \leq 4\}$
\label{hai-tri:dense+gap}
\z

\noindent This is a dense scale, as between any two elements of $S$ there is an element of $S$. However, $S$ contains a gap: missing from it are numbers greater than 3 but not greater than 4, for example 3.5. Merely stipulating that the scale is dense, therefore, will not guarantee that \textit{John read 3.5 novels} is meaningful, which we claim is a robust intuition that linguistic theory has to account for.

Note that \citet{foxhackl2006universal}, according to our understanding, seems to assume that density of a scale alone guarantees the absense of gaps in it. The authors claim, for example, that density guarantees that exhaustification of \textit{John has more than 3 children} would negate every element of $\{\texttt{John has more than n children} \mid \texttt{n} \in \mathbb{Q} \wedge \texttt{n} > 3\}$. We quote from page 543 of \citet{foxhackl2006universal}: ``Without the UDM [i.e. the assumption that the set of degrees is dense], [...] [t]he set of degrees relevant for evaluation would be, as is standardly assumed, possible cardinalities of children (i.e. 1, 2, 3, ...). The sentence would then assert that John doesn't have more than 4 children [...] If density is assumed, however, [...] the assertion would now not just exclude 4 as a degree exceeded by the number of John’s children. It would also exclude any degree between 3 and 4.'' Taken at face value, this claim is wrong, as is evident from the example in \REF{hai-tri:dense+gap}. %\vspace{-18pt}

\subsection{Seventh prediction} 

On the assumption that overt \textit{many} and the comparative \textit{more} instantiate {\sc many}, we predict that the argument expressed by the sequence in \REF{hai-tri:ManyComp} is is invalid.

\ea John read 2.5 novels and Mary read 2 novels. \#Therefore, John read more novels than Mary.\label{hai-tri:ManyComp}
\z

\noindent By the definitions in \REF{hai-tri:many} and \REF{hai-tri:mu-A}, the scale $[\lambda x\lambda d.\,\eval{\mbox{\sc many}}{}(d)(\eval{\mbox{\ttfamily novels}}{})(x)]$ is non-monotonic.\footnote{Let $S$ be a scale, conceived of as a function from entities and degrees to truth values, such that for all $x$ the degree function $S(x)$ is monotonic (i.e. such that $S(x)(d)\rightarrow S(x)(d')$ for all $d,d'$ such that $d'\le d$). Then, the scale $S$ is monotonic iff $S(x)(d)=1\rightarrow S(x')(d)=1$ for all $d$ and $x,x'$ such that $x\sqsubseteq x'$ (cf. \citealt{Krifka:1989,Schwarzschild:2002}).} For instance, if $b'$ is half of \textit{ Brothers Karamazov} and $c'$ half of \textit{ Crime and Punishment}, then $[\lambda x\lambda d.\,\eval{\mbox{\sc many}}{}(d)(\eval{\mbox{\ttfamily novels}}{})(x)](b')(0.5)=1$ but $[\lambda x\lambda d.\,\eval{\mbox{\sc many}}{}(d)(\eval{\mbox{\ttfamily novels}}{})(x)](b'\sqcup c')(0.5)=0$. Therefore, \REF{hai-tri:ManyComp} is not valid, since it would only be valid if the scale were monotonic, i.e. were a scale of comparison \citep{WellwoodEtAl:2012}. 

This is illustrated in \REF{hai-tri:non-mon}. The temperature scale is non-monotonic. Hence, the temperature scale cannot function as the scale of comparison of the comparative in the second sentence of \REF{hai-tri:non-mon-a}. Therefore, the sequence of the two sentences in \REF{hai-tri:non-mon-a} is an invalid argument. The weight scale, in contrast, is monotonic.  Hence, the weight scale can function as the scale of comparison of the comparative in the second sentence of \REF{hai-tri:non-mon-b}, as evidenced by the validity of the argument expressed by \REF{hai-tri:non-mon-b}.

\ea \label{hai-tri:non-mon}
\ea John ate 90 degree hot spaghetti and Mary 70 degree hot spaghetti. \#Therefore, John ate more spaghetti than Mary. \label{hai-tri:non-mon-a}
\ex John ate 500 grams of spaghetti and Mary ate 200 grams of spaghetti. Therefore, John ate more spaghetti than Mary. \label{hai-tri:non-mon-b}
\z
\z

\noindent To account for the fact that the arguments in \REF{hai-tri:ValidArgs} are valid, we tentatively assume that {\sc many} can be restricted to atoms and sums of atoms in equatives and comparatives.

\ea\label{hai-tri:ValidArgs}
\ea John read 3.5 novels and Mary read 2 novels. Therefore, John read more novels than Mary.
\ex John read 2.5 novels and Mary read 2 novels. Therefore, John read as many novels as Mary.
\z
\z

\noindent This means to say that the scale of comparison of {\ttfamily more than}/{\ttfamily as many as} in \REF{hai-tri:ValidArgs} is the monotonic scale $[\lambda x\in\eval{\mbox{\ttfamily novels}}{\sqcup}_{at}.\lambda d.\,\eval{\mbox{\sc many}}{}(d)(\eval{\mbox{\ttfamily novels}}{})(x)]$ (where $A_{at}^{\sqcup}$ is the closure of $A_{at}$ under the join operation).

\section{Excursus: Conditions on predicates}

The semantics we propose for {\sc many}, as presented in \REF{hai-tri:many}, \REF{hai-tri:mu-A} and \REF{hai-tri:mu-a}, requires that for each atom $a$ of a predicate $A$ the measure function $\mu_a$ have $(0,1]\cap\mathbb{Q}$ as its range, and be additive with respect to non-overlapping subparts of atoms.

\ea Conditions on $\mu_a$
\ea \cnst{ran}$(\mu_a) = (0,1]\cap\mathbb{Q}$ 
\label{hai-tri:range}
\ex $\mu_a(x \sqcup y) = \mu_a(x) + \mu_a(y)$ if $x, y \sqsubseteq a$ and $x \sqcap y = \emptyset$ 
\label{hai-tri:additivity}
\z
\z

\noindent This section details the conditions under which such measure functions $\mu_a$ exist. While it is possible to derive empirical predictions from these conditions (see footnote \Ref{hai-tri:fn:indivisible} below), which could have been added to \sectref{hai-tri:sec:predictions}, the main purpose of the current section is to tie in our proposal with a general theory of measurement. Conditions on the existence of measure functions $\mu_a$ of the right kind are conditions on subsets $A$ of $\mathcal{D}_e$ with $A_{at}\ne\emptyset$ such that for each $a\in A_{at}$ there is a function $\mu_a$ that satisfies \REF{hai-tri:range} and \REF{hai-tri:additivity}. Call such subsets of $\mathcal{D}_e$ `measurable predicates'.

Let $A$ be an arbitrary subset of $\mathcal{D}_e$ such that $A_{at}\ne\emptyset$. The first assumption we need to make for $A$ to be a measurable predicate is that all of its atoms are divisible into arbitrarily many discrete parts.\footnote{It seems that a stricter condition might be desirable, viz. that every entity is arbitrarily divisible into discrete parts. However, such a condition would afford a notion of \textit{possible division} of an entity and it is doubtful whether such a notion can be defined independently of the partial order $\langle{\mathcal D}_e\cup\{\emptyset\},\sqsubseteq\rangle$.}$^{\text{, }}$\footnote{\label{hai-tri:fn:indivisible}There are predicates whose members withstand being conceived of as being (arbitrarily) divisible. For example, it is hard to conceive of partial results of an achievement. Correspondingly, combining nominalizations of achievement verbs with non-integer nominals leads to deviance:

\ea \label{hai-tri:ex:deviant}
\ea[\#]{Ann fired 3.5 shots.}
\ex[\#]{Bob witnessed 1.5 arrivals.}
\z
\z 
} This is stated in \REF{hai-tri:divisible}, where ${\mathcal P}_a := \{x\in{\mathcal D}_e\mid x\sqsubseteq a\}$.

\ea For all $a \in A_{at}$ and $n\in\mathbb{N}$, there is a set $S \subseteq {\mathcal P}_a$ such that $|S|=n$, $\bigsqcup S=a$, and $\bigsqcap S'=\emptyset$ for all $S'\subseteq S$ with $|S'|>1$
\label{hai-tri:divisible}
\z

\noindent  It follows from \REF{hai-tri:divisible} that no $A$ atom $a$ has a smallest part, and also, that there is no smallest difference between two parts of $a$. This condition is necessary to guarantee that the range of a measure function $\mu_a$ can be the rational interval $(0,1]\cap\mathbb{Q}$, as demanded in \REF{hai-tri:range}.

The second and final assumption we need to make about a measurable predicate $A$ is that its atoms satisfy the condition in \REF{hai-tri:sigma-alg}.

\ea \label{hai-tri:sigma-alg}
For all $a \in A_{at}$, $\langle{\mathcal P}_a,\sqsubseteq\rangle$ is a $\sigma$-algebra on $\langle{\mathcal D}_e\cup\{\emptyset\},\sqsubseteq\rangle$\footnotemark
\z
%
\footnotetext{A partial order $\langle A,\sqsubseteq\rangle$ is a \textsc{$\sigma$-algebra} on a lower bounded partial order $\langle B,\sqsubseteq\rangle$, with $A\subseteq B$, iff (i) it is upper bounded, (ii) closed under complementation, and (iii) closed under countable joins, where $\langle B,\sqsubseteq\rangle$ is \textsc{lower bounded} iff $\bigsqcap B\in B$, and $\langle A,\sqsubseteq\rangle$ is \textsc{upper bounded} iff $\bigsqcup A\in A$, \textsc{closed under complementation} iff for all $x\in A$ there is a $y\in A$ such that $x\sqcup y=\bigsqcup A$ and $x\sqcap y=\bigsqcap B$, and \textsc{closed under countable joins} iff for all countable subsets $S$ of $A$ it holds that $\bigsqcup S\in A$.}

\noindent $\sigma$-algebras are well-known structures of measure theory (see e.g. \citealt{Cohn:1980}) which guarantee, in our case, that the parts of an entity $a$ are measurable in the sense of there being a function $\mu_a$ that satisfies \REF{hai-tri:range} and \REF{hai-tri:additivity}. In simple words, what we require with \REF{hai-tri:sigma-alg} is that each $a \in A_{at}$ satisfy the following conditions: (i) the set of parts of $a$ contains a greatest element (trivially satisfied, since $a$ is a part of itself); (ii) for every (proper) part of $a$, there is another part of $a$, discrete from the first, such that the two parts together are $a$; and (iii) countably many parts of $a$ joined together are a part of $a$. We add another condition to make sure that counting the atoms of a member $x$ of a measurable predicate $A$ is consistent with measuring all of its subatomic parts. For this to be the case, the atoms of $A$ must be pairwise discrete from each other, as stated in \REF{hai-tri:condition-subatomic-parts}.

\ea For all $a, b \in A_{at}$, if $a\sqcap b\ne\emptyset$ then $a=b$\label{hai-tri:condition-subatomic-parts}
\z

\section{Open questions}

We end with some open questions for future research. Again, the list below is not intended to be exhaustive. %\vspace{-18pt}

\subsection{Concepts}  

The semantics we propose for {\sc many} predicts the contrast between \REF{hai-tri:0.5novels} and \REF{hai-tri:0.5quantities}, repeated in \REF{hai-tri:rep0.5novels} and \REF{hai-tri:rep0.5quantities} below, because it entails that to be half an $A$ is to be half an $A$ atom. This semantics, as it is, makes the wrong prediction that \REF{hai-tri:schubert} is false.\footnote{According to an anonymous reviewer, this prediction is not wrong. Specifically, the reviewer says that s/he sees the \textit{Unvollendete} (lit. `unfinished') not as half of a symphony, but as a symphony, hence finds \REF{hai-tri:schubert} to be false. We are not sure to what extent this opinion of the \textit{Unvollendete} can be accounted for within a semantic theory of numerals. Our point concerns the problems faced by our account given the understanding that the \textit{Unvollendete} is not a whole symphony, i.e. is ``unvollendet''. That there is a different understanding is orthogonal to the discussion.}

\ea 
The \textit{ Unvollendete} is 0.5 symphonies.
\label{hai-tri:schubert}
\z

\noindent Let $u$ be the \textit{ Unvollendete}. From \REF{hai-tri:mu-A} and \REF{hai-tri:mu-a}, it follows that $\mu_{\eval{\texttt{symphonies}}{}}(u) \neq 0.5$, as there is no $a \in \eval{\texttt{symphonies}}{}_{at}$ such that $u \sqsubseteq a$. Obviously, modality is involved: while there is no singular symphony $s$ such that $\mu_s(u) = 0.5$, there could be one, since the last two movements could have been completed. Thus, counting symphonies seems to be about what could be a symphony, not what is actually a symphony. In other words, it is concepts, not predicates, that seem to be at play. This means we should, perhaps, revise our semantics so as to predict that to be half an $A$ is to be half of something which is an $A$ atom in some possible world. There is a possible world, say one where Schubert died at 41 instead of 31, in which the \textit{ Unvollendete} is part of a whole symphony, and this is what makes \REF{hai-tri:schubert} true. However, we do not want to predict, incorrectly, that \REF{hai-tri:beethoven} is true, for example.

\ea 
\textit{ Crime and Punishment} is 0.5 symphonies.
\label{hai-tri:beethoven}
\z

\noindent Thus, while there certainly is a possible world $w$ in which \textit{ Crime and Punishment} is a subpart of a symphony, we want $w$ to be inaccessible from the world of evaluation. Plausibly, specifying the relevant accessibility relation in this particular case amounts to fleshing out the concept of `symphony', and specifying it in the general case, to fleshing out the concept of `concept'. We leave this task to future work. %\vspace{-18pt}

\subsection{Analyticities} 

Suppose John read one quarter of \textit{ Brothers Karamazov} and one quarter of \textit{ Crime and Punishment}, our semantics of {\sc many} predicts, correctly, that neither \REF{hai-tri:rep0.5novels} nor \REF{hai-tri:rep0.5quantities} is true.

\ea
\ea[]{John read 0.5 novels.}
\label{hai-tri:rep0.5novels}
\ex[\#]{John read 0.5 quantities of literature.}
\label{hai-tri:rep0.5quantities}
\z
\z

\noindent Both sentences claim of something, which does not exist, that John read one-half of it: in the case of \REF{hai-tri:rep0.5novels}, a novel which contains parts of both \textit{ Brothers Karamazov} and \textit{ Crime and Punishment}, and in the case of \REF{hai-tri:0.5quantities}, an quantity of literature which contains no subpart that is also an quantity of literature. Our semantics, however, does not predict the contrast in acceptability between \REF{hai-tri:rep0.5novels} and \REF{hai-tri:rep0.5quantities}: while the former is perceived as false, the latter is perceived as deviant. In \sectref{hai-tri:sec:pred5}, we said that this contrast has to do with analyticity: it lies in the meaning of the word \textit{quantity} that any subquantity is a quantity, while nothing in the meaning of \textit{novel} rules out a novel which contains parts of both \textit{ Brothers Karamazov} and \textit{ Crime and Punishment}. Analyticity has been appealed to in explanations of deviance \citep[cf.][]{Barwise:1981, vonFintel:1993, krifka1995semantics, abrusan2007contradiction}. However, it has been pointed out that all analyticities are not equal: both \REF{hai-tri:bachelors} and \REF{hai-tri:exceptive} are analytically false, but only the latter is deviant.\footnote{Assuming that \REF{hai-tri:bachelors} has the truth condition in \REF{hai-tri:tcbachelors} \citep[cf.][]{heimkratzer1998semantics} and \REF{hai-tri:exceptive} the truth condition in \REF{hai-tri:tcexceptive} \citep[cf.][]{vonFintel:1993}.

\ea
\ea $\{x \mid x \in \eval{\texttt{bachelor}}{} \wedge x \in \eval{\texttt{married}}{}\} \neq \emptyset$
\label{hai-tri:tcbachelors} 
\ex{$\{x \mid x \in \eval{\texttt{student}}{} \wedge x \not\in \{\text{John}\} \wedge x \in  \eval{\texttt{smoked}}{}\} \neq \emptyset \wedge \break \wedge \;
\forall P (\{x \mid x \in \eval{\texttt{student}}{} \wedge x \not\in P \wedge x \in  \eval{\texttt{smoked}}{}\} \neq \emptyset \rightarrow \{\text{John}\} \subseteq P)$}
\label{hai-tri:tcexceptive}
\z
\z
} 

\ea 
\ea[]{Some bachelor is married.}
\label{hai-tri:bachelors}
\ex[\#]{Some student but John smoked.}
\label{hai-tri:exceptive} 
\z
\z

\noindent \citet{gajewski2003analyticity} proposes that the kind of analyticity which leads to deviance is `L-analyticity'. Thus, while \REF{hai-tri:bachelors} is analytically false, \REF{hai-tri:exceptive} is L-analytically false, and therefore is deviant. Discussing Gajewski's notion of L-analyticity will take us beyond the scope of this chapter. Hence, we will leave to future research the question whether, and if yes how, sentences such as \REF{hai-tri:rep0.5quantities} can be considered L-analytical.%\vspace{-18pt}

\subsection{Countabilities} \label{hai-tri:sec:countabilities}

Words such as {\textit quantity} have been analyzed as a sort of `classifier' which turns a [$-$count] noun into a [$+$count] one \citep[cf.][]{Chierchia:2010}. This analysis is motivated by such contrasts as that in \REF{hai-tri:vampire}. 

\ea \label{hai-tri:vampire}
\ea[\#]{The vampire drank 2 bloods.}
\label{hai-tri:bloods}
\ex[]{The vampire drank 2 quantities of blood.}
\label{hai-tri:quantitiesofblood}
\z
\z

\noindent Since \textit{blood} is a [$-$count], it cannot be counted. On the other hand, \textit{quantity of blood} is [$+$count], therefore it can be. However, such contrasts as that between \REF{hai-tri:quantitiesofblood} and \REF{hai-tri:2.3}, to the best of our knowledge, have not been paid attention to.

\ea[\#]{The vampire drank 2.3 quantities of blood.}
\label{hai-tri:2.3}
\z

\noindent The semantics we proposed for {\sc many}, unfortunately, makes no distinction between \REF{hai-tri:quantitiesofblood} and \REF{hai-tri:2.3}: both are predicted to be analytically false. The proposal thus shares with several others the shortcoming of not being able to differentiate between subtypes of [$+$count] noun phrases. The task remains, therefore, of refining the semantics of {\sc many} so as to predict the contrast in question.

It should be noted, in addition, that words like \textit{quantity} may pose a challenge for the theory of measurement proposed in \citet{foxhackl2006universal}.\footnote{These include \textit{amount} and \textit{fraction}, among possibly others.

\ea
\ea[\#]{The vampire drank 2.3 amounts of blood.}
\ex[\#]{The vampire ate 2.3 fractions of the apple.}
\z
\z
} These authors derive the fact that \REF{hai-tri:morethan2} does not license the scalar implicature \REF{hai-tri:morethan3} 

\ea
\ea The vampire drank more than 2 quantities of blood.
\label{hai-tri:morethan2}
\ex $\neg$The vampire drank more than 3 quantities of blood.
\label{hai-tri:morethan3}
\z
\z

\noindent from the assumption that the scale mates of \textit{2}, for the deductive system (DS) which computes scalar implicatures, are not the set of natural numbers, but the set of rational numbers. The proposal, therefore, claims that \REF{hai-tri:2.3} is a scalar alternative of \REF{hai-tri:morethan2} (see \sectref{hai-tri:sec:pred6}). To the extent that the deviance of \REF{hai-tri:2.3} is due to this sentence being deemed ill-formed by the DS itself (see \citealt{gajewski2003analyticity, foxhackl2006universal}, and the discussion in the previous subsection), the question arises as to whether DS uses a sentence which it deems ill-formed in its computation. Again, we leave this topic to future work. %\vspace{-18pt} 

\subsection{Morphology} 

The plural vs. singular distinction in number marking languages has usually been considered to mirror the bare vs. classified distinction in classifier languages \citep[cf.][]{Chierchia:1998, cheng1999bare}. Specifically, plural/bare nouns have been analyzed as denoting `number-neutral' predicates, i.e. sets containing both singularities and pluralities, while singular/classified nouns have been analyzed as denoting `atomic' predicates, i.e. sets containing only singularities. However, with respect to numerical statements involving non-integers in English, a number marking language, and Vietnamese, a classifier language, the correlation falls apart: what is obligatory is a plural noun in English and a classified noun in Vietnamese.

\ea
\ea John ate 0.5 cake-*(s).
\ex \gll John ăn 0.5 *(cái) bánh.\\
         John ate 0.5 \textsc{cl} cake\\
    \glt `John ate 0.5 cakes.'     
\z
\z

\noindent We know of no account for this fact, and leave an investigation of it for future research. %\vspace{-18pt}

\subsection{Reals}\label{hai-tri:sec:reals}

We have been assuming that the set of numbers underlying measurement in natural language is $\mathbb{Q}$, the set of rationals. But what prevents us from assuming that it is in fact $\mathbb{R}$, the set of reals? Clearly, that assumption will be true to the extent that sentences containing reals which are not rationals are meaningful. Is \REF{hai-tri:pi} meaningful?

\ea
John ate $\pi$ (many) cakes.
\label{hai-tri:pi}
\z

\noindent We have no clear intuition about \REF{hai-tri:pi}. A confounding factor for such examples as \REF{hai-tri:pi} might be that `$\pi$' is too `artificial' to be perceived as part of natural language. One might, then, imagine an experiment along the following lines. Let $ABC$ be a cirle on which lie the three points $A$, $B$, and $C$. Let $AB$ be the diameter of $ABC$. Now suppose a mathematican, say Euclid, uttering the sentence in \REF{hai-tri:euclid}.

\ea
If $AB$ is one novel, then $ABC$ is how many novels John read.
\label{hai-tri:euclid}
\z

\noindent Obviously, there is no natural language numeral \textit{n} such that Euclid's thought can be expressed as \textit{John read n novels}. The question is whether this thought is, nevertheless, representable by grammar, or more specifically DS, and thus plays a role in inferences such as scalar implicatures (see \sectref{hai-tri:sec:countabilities}). We leave this question to future research.


\section*{Acknowledgements}
We thank Brian Buccola, Luka Crni\v{c}, Danny Fox, Manfred Krifka, the audiences at the MIT Exhaustivity Workshop, at the Semantikzirkel at ZAS Berlin, and at SinFonIJA 12 for valuable discussion. This work is supported by a research grant from the Vietnam Institute for Advanced Study in Mathematics, the ERC Advanced Grant (ERC-2017-ADG 787929) `Speech Acts in Grammar and Discourse' (SPAGAD), and grant 2093/16 of the Israel Science Foundation.

{\sloppy\printbibliography[heading=subbibliography,notkeyword=this]}

\end{document}
