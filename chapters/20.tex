\documentclass[output=paper]{langscibook} 

\author{Mojmír Dočekal\affiliation{Masaryk University} and Marcin Wągiel\affiliation{Masaryk University}}
\title{Final words} 

\ChapterDOI{10.5281/zenodo.5082488}

\abstract{} %newest class

\begin{document}
\SetupAffiliations{mark style=none}
\maketitle
 
\noindent In this monograph, we explored how the cognitive and grammatical modules of the human mind utilize number, numerals, and other categories related to the distinction between atoms and pluralities as conceptualized by humans and expressed in natural languages. All the chapters in this monograph belong in the formal part of linguistics despite them being based on different theoretical and methodological perspectives, and all of them bring new data and insights for the theories of plurality. Suppose we zoom out from the particular problems of current theories of plurality. In that case, we can schematically divide its agenda into two parts: nominal plurality and verbal plurality (sometimes called pluractionality). Both in the nominal and verbal domain, the interpretation of atomicity versus plurality is a topic of much discussion. This monograph provides new insights into both areas and linguistic territories related to the two central topics.

Even if formal linguistics uses tools of mathematics, logic and statistics, it  is still an inductive enterprise, unlike logic or mathematics. And from this, it follows that our theories of plurality are only as good as the data upon which we built them. To slightly paraphrase the words of the statistician Michael J. Crawley: \textit{All theories are wrong, but some are better than others}. In our case, after the basic building blocks of plurality theories were laid, a plethora of problems appear once we move beyond the set of English sentences or data patterns on which they were built. And we are finding ourselves exactly at this point: the torrent of new findings tells us that there is something wrong with our understanding of pluralities, and we must look for patches for and updates of our theories. The sources of new data are manifold: they come from understudied languages, cross-linguistic data patterns, experiments, big data (corpus) surveys, and many others. This monograph brings many valuable empirical patterns of this sort and shows how they bear upon the theoretical issues.

The monograph is divided into four parts in which different aspects of the theories of plurality are confronted with new empirical findings. We will now list a non-comprehensive list of big questions and then their respective sub-questions, which can be understood as some of the most important issues discussed in the individual chapters.\largerpage

In the first part, we find papers focused on the notions of number, countability, and maximality. These contributions provide both empirical and theoretical insights into the cognitive and linguistic nature of these issues. Some of the questions behind the chapters in this part are at least the following ones: can experiments answer the question of the relationship between number as a linguistic category and number as a cognitive notion? How do verbal and nominal pluralities relate? What are the real linguistic markers of maximality and number? Which syntactic mechanisms express distributivity or plurality? 

The second part of the monograph explores the core topics of theories of plurality -- the possible interpretations of sentences containing plurality-denoting expressions, higher-order atoms, and many others. The questions tackled in this part are at least the following ones: how much must we revise our theories of plurality if we take seriously cross-linguistic patterns of (sine qua unexpected) cumulative readings? What can we learn about cumulative readings if we also take into account opaque contexts? What is the nature of collective interpretation once we move beyond such nouns as \textit{team} and \textit{swarm}?

The third part gathers contributions addressing the proper treatment of numerals, their modifiers, and classifiers (bridges between numerals and nouns). And we can interpret the chapters in this part as inspired by questions like these: Was Frege right in treating numerals as equinumerousity of concepts? How much must we update our theories in order to properly describe non-integers? How compositional are numerals in natural languages? What can we learn from the interaction of focus particles with superlative modifiers? Do our theories of the mass/count distinction still work once we take into account optional classifiers in languages such as Hungarian?

The last part of the monograph focuses on quantifiers other than numerals, indefinites, and some interactions of degree expressions with polarity licensing. The research behind the chapters in this part was driven by questions like these: to what extent do quantifier semantics predetermine the verification procedure of a human agent? And what can an eye tracker experiment with Polish speakers tell us about the issue? Why do some indefinites not yield implicatures (as predicted by our current theories)? And how much can we learn about that from Hungarian child acquisition data? 

The previous four paragraphs aim at offering the interested reader insight into the nature of issues discussed in the monograph. The chapters build upon different frameworks: linguistic typology, theories of plurality confined within Frege's boundary, and psycholinguistics, where a lot of attention is paid to ideas in mind, something which Frege termed ``Vorstellung'' and put on the back burner of logic and mathematics. As stated in the introductory chapter, we are far away from a unification of these frameworks, if that is even possible. We can find a slightly parallel debate in the 20th-century philosophy of mathematics, the one between Frege, Russell, and Hilbert's formalism/logicism and Brouwer's intuitionism, which seems still unsettled today. But no matter how the biggest questions are answered, this monograph brings together people looking for possible solutions along with the frameworks they work in to scrutinize the existing theories and then confront them with the new data gathered experimentally, via big corpus searches, traditional intuition reflections, or cross-linguistic data surveys. A lot of attention was paid to Slavic: eleven chapters are based on data from various Slavic languages like Bosnian/Croatian/Montenegrin/Serbian, Czech, Macedonian, Polish, Russian, Slovak, and Slovenian. And it was one of the goals of this book to bring the Slavic data to the debates in theories of plurality. Among other languages which you can find discussed in the monograph are English, German, Greek, Hungarian, Japanese, Mandarin Chinese, and Wolof. The empirical landscape is colorful, as are the approaches which describe it. The result is the monograph in which you, an avid reader, just read the last chapter. All in all, it was a fascinating journey.



{\sloppy\printbibliography[heading=subbibliography,notkeyword=this]}
\end{document}
