\documentclass[output=paper]{langscibook} 
\ChapterDOI{10.5281/zenodo.5082452}

\author{Piotr Gulgowski\affiliation{University of Wrocław} and  Joanna Błaszczak\affiliation{University of Wrocław}}
\title[Conceptual representation of lexical and grammatical number]{Conceptual representation of lexical and grammatical number: Evidence from SNARC and size congruity effect in the processing of Polish nouns}  
\abstract{The goal of the present study was to investigate the numerical representation of the referents of collective singular nouns in comparison with non-collective singular and plural nouns. Specifically, we asked whether the representation of collective singulars is influenced by the grammatical number (singularity) or the lexical specification (plurality of collection elements). This question was addressed in two psycholinguistic experiments using a technique based on two number-related phenomena: the spatial-numerical association of response codes (SNARC) effect and the size congruity effect. Participants performed semantic (Experiment 1) or grammatical (Experiment 2) number judgments for collective and non-collective Polish nouns, while the response hand, grammatical number and font size of the words were manipulated. A weak SNARC effect was found in the form of faster responses for grammatically singular nouns with the left hand and for grammatically plural nouns with the right hand. Collective singulars patterned with non-collective singulars suggesting that the primary representation of collective referents does not include conceptual plurality. The numerical interpretation seems to be driven more by grammatical than lexical factors. The SNARC effect was present only in Experiment 1, which points to its dependence on the task type. No size congruity effect occurred in either experiment, so the size of the denoted set does not appear to be a salient property of the conceptual representation of linguistic number.

\keywords{collectivity, number, plurality, size congruity effect, SNARC}}

\begin{document}
\SetupAffiliations{mark style=none}
\maketitle

\section{Introduction} 
In many languages number has the status of a grammatical category as illustrated by contrasts like \textit{dog} vs. \textit{dogs} in English. These contrasts are linked with certain conceptual distinctions, specifically with communicating whether the speaker has in mind one thing or multiple things. Linking number form with number meaning is not always a straightforward task. Collective nouns are a class of words characterized by an inherent plurality. A grammatically singular collective noun, like the English word \textit{committee}, is lexically specified as a collection with multiple elements. Proper comprehension of a singular collective noun requires the ability to reconcile those two sources of numerical information and to construct the correct interpretation. The goal of the present study is to shed more light on how language comprehenders represent the denotation of collective singular nouns (e.g., \textit{army}) and how those representations compare to non-collective singular nouns (e.g., \textit{soldier}) and plural nouns (e.g., \textit{soldiers}). We were particularly interested in whether the numerical construal of a collective referent is primarily affected by the lexical or the grammatical factors. Past research \citep{bockMeaningSoundSyntax1993,bockNumberAgreementBritish2006,nenonenMismatchesGrammaticalNumber2010} revealed that the plural reading of collective nouns is less common than the singular reading, which might suggest that the reading of such words is determined mostly by their grammatical number. However, the methods used in past studies may not have been able to capture the way in which the participants actually construed the objects denoted by collective nouns (as discussed below). To investigate this issue we used a technique based on two phenomena known to be related to general numerical cognition: the spatial-numerical association of response codes (SNARC) effect and the size congruity effect. Both effects belong to the class of interference phenomena in which two dimensions (e.g., conceptual number and size) collide resulting in a conflict detectable in reaction times. Employing these effects as diagnostics of conceptual singularity and plurality allowed us to investigate the numerical representations built automatically by language users as they encounter singular, plural and collective nouns. 

\section{Past research}
The semantics of grammatical number has long been an important topic of formal linguistic analyses. Notable work has been done within the framework which applied mereological tools to extend the ontological domain of language in order to include plural objects and groups as well as singular atoms \citep{linkGeneralizedQuantifiersPlurals1987, landmanGroups1989}.\footnote{For a more recent discussion of the semantics of number, see \citet{moltmannPluralReferenceReference2016}.} Since grammatically singular nouns naming a collection (e.g., \textit{army}) can refer to the collection as a whole (a collective or singular reference) or to its elements (a distributive or plural reference), a proper description of their semantics has been challenging. Consequently, collectivity has been the subject of multiple theoretical accounts \citep[for an overview, see][Section 1.2]{levinAgreementCollectiveNouns2001}. The problem of singular nouns denoting multiple entities also attracted the attention of experimental researchers. Some of the empirical findings are discussed below.

\citet{bockMeaningSoundSyntax1993} showed participants a list of English nouns (collective and non-collective) that were either singular or plural. The participants were asked to indicate how many things each word represented. The results revealed that collective singulars were significantly more likely to be associated with the “more than one thing” answer (41\% of responses) than non-collective singulars (10\% of responses). In contrast, this answer constituted around 90\% of responses for grammatically plural nouns. \citet{nenonenMismatchesGrammaticalNumber2010} conducted a similar judgment test for several classes of Finnish nouns, including derivationally created collectives. The results showed again that participants allowed plural referents for grammatically singular collective nouns, though less commonly than in Bock \& Eberhard’s English study: the “more than one thing” answers constituted around 20\% of responses in this condition. Overall, a plural interpretation of collective singulars was available, although it was clearly not the dominant one. Additionally, the authors reported a considerable variability for individual collective nouns, which ranged from 0\% to around 40\% of the “more than one thing” responses, suggesting that not all nouns commonly treated as collective by linguists may in fact have this status for the majority of speakers.

In some varieties of English, grammatically singular collective subjects can appear with both singular and plural agreement morphology on the verb. This is known as conceptual (or notional) agreement.

\ea The committee has/have finally made a decision.
\z

\noindent An investigation of the agreement patterns for collectives in two major varieties of English can be found in \citet{bockNumberAgreementBritish2006}. In a sentence completion study, participants (British English and American English speakers) were instructed to turn simple definite noun phrases containing different types of nouns into full sentences. Collective singular nouns were followed by plural verbs in around 20\% of continuations for BE speakers and in around 2.3\% of continuations for AE speakers. This was in contrast to the near lack of plural agreement continuations following ordinary singular nouns and nearly 100\% of plural agreement continuations following plural nouns for both language varieties. A similar pattern was found in a corpus survey of American and British financial press also presented in \citet{bockNumberAgreementBritish2006}. In the studied sample, collective singular nouns were followed by plural verbs in around 26\% of cases in the British corpus and in around 7\% of cases in the American corpus. The study confirmed that plural verb agreement for collective singular subjects is available as an option for the speakers of contemporary British English, although it is chosen less frequently than singular agreement. 

That singular nouns can denote multiple objects has also been demonstrated with words known as object-mass nouns (e.g., \textit{furniture}, \textit{jewelry}, \textit{clothing}), which have been argued to individuate their meaning despite being morphosyntactically uncountable \citep{aBarnerQuantityjudgmentsindividuation2005}. Object-mass nouns resemble collective nouns, the main difference being that the former disallow plural forms (e.g., *\textit{furnitures}) whereas the latter can be pluralized (e.g., \textit{armies}).

A phenomenon similar to lexical collectivity also exists at the level of predicates. Sentences with plural subjects, like in the example below, can be ambiguous. 

\ea Three students lifted a piano.
\z

\noindent The sentence can be understood as referring to a situation where all three students lifted the piano together (collective reading) or to separate events of piano lifting (distributive reading). In an eye-tracking experiment, \citet{aFrazierTakingsemanticcommitments1999} presented participants with sentences containing conjoined subjects that were ambiguous between a collective and a distributive reading (e.g., \textit{Jane and Martha weighed 220 pounds\dots}). The sentences contained also a disambiguating adverb located in different places depending on the condition. If the disambiguating adverb appeared after the predicate, participants needed more effort (longer fixation times, more regressions) to process the disambiguation when the adverb was distributive (\textit{each}) than when it was collective (\textit{together}). This indicates that a collective reading of a sentence might be the preferred interpretation. An ambiguous predicate is by default assumed to be collective and the comprehender needs some time to recover if this initial assumption turns out to be wrong. 

The studies discussed above extended our understanding of collectivity by providing more information about the likelihood of the singular (collective) and plural (distributive) reading of such words. The results indicate that the dominant interpretation associated with a collective noun is singular. It is not clear, however, whether the observed effects reflect the way in which the referents of collectives are truly conceptualized when they are encountered. The number judgment studies by \citet{bockMeaningSoundSyntax1993} and \citet{nenonenMismatchesGrammaticalNumber2010} or the sentence completion study by \citet{bockNumberAgreementBritish2006} did not control for the possibility that participants used (at least partially) the response strategy of deliberately following the grammatical number marking on the noun, so the preponderance of singular responses in those studies may not correspond to the basic representation of collective referents. The eye-tracking experiment of \citet{aFrazierTakingsemanticcommitments1999} suggests a general tendency to represent collections primarily as wholes instead of focusing on the individual elements. However, the materials used in that experiment contained conjoined noun phrases instead of collective nouns. Additionally, a preference at the sentence level might not generalize to the level of words. 

Three possibilities exist. The first possibility is that the singular construal (the collection as a whole) is indeed the primary representation of the referents of collective nouns, as suggested by the results of past research. The plural reading under this scenario must be derived from this default singular interpretation by some process, perhaps by highlighting constituent parts through a kind of profiling mechanism described by \citet{lagnackerConceptImageSymbol1991}. The second possibility is that conceptual plurality following from lexical semantics is primary for collectives. In this case, the predominant singular judgments and agreement patterns reported in the past studies could result from a deliberate response strategy and should be absent in measures of more automatic processes. One more possibility is that both construals of a collective word (conceptual singularity and plurality) are activated simultaneously leading to a competition. 

Distinguishing between those three possibilities requires applying a tool sensitive to number-related concepts and capable of capturing early mental construals. For this reason, the method chosen for the present study depended on measuring reaction times, which may reveal aspects of the numerical representations not reflected in the elicited judgments. The method was based on two interference phenomena well documented in the literature on numerical cognition. The following section introduces both phenomena and discusses their suitability for studying grammatical number in general and collectivity in particular.

\section{Number interference effects}
Numerical cognition is the name for the psychological mechanisms responsible for processing numbers and quantities. It has been established that humans share with many other animal species the ability to quickly determine the exact number of elements in a set of up to four things and to estimate the approximate numerosity of larger sets \citep{feigensonCoreSystemsNumber2004}. Another finding has been that processing a numerical quantity (expressed, for instance, by a digit or a number word) can be disrupted by processing other types of information, like spatial relations or size \citep{dehaeneMentalRepresentationParity1993,henikThreeGreaterFive1982,fitousiRoleParityPhysical2009,cohenkadoshEffectOrientationNumber2007}. Such interference can be used to find out whether a specific stimulus activates a numerical concept in the mind of an experiment participant.

\subsection{Number and space: The SNARC effect}

In a series of experiments designed to test the representation and extraction of number-related information (parity and numerical magnitude) associated with number symbols, \citet{dehaeneMentalRepresentationParity1993} asked participants to determine whether numbers (single digits in the range 0--9) appearing individually on the screen are odd or even by pressing a button with the left hand or the right hand. The assignment of the correct response to the hand was manipulated. There was a significant interaction between the magnitude of the displayed numbers and the response hand, with faster responses to small numbers using the left hand and to big numbers using the right hand. The effect was sensitive to relative, rather than absolute, numerical values (numbers 4 and 5 received faster responses with the right hand when they were tested in the range 0--5 and with the left hand in the range 4--9) as well as to reading and writing habits (it was much weaker or even reversed for Iranian subjects more familiar with a right-to-left writing system). The phenomenon has been labeled the \textsc{spatial-numerical association of response codes} (SNARC) effect.

The SNARC effect has been found in auditory as well as visual modality, for Arabic digits and for number words \citep{nuerkUniversalSNARCEffect2005}. The existence of the SNARC effect has been used as an argument in favor of the mental number line hypothesis, i.e., the idea that magnitudes associated with numbers are represented mentally as if on an imaginary line, typically with small numbers on the left and large numbers on the right \citep{dehaeneMentalRepresentationParity1993,gobelCulturalNumberLine2011,paveseSymbolicDistanceNumerosity1998}. The effect has also been found for tasks involving determining the size \citep{fitousiRoleParityPhysical2009} or color \citep{keusSearchingFunctionalLocus2005} of number symbols. Performing those tasks does not require accessing the number value of the symbols, so numerical information seems to be activated automatically even if participants do not pay attention to it. However, the kind of task does matter. \citet{rottgerGrammaticalNumberElicits2015} carefully tested the influence of the task demands on the SNARC effect. They gave participants four kinds of tasks using written German numerals as stimuli. No SNARC effect was found for the tasks focusing on visual features (type of font) or lexical features (real word or pseudoword), however the effect was present for two semantic tasks (parity and magnitude determination).\footnote{Number words seem more sensitive to the type of task than digits, as demonstrated in phoneme monitoring experiments \citep{fiasTwoRoutesProcessing2001,fiasImportanceMagnitudeInformation1996}.}   

Although the numerical concepts associated with grammatical number (singularity vs. plurality) are less precise than the values encoded by numerals,   they too can give rise to the SNARC effect, as demonstrated by \citet{rottgerGrammaticalNumberElicits2015}. Singular and plural German nouns were used as stimuli in an experiment resembling closely the experiment with numerals described above. The task once again probed four levels of processing: visual features (font type), lexical features (real word or pseudoword), non-numerical semantics (animacy) and numerical semantics (singular or plural meaning). The analysis of response times indicated that participants exhibited a left hand facilitation for singular nouns and a right hand facilitation for plural nouns. This pattern resembled the classic SNARC effect for small and large numbers and was consistent with the possibility that singular nouns (denoting a small amount) are linked with the left end of the mental number line, while plural nouns (activating the concept of a large quantity) are linked with the right end. The effect was statistically significant only for the task requiring direct access to number semantics (i.e., deciding whether a given noun names one or more than one entity). 

\subsection{Number and size: The size congruity effect}
A different mental mechanism in the form of \textsc{size congruity effect} (SCE) connects numerical cognition with the processing of size. The non-numerical variant of the effect was originally demonstrated by \citet{paivioPerceptualComparisonsMind1975}. Participants in that study were shown pairs of pictures of animals and objects. The pictures differed in sizes. In the incongruent condition, the entity smaller in real life was represented as visually larger (e.g., a lamp bigger than a zebra). In the congruent condition, the depicted objects were of the expected proportions. Participants were asked to indicate which object is larger in real life while ignoring the sizes of the pictures. The responses were faster when the picture sizes matched the real life sizes. A numerical version of the effect was described by \citet{henikThreeGreaterFive1982}. Pairs of Arabic digits of varying font sizes were used in a magnitude comparison experiment. The numerical and visual magnitudes were either congruent (e.g., {\small 3} vs. {\Large \textbf{5}}) or incongruent (e.g., {\Large \textbf{3}} vs. {\small 5}). The average response times in the congruent condition were faster than in the incongruent condition. This interference effect has been replicated in subsequent studies both with digits and number words \citep{besnerIdeographicAlphabeticProcessing1979,cohenkadoshEffectOrientationNumber2007,foltzMentalComparisonSize1984}.\footnotemark{} To our knowledge, a size congruity effect for grammatical number (or lexical collectivity) has not yet been demonstrated. However, given that interpreting number in language gives rise to a mental representation of quantity \citep{patsonSingularInterpretationsLinger2016,patsonConceptualRepresentationNumber2014}, it should also activate set size information.
\footnotetext{SCE can be used as an argument for the existence of a general magnitude processing mechanism where a common, modality-independent representation is assigned to all kinds of quantity. However, critics of this hypothesis \citep{aVanOpstaltheregeneralizedmagnitude2013} point out that the observed interaction between number and physical magnitude may take place at a relatively late decision-making stage where the outputs of completely or partially distinct systems compete for response selection (e.g., \textit{small} from number magnitude interpretation competing with \textit{big} from visual size analysis). Similarities in the processing of discrete (number) and continuous (size) quantities may result from similar task demands or the limitations of the basic cognitive systems, like working memory. See also the discussion in \citet{aSantenssizecongruityeffect2011}.}

\subsection{Combining SNARC with SCE}
An experimental design combining the two phenomena has been presented in \citet{fitousiRoleParityPhysical2009}. In order to find out whether the SNARC effect and the SCE would interact, participants were asked to determine the font size of numbers displayed on the screen (Arabic digits 1--9 except 5) by responding with the right or left hand for large font or small font digits (the assignment of correct responses to the left or right hand varied between blocks). The number value and size of stimuli were thus independently manipulated. Participants were asked to ignore the numerical value of the digit. There was a clear size congruity effect and a significant SNARC effect. The authors found no statistical evidence in the data for any interaction between the two effects, but the study showed that the two effects can be elicited simultaneously in a single experiment. The same was also attempted in the present work. We decided to combine both effects in order to create a more sensitive tool for detecting the activation of numerical concepts and, consequently, to provide a more comprehensive picture of how the referents of collective nouns are numerically represented and of the role of grammatical and lexical factors. Additionally, by using a SNARC-SCE technique we hoped to determine whether the numerosity representations constructed from nouns resemble the representations evoked by numerals and digits in terms of relations with both size and space.

\section{Experiment 1}

\subsection{Research questions and predictions}
The goal of Experiment 1 was to investigate whether the numerical representations associated with collective singular nouns depend more on the grammatical singularity or lexical plurality of those words. This was done by comparing collective singulars with non-collective singular and plural nouns. The number concepts activated by each noun type were measured by the capacity of the words to produce the SNARC effect and the size congruity effect. The design consisted of a semantic number judgment task (determining how many things a word denotes) combined with manipulating the response hand, grammatical number and font size of collective and non-collective (henceforth \textsc{unitary}) Polish nouns.

The predictions for unitary singulars and plurals were straightforward, based on the results from previous studies of the SNARC effect \citep[e.g.,][]{rottgerGrammaticalNumberElicits2015} and the size congruity effect \citep[e.g.,][]{henikThreeGreaterFive1982}. Unitary singular nouns were predicted to activate the concept of `one', congruent with the left side (SNARC) and with small font (SCE). Plural nouns were predicted to evoke the notion of `more than one', congruent with the right side and with big font. The congruent conditions were expected to result in a facilitation in the form of faster responses.

The results for collective singulars were of particular interest. If the primary representation of their meaning is determined by the lexical information about the multiplicity of constituent elements, they should pattern with grammatically plural nouns. If the referent of collectives is conceptualized primarily as singular, in accordance with their grammatical number designation, then they should behave like unitary singular nouns. If both construals (conceptual singularity and plurality) are initially activated resulting in a conflict and competition, collective singular nouns could fall somewhere between unitary singular and plural nouns in terms of their capacity to elicit the SNARC effect and the SCE.

\subsection{Design}

\subsubsection{Materials}
Thirty unitary singular nouns (e.g., \textit{wilk} ‘wolf’) were selected for the experiment. Thirty plural forms were created from the singulars (e.g., \textit{wilki} ‘wolves’). 

Additionally, 20 collective singular nouns (e.g., \textit{ławica} ‘shoal’) were chosen. Although collective singular nouns in Polish do not allow for a plural subject-verb agreement, the collective status of Polish nouns can be demonstrated by their compatibility with predicates which normally require plural subjects (e.g., \textit{zebrać się} ‘to gather’). This was used as a criterion for the selection of collective nouns for the experiment from a candidate set prepared based on the authors' intuition.

Plural equivalents of collective singulars were not created by simply pluralizing them. Instead, a plural form of a closely semantically related unitary noun was selected for each collective singular (e.g., plural \textit{śledzie} ‘herrings’ for collective singular \textit{ławica} ‘shoal’). This was done for two reasons. First, many Polish collective nouns show case syncretism across grammatical number (e.g., \mbox{\textit{grup-y}} ‘group-\textsc{nom.pl}’ or ‘group-\textsc{gen.sg}’). Such number ambiguity is easily disambiguat\-ed with context, but, in the present experiment, words were shown in isolation and the results hinged on a fast recognition and activation of number values. None of the plural forms used in the study was ambiguous in this way. The second reason was to avoid the possible difficulties with processing “doubly plural” forms like \textit{teams}. 

Overall, there were 100 nouns (60 unitary and 40 collective), 50 singular and 50 plural, each occurring in a big font and a small font condition as well as in a left response hand and a right response hand condition. This design resulted in 400 trials presented in two blocks. Every participant saw every item. The presentation order was fully randomized across blocks for every participant.

\subsubsection{Procedure}
The experiment was conducted on a standard PC computer using a 23.6 inch monitor (LG 24M35D-B) with a 1920×1080 resolution. With the distance of a participant from the screen of approximately 60cm, a single character in the small font condition (50 pixels) subtended $\sim0.45°$ (horizontally) by $\sim0.75°$ (vertically) of visual angle, while a single character in the big font condition (150 pixels) subtended $\sim1.62°$ (horizontally) by $\sim2.39°$ (vertically) of visual angle.

The experimental procedure was based on the techniques presented in \citet{rottgerGrammaticalNumberElicits2015} and  \citet{fitousiRoleParityPhysical2009}, who used a pure SNARC effect and a combination of the SNARC effect with the SCE, respectively. At the beginning of each trial, five asterisks appeared at the center of the screen. The symbols were automatically replaced after 300ms by an experimental stimulus. The stimulus was a singular or plural Polish noun displayed either in small font or big font. The participant’s task was to determine whether the noun referred to one or more than one thing (semantic number judgment) while ignoring the visual size of the stimulus. The stimulus remained on the screen until the participant made a decision by pressing the “z” or “/” key on a standard QWERTY keyboard corresponding to the answers “one” or “more than one”. There was a 300ms blank screen between trials.

The experiment consisted of two blocks. The assignment of keys to responses changed after the first block (e.g., if “z” in Block 1 meant “more than one”, in Block 2 it meant “one”). A message before each block informed the participant about the current assignment of keys. The order of key assignments in blocks was counterbalanced across participants. There were three breaks within each block. During a break the participant was encouraged to rest and resume the experiment by pressing a button. In each block, the experiment proper was preceded by a training session with 24 trials. The set of training items consisted of nouns balanced in terms of grammatical number, font size and response hand. None of the items used in the training session appeared later in the experiment proper. Feedback was provided if the participant made a mistake in the form of a message (\textit{źle} ‘incorrect‘) that stayed on the screen for 1 second. In the training session a message appeared also after correct responses (\textit{dobrze} ‘correct‘). During the experiment proper, feedback was provided only for incorrect responses. The main purpose of the feedback was to facilitate learning the correct assignment of keys. 

The experiment was designed and presented using the PsychoPy software (version 1.84.2) \citep{peircePsychoPyPsychophysicsSoftware2007, peirceGeneratingStimuliNeuroscience2009}.

\subsubsection{Participants}
Twenty-two students of the Institute for English Studies of the University of Wrocław (9 women, 13 men) took part in the experiment. Participants were all native speakers of Polish. The average age was 20.8 ($\text{SD}=2.5$).

\subsection{Results: Number judgments}\largerpage
To determine the general availability of a plural reading of collective nouns in Polish, the first analysis looked at the judgments the participants made regarding the semantic number of the nouns (determining whether a word named one or more than one thing).

The percentage of ``more than one thing'' responses for collective singulars ($\text{M}=20.7\%, \text{SD}=31.2$) was considerably lower than for plurals ($\text{M}=97.4\%, \text{SD}=2.7$), but it was higher than for unitary singulars ($\text{M}=2.4\%, \text{SD}=2.9$). The participants regarded grammatically plural nouns as almost always referring to multiple entities. Unitary singular nouns were almost always interpreted as denoting a single thing.  The answers for collective singulars were less consistent. Nouns in this condition were predominantly interpreted as referring to one thing, but around a fifth of responses indicated a plural reading. A pair of one-way ANOVA tests (by subjects and by items) with the percentage of plural responses as the dependent variable and the type of number (collective singular, unitary singular, plural) as the independent factor confirmed that the difference was statistically significant ($\text{F}_{1}(2,42)=172.990, p<0.001, \eta^2=0.892; \text{F}_{2}(2,97)=12209.997$, $p<~0.001, \eta^2=0.996$).

The variance among collective singulars was larger than for the other conditions. The most plural-like collectives (\textit{armia} ‘army’, \textit{brygada} ‘brigade’) received the ``more than one thing'' answer in 26\% of cases, while for the most singular-like collective (\textit{zbiór} ‘set’, the only collective noun used in the experiment that was not clearly animate) the singular answer was given in 13\% of cases.\footnote{A high variance for collectives has been reported before by \citet{nenonenMismatchesGrammaticalNumber2010}.}

Some variance existed also among the participants. Four participants never chose the ``more than one thing'' answer in the collective condition, meaning that they treated collective nouns as exclusively singular. On the other end of the scale, two participants chose the ``more than one thing'' response for 92\% of collectives, meaning that nouns from this group were predominantly plural for them. For the majority of the participants, the ``more than one thing'' answers in this condition did not exceed 35\% of responses. See \tabref{gul-bla:tab:reaction-times-exp1} for percentages in individual conditions.

\subsection{Results: Reaction time}
The data were cleaned first by removing all incorrect responses (with the exception of answers to collective singulars) %\footnotemark{} 
and then eliminating all trials with reaction times (RT) 3 standard deviations above and below the mean for every participant.\footnotemark{} This resulted in eliminating 184 data points, which constituted 2.1\% of correct responses. The remaining trials were subjected to tests performed with the SPSS software (version 22).
\footnotetext{Because no response could be considered objectively wrong for collective singulars, all answers in this condition were included in the final analysis.}

A pair of 3×2×2 ANOVA tests (by subjects and by items) were conducted with RT as the dependent variable and the following independent factors and all their interactions:\largerpage

\begin{itemize}
\item Number Type (collective singular, unitary singular, plural)
\item Font Size (small, big) 
\item Response Hand (left, right)
\end{itemize}

Results of the ANOVA tests are given in \tabref{gul-bla:tab:ANOVA-exp1}. Mean reaction times and accuracy in each condition are given in \tabref{gul-bla:tab:reaction-times-exp1}.

% \input {./tables/table-1.tex}

\begin{table}[H]
\sisetup{table-space-text-post = *}
\caption{ANOVA test results for Experiment 1. NT: Number Type; FS: Font Size; RH: Response Hand.}
\label{gul-bla:tab:ANOVA-exp1}
\begin{tabular}{l cc *{2}{S[table-format=1.2]} *{2}{S[table-format=<1.3,table-align-text-post=false]} *{2}{S[table-format=1.2]}}
\lsptoprule
Source  & \multicolumn{2}{c}{df} & \multicolumn{2}{c}{F} & \multicolumn{2}{c}{$p$}  & \multicolumn{2}{c}{Partial $\eta^2$} \\
\cmidrule(lr){2-3}\cmidrule(lr){4-5}\cmidrule(lr){6-7}\cmidrule(lr){8-9}
                 & F1      & F2    & {F1}     & {F2}    & {F1}       & {F2}     & {F1}      & {F2}    \\
\midrule
NT     &  2, 42   &  2, 97&  18.67 &  35.35&  <0.001*  &  <0.001*&  0.47   &  0.42  \\
FS       &  1, 21   &  1, 97&  0.26  &  0.05 &  0.615   &  0.942  &  0.01   &  0.00  \\
RH   &  1, 21   &  1, 97&  0.54  &  1.17 &  0.471   &  0.283  &  0.03   &  0.01  \\
NT×FS       &  2, 42   &  2, 97&  0.66  &  0.19 &  0.520   &  0.828  &  0.03   &  0.00  \\
NT×RH   &  2, 42   &  2, 97&  1.25  &  6.06 &  0.296   &  0.003* &  0.06   &  0.11  \\
FS×RH &  1, 21   &  1, 97&  0.45  &  0.14 &  0.508   &  0.712  &  0.02   &  0.00  \\
NT×FS×RH &  2, 42   &  2, 9 &  0.22  &  0.11 &  0.802   &  0.893  &  0.01   &  0.00  \\
\lspbottomrule
\end{tabular}
\end{table}

% \input {./tables/table-2.tex}

% \begin{table}[h!]
% \caption{Mean reaction times (ms) and number judgment answers (percent of plural responses) in all conditions in Experiment 1. Standard errors in parentheses}
% \label{gul-bla:tab:reaction-times-exp1}
% \begin{tabular}{cccccccc}
% \lsptoprule
% \multirow{4}{*}{\begin{tabular}[c]{@{}c@{}}\\ Num\\ Type\end{tabular}} & \multirow{4}{*}{\begin{tabular}[c]{@{}c@{}}\\ Font\\ Size\end{tabular}} & \multicolumn{6}{c}{Response Hand}                                                                                                                                                                                                                                                                                                             \\
%                                                                     &                                                                      & \multicolumn{2}{c}{Left}                                                                                      & \multicolumn{2}{c}{Right}                                                                                     & \multicolumn{2}{c}{\begin{tabular}[c]{@{}c@{}}Congruity\\ (Left-Right)\end{tabular}}                          \\
%                                                                     &                                                                      & \begin{tabular}[c]{@{}c@{}}RT\\ (ms)\end{tabular} & \begin{tabular}[c]{@{}c@{}}ANSW\\ (\% of pl)\end{tabular} & \begin{tabular}[c]{@{}c@{}}RT\\ (ms)\end{tabular} & \begin{tabular}[c]{@{}c@{}}ANSW\\ (\% of pl)\end{tabular} & \begin{tabular}[c]{@{}c@{}}RT\\ (ms)\end{tabular} & \begin{tabular}[c]{@{}c@{}}ANSW\\ (\% of pl)\end{tabular} \\
% \midrule
% \multirow{2}{*}{Col Sg}                                            & Small                                                                & $854 (47)$                                          & $19.3\% (7.0)$                                              & $910 (54)$                                          & $23.0\% (6.7)$                                              & $-56$                                               & $-3.7\%$                                                    \\
%                                                                     & Big                                                                  & $853 (45)$                                          & $19.5\% (6.7)$                                              & $902 (59)$                                          & $21.1\% (6.9)$                                              & $-49$                                               & $-1.6\%$                                                    \\
% \tablevspace
% \multirow{2}{*}{Unit Sg}                                           & Small                                                                & $772 (41)$                                          & $02.0\% (0.8)$                                              & $784 (42)$                                          & $02.1\% (0.8)$                                              & $-12$                                               & $-0.1\%$                                                    \\
%                                                                     & Big                                                                  & $776 (38)$                                          & $02.6\% (1.0)$                                              & $789 (43)$                                          & $03.0\% (0.7)$                                              & $-13$                                               & $-0.4\%$                                                    \\
% \tablevspace
% \multirow{2}{*}{Plural}                                             & Small                                                                & $821 (47)$                                          & $97.5\% (0.6)$                                              & $802 (35)$                                          & $97.0\% (0.7)$                                              & $19$                                                & $0.5\%$                                                     \\
%                                                                     & Big                                                                  & $818 (46)$                                          & $97.5\% (0.7)$                                              & $779 (32)$                                          & $97.6\% (0.6)$                                              & $39$                                                & $-0.1\%$ \\ 
% \lspbottomrule
% \end{tabular}
% \end{table}


\begin{table}[H]
\caption{Mean reaction times (ms) and number judgment answers (percent of plural responses) in all conditions in Experiment 1. Standard errors in parentheses}
\label{gul-bla:tab:reaction-times-exp1}
\begin{tabularx}{\textwidth}{X r@{~~~}r r@{~~~}r r@{~~~}r}
\lsptoprule
&\multicolumn{6}{c}{Response Hand}\\\cmidrule(lr){2-7}
&\multicolumn{2}{c}{\multirow{2}{*}{Left}}&\multicolumn{2}{c}{\multirow{2}{*}{Right}}&\multicolumn{2}{c}{Congruity}\\
&&&&&\multicolumn{2}{c}{($\text{Left}-\text{Right}$)}\\\cmidrule(lr){2-3}\cmidrule(lr){4-5}\cmidrule(lr){6-7}
\textit{Num Type}&\multicolumn{1}{c}{RT}&\multicolumn{1}{c}{Answ}&\multicolumn{1}{c}{RT}&\multicolumn{1}{c}{Answ}&\multicolumn{1}{c}{RT}&\multicolumn{1}{c}{Answ}\\
\hspace{6pt}Font Size&\multicolumn{1}{c}{(ms)}&\multicolumn{1}{c}{(\% of pl)}&\multicolumn{1}{c}{(ms)}&\multicolumn{1}{c}{(\% of pl)}&\multicolumn{1}{c}{(ms)}&\multicolumn{1}{c}{(\% of pl)}\\\midrule
\textit{Col Sg}\\
\hspace{6pt}Small& $854 (47)$& $19.3\% (7.0)$& $910 (54)$& $23.0\% (6.7)$& $-56$& $-3.7\%$\\
\hspace{6pt}Big& $853 (45)$& $19.5\% (6.7)$& $902 (59)$& $21.1\% (6.9)$& $-49$& $-1.6\%$\\\tablevspace
\textit{Unit Sg}\\
\hspace{6pt}Small& $772 (41)$& $2.0\% (0.8)$& $784 (42)$& $2.1\% (0.8)$& $-12$& $-0.1\%$\\
\hspace{6pt}Big& $776 (38)$& $2.6\% (1.0)$& $789 (43)$& $3.0\% (0.7)$& $-13$& $-0.4\%$\\\tablevspace
\textit{Plural}\\
\hspace{6pt}Small& $821 (47)$& $97.5\% (0.6)$& $802 (35)$& $97.0\% (0.7)$& $19$& $0.5\%$\\
\hspace{6pt}Big& $818 (46)$& $97.5\% (0.7)$& $779 (32)$& $97.6\% (0.6)$& $39$& $-0.1\%$ \\
\lspbottomrule
\end{tabularx}
\end{table}



\subsubsection{Number Type effect}
The main effect of Number Type was significant, see \tabref{gul-bla:tab:ANOVA-exp1}. Responses to collective singular nouns were on average longest ($\text{M}=880\text{ms}, \text{SE}=45$), followed by responses to plural nouns ($\text{M}=805\text{ms}, \text{SE}=38$) and unitary singular nouns ($\text{M}=780\text{ms}, \text{SE}=38$). However, this significant main effect has to be considered in the context of a significant (by items) interaction between Number Type and Response Hand. No other main effect was significant.

\subsubsection{SNARC effect}
The interaction of Number Type×Response Hand was not significant by subjects but it was significant by items, see \tabref{gul-bla:tab:ANOVA-exp1}. For unitary singulars and plurals the interaction was consistent with the predicted SNARC effect. Responses for unitary singular nouns were faster with the left hand than with the right hand. The opposite was true for plural nouns. Collective singulars patterned with unitary singular nouns. The left hand preference for collectives was numerically even bigger than for unitary nouns. See \tabref{gul-bla:tab:SNARC-exp1} for reaction times and number judgments.

% \input {./tables/table-3}

% \begin{table}[h!]
% \caption{Congruity of response hand and number type (SNARC) in Experiment 1 measured in reaction times (ms) and number judgment answers (percent of plural responses). Standard errors in parentheses}
% \label{gul-bla:tab:SNARC-exp1}
% \begin{tabular}{ccccccc}
% \lsptoprule
%                  & \multicolumn{6}{c}{Response Hand}                                \\
% \begin{tabular}[c]{@{}c@{}}Num\\ Type\end{tabular} &
%   \multicolumn{2}{c}{Left} &
%   \multicolumn{2}{c}{Right} &
%   \multicolumn{2}{c}{\begin{tabular}[c]{@{}c@{}}Congruity\\ (Left-Right)\end{tabular}} \\
%  &
%   \begin{tabular}[c]{@{}c@{}}RT\\ (ms)\end{tabular} &
%   \begin{tabular}[c]{@{}c@{}}ANSW\\ (\% of pl)\end{tabular} &
%   \begin{tabular}[c]{@{}c@{}}RT\\ (ms)\end{tabular} &
%   \begin{tabular}[c]{@{}c@{}}ANSW\\ (\% of pl)\end{tabular} &
%   \begin{tabular}[c]{@{}c@{}}RT\\ (ms)\end{tabular} &
%   \begin{tabular}[c]{@{}c@{}}ANSW\\ (\% of pl)\end{tabular} \\
%   \midrule
% Col Sg  & $853 (45)$ & $19.4\% (6.8)$ & $906 (55)$ & $22.0\% (6.7)$ & $-53$ & $-2.6\%$ \\
% Unit Sg & $774 (39)$ & $02.3\% (0.7)$ & $787 (42)$ & $02.6\% (0.7)$ & $-13$ & $-0.3\%$ \\
% Plural  & $820 (46)$ & $97.5\% (0.6)$ & $791 (33)$ & $97.3\% (0.6)$ & $29$  & $0.2\%$  \\
% \lspbottomrule
% \end{tabular}
% \end{table}

\begin{table}[h!]
\caption{Congruity of response hand and number type (SNARC) in Experiment 1 measured in reaction times (ms) and number judgment answers (percent of plural responses). Standard errors in parentheses}
\label{gul-bla:tab:SNARC-exp1}
\begin{tabular}{l rr rr rr}
\lsptoprule
&\multicolumn{6}{c}{Response Hand}\\\cmidrule(lr){2-7}
&\multicolumn{2}{c}{\multirow{2}{*}{Left}}&\multicolumn{2}{c}{\multirow{2}{*}{Right}}&\multicolumn{2}{c}{Congruity}\\
&&&&&\multicolumn{2}{c}{($\text{Left}-\text{Right}$)}\\\cmidrule(lr){2-3}\cmidrule(lr){4-5}\cmidrule(lr){6-7}
&\multicolumn{1}{c}{RT}&\multicolumn{1}{c}{Answ}&\multicolumn{1}{c}{RT}&\multicolumn{1}{c}{Answ}&\multicolumn{1}{c}{RT}&\multicolumn{1}{c}{Answ}\\
Num Type&\multicolumn{1}{c}{(ms)}&\multicolumn{1}{c}{(\% of pl)}&\multicolumn{1}{c}{(ms)}&\multicolumn{1}{c}{(\% of pl)}&\multicolumn{1}{c}{(ms)}&\multicolumn{1}{c}{(\% of pl)}\\\midrule
Col Sg  & $853 (45)$ & $19.4\% (6.8)$ & $906 (55)$ & $22.0\% (6.7)$ & $-53$ & $-2.6\%$ \\
Unit Sg & $774 (39)$ & $2.3\% (0.7)$ & $787 (42)$ & $2.6\% (0.7)$ & $-13$ & $-0.3\%$ \\
Plural  & $820 (46)$ & $97.5\% (0.6)$ & $791 (33)$ & $97.3\% (0.6)$ & $29$  & $0.2\%$  \\
\lspbottomrule
\end{tabular}
\end{table}


\subsubsection{Size congruity effect}
The Number Type×Font Size interaction was not significant either by subjects or by items, see \tabref{gul-bla:tab:ANOVA-exp1}. There was, therefore, no statistically valid evidence for any size congruity effect. See \tabref{gul-bla:tab:SCE-exp1} for reaction times and number judgments.

% \input {./tables/table-4}

% \begin{table}[]
% \caption{Congruity of font size and number type (SCE) in Experiment 1 measured in reaction times (ms) and number judgment answers (percent of plural responses). Standard errors in parentheses}
% \label{gul-bla:tab:SCE-exp1}
% \begin{tabular}{ccccccc}
% \lsptoprule
%                  & \multicolumn{6}{c}{Font Size}                                   \\
% \begin{tabular}[c]{@{}c@{}}Num\\ Type\end{tabular} &
%   \multicolumn{2}{c}{Small} &
%   \multicolumn{2}{c}{Big} &
%   \multicolumn{2}{c}{\begin{tabular}[c]{@{}c@{}}Congruity\\ (Small-Big)\end{tabular}} \\
%  &
%   \begin{tabular}[c]{@{}c@{}}RT\\ (ms)\end{tabular} &
%   \begin{tabular}[c]{@{}c@{}}ANSW\\ (\% of pl)\end{tabular} &
%   \begin{tabular}[c]{@{}c@{}}RT\\ (ms)\end{tabular} &
%   \begin{tabular}[c]{@{}c@{}}ANSW\\ (\% of pl)\end{tabular} &
%   \begin{tabular}[c]{@{}c@{}}RT\\ (ms)\end{tabular} &
%   \begin{tabular}[c]{@{}c@{}}ANSW\\ (\% of pl)\end{tabular} \\
%   \midrule
% Col Sg  & $882 (45)$ & $21.1\% (6.7)$ & $877 (46)$ & $20.3\% (6.6)$ & $5$  & $0.8\%$  \\
% Unit Sg & $778 (39)$ & $20.0\% (0.7)$ & $783 (38)$ & $28.0\% (0.7)$ & $-5$ & $-8.0\%$ \\
% Plural  & $812 (40)$ & $97.3\% (0.6)$ & $798 (36)$ & $97.5\% (0.6)$ & 14 & $-0.2\%$ \\
% \lspbottomrule
% \end{tabular}
% \end{table}

\begin{table}
\caption{Congruity of font size and number type (SCE) in Experiment 1 measured in reaction times (ms) and number judgment answers (percent of plural responses). Standard errors in parentheses}
\label{gul-bla:tab:SCE-exp1}
\begin{tabular}{l rr rr rr}
\lsptoprule
&\multicolumn{6}{c}{Font Size}\\\cmidrule(lr){2-7}
&\multicolumn{2}{c}{\multirow{2}{*}{Small}}&\multicolumn{2}{c}{\multirow{2}{*}{Big}}&\multicolumn{2}{c}{Congruity}\\
&&&&&\multicolumn{2}{c}{($\text{Small}-\text{Big}$)}\\\cmidrule(lr){2-3}\cmidrule(lr){4-5}\cmidrule(lr){6-7}
&\multicolumn{1}{c}{RT}&\multicolumn{1}{c}{Answ}&\multicolumn{1}{c}{RT}&\multicolumn{1}{c}{Answ}&\multicolumn{1}{c}{RT}&\multicolumn{1}{c}{Answ}\\
Num Type&\multicolumn{1}{c}{(ms)}&\multicolumn{1}{c}{(\% of pl)}&\multicolumn{1}{c}{(ms)}&\multicolumn{1}{c}{(\% of pl)}&\multicolumn{1}{c}{(ms)}&\multicolumn{1}{c}{(\% of pl)}\\\midrule
Col Sg  & $882 (45)$ & $21.1\% (6.7)$ & $877 (46)$ & $20.3\% (6.6)$ & $5$  & $0.8\%$  \\
Unit Sg & $778 (39)$ & $20.0\% (0.7)$ & $783 (38)$ & $28.0\% (0.7)$ & $-5$ & $-8.0\%$ \\
Plural  & $812 (40)$ & $97.3\% (0.6)$ & $798 (36)$ & $97.5\% (0.6)$ & 14 & $-0.2\%$ \\
\lspbottomrule
\end{tabular}
\end{table}


\subsection{Discussion}

\subsubsection{Plural interpretation of collectives}
The judgment data showed that participants chose the “more than one thing” answer in 20.7\% of responses in the collective singular condition, compared to just 2.4\% in the unitary singular condition and 97.4\% in the plural condition. This outcome is similar to the number judgment results for collectives obtained in earlier studies with speakers of English \citep{bockMeaningSoundSyntax1993} and Finnish \citep{nenonenMismatchesGrammaticalNumber2010}. Polish speakers participating in the experiment were aware that collective nouns can refer to multiple objects despite their grammatical singularity, even though they were more likely to treat nouns from this category as semantically singular. 

\subsubsection{SNARC effect}
The interaction of the type of number (collective singular, unitary singular, plural) and the response hand was significant, although only in a by-items analysis.

For unitary singular nouns, participants responded faster with the left hand than with the right hand. The opposite was true for plural nouns. This pattern resembled the SNARC effect observed for small and large numbers \citep{dehaeneMentalRepresentationParity1993,geversNumbersSpaceComputational2006,gobelCulturalNumberLine2011} and the findings for grammatical number in German \citep{rottgerGrammaticalNumberElicits2015}. Polish comprehenders in the experiment automatically associated grammatically singular nouns with the left side of the mental space, while grammatically plural nouns were linked with the right side. This is consistent with the idea that processing numerical magnitudes engages representations arranged on a mental number line \citep{dehaeneMentalRepresentationParity1993,gobelCulturalNumberLine2011,paveseSymbolicDistanceNumerosity1998}. Crucially for the main research question, collective singulars behaved like unitary singulars. This suggests that overall collective singulars were automatically conceptualized as referring to the collection as a whole, which is consistent with the semantic number judgments in the present experiment and the results of past research  \citep{bockMeaningSoundSyntax1993, nenonenMismatchesGrammaticalNumber2010, bockNumberAgreementBritish2006}. Thus, the primary factor determining the conceptual representation of the objects denoted by collective nouns appears to be their grammatical number.

\subsubsection{Size congruity effect}
The interaction between the type of number and the visual size of the font was not significant. There was, therefore, no evidence that either grammatical number or collectivity can cause a size congruity effect. In particular, grammatical singularity and plurality did not activate small size and big size representations, respectively, despite giving rise to a SNARC effect. This result is surprising. A group of individuals is typically larger than a single individual of this category, yet the group size does not seem to be part of the mental representation of number for language comprehenders. Perhaps this underrepresentation in terms of size is due to the fact that plurals can easily refer to very small groups, possibly of just two individuals. The lack of a size congruity effect for grammatical number may also suggest that understanding the semantic contribution of grammatical number depends on the part of numerical cognition linking numerosities with the processing of space (hence the observed SNARC effect), but not with the processing of continuous magnitudes, like size.

It is also possible that the emergence of a size congruity effect was blocked by certain design features of Experiment 1. Experiment 2 tested this possibility. 

\section{Experiment 2}
Experiment 1 showed no sign of a size congruity effect. The SNARC effect was present, but it was statistically significant only in a by-items analysis. The lack of an SCE and a statistically weak SNARC effect may have been due to design choices, so another experiment was conducted addressing some of the possible problems. Changes were introduced in three areas: the selection of nouns for the collective singular condition, the choice of plural counterparts for collective singulars and the choice of the task.

\subsection{Research goal and predictions}
As in Experiment 1, the research problem investigated in Experiment 2 concerned whether the primary numerical representation of the referents of collective singular nouns is driven by their grammatical or lexical status.  If collective singulars are associated primarily with the conceptual singularity based on their grammatical number, they should behave more like unitary singular nouns. If collective singulars are linked with conceptual plurality through the lexical emphasis on the elements of the collection, they should pattern with grammatically plural nouns in terms of the SNARC effect and, possibly, the SCE. If both representations are automatically activated early on (competing for selection), the results for collective singulars should fall somewhere between unitary singulars and plurals.

\subsection{Design}

\subsubsection{Materials}
Collective nouns for Experiment 1 were chosen based on the authors’ intuition. For Experiment 2, a pretest was organized to select nouns  whose collective reading is most salient. A questionnaire with a list of words was presented to participants, who evaluated how often every word was used to refer to more than one entity. Participants made their decision on a scale from 1 (very rarely) to 5 (very often). The list contained 188 words of which 62 were singular nouns with a potentially collective reading (e.g., \textit{ekipa} ‘squad’). The remaining words were unitary singulars (e.g., \textit{wilk} ‘wolf’), \textit{pluralia tantum} (e.g., \textit{nożyce} ‘scissors’), mass nouns (e.g., \textit{błoto} ‘mud’) and ordinary plurals (e.g., \textit{drzewa} ‘trees’). The questionnaire was distributed online through Google Forms. Ten native speakers of Polish took part. Responses for each item were averaged over all participants. Thirty collective nouns with the highest scores were selected for the experiment. Of the selected nouns, the lowest rated item (\textit{sztab} ‘military headquarters’) received 3.6 points and the highest rated (\textit{trzoda} ‘lifestock’) received 4.7 points ($\text{M}=4.22, \text{SD}=0.27$). In Experiment 1, instead of pluralizing collective singulars (e.g., \textit{armie} ‘armies’ for \textit{armia} ‘army’), plural forms of related unitary nouns (e.g., \textit{żołnierze} ‘soldiers’ for \textit{armia} ‘army’) were used. While this was done to avoid a potential effect of number syncretism and “double plurality”, it may have introduced more variance among items. In Experiment 2, plural forms were created from collective singulars. In addition to the collective nouns, 30 unitary singular nouns and their plural forms were selected. 

Overall there were 60 singular and 60 plural nouns. Each noun was presented in big font and small font as well as with a left hand and right hand response. Every participant saw all items. This resulted in 480 trials distributed over two blocks. The presentation order was fully randomized for every participant.

\subsubsection{Procedure}
Experiment 2 was conducted on the same standard PC and 23.6 inch monitor as Experiment 1.
The design was mostly the same as in Experiment 1, the only difference being the task. The task used in Experiment 1 (semantic number judgment) was chosen to make the results comparable with past number judgment studies \citep{bockMeaningSoundSyntax1993, nenonenMismatchesGrammaticalNumber2010} and to follow closely the design of \citet{rottgerGrammaticalNumberElicits2015}, where a SNARC effect for grammatical number was demonstrated. However, that task may have drawn the participants’ attention to the number ambiguity of collectives, thereby affecting the outcome. Experiment 2 addressed this problem by encouraging participants to focus on the grammatical number instead. The participants were instructed to determine whether the noun is grammatically singular or plural (grammatical number judgment) while ignoring the visual size of the stimulus. The font sizes in the two size conditions and the resulting visual angles for stimuli were the same as in the previous experiment.

Experiment 2 again consisted of two blocks, with the assignment of keys to responses changing after the first block. There were three breaks within each block (every 60 trials). In each block, the experiment proper was preceded by a training session with 22 trials. The set of training items consisted of nouns balanced in terms of grammatical number, font size and response hand. None of the items used in the training session appeared later in the experiment proper. If the participant made a mistake, feedback was provided  in the form of a message (\textit{źle} `incorrect') that stayed on the screen for 1 second. In the training session a message also appeared after correct responses (\textit{dobrze} `correct'). The main purpose of the feedback was to facilitate learning the correct assignment of keys.

The experiment was designed and presented using the PsychoPy software (version 1.84.2) \citep{peircePsychoPyPsychophysicsSoftware2007,peirceGeneratingStimuliNeuroscience2009}.

\subsubsection{Participants}
Twenty-three students of the Institute for English Studies of the University of Wrocław (15 women, 8 men) took part in the experiment. Participants were all native speakers of Polish. The average age was 22.4 ($\text{SD}=5.5$).

\subsubsection{Results: Accuracy}
In Experiment 2, participants were required to focus on the grammatical number of words and decide whether each noun is gramatically singular or plural. The accuracy measure, therefore, did not reflect the numerical semantics of the nouns. This time the differences between the types of number were very small. Participants were on average most accurate with unitary singular nouns ($\text{M}=98.5\%$, $\text{SE}=0.6$) and slightly less accurate with collective singulars ($\text{M}=97.3\%$, $\text{SE}=0.6$) and plurals ($\text{M}=97\%, \text{SE}=0.4$). A pair of one-way ANOVA tests (by subjects and by items) with Accuracy as the dependent variable and Number Type (collective singular, unitary singular, plural) as the independent factor showed that these differences were significant by subjects ($\text{F}_{1}(2,44)=5.46, p=0.008, \eta^2=0.20$) but not by items ($\text{F}_{2}(2,117)=1.34, p=0.27$).

\subsection{Results: Reaction times}
The data were cleaned first by removing all incorrect responses. After that, all trials with reaction times (RT) 3 standard deviations above and below the mean for every participant were removed. This resulted in eliminating 215 data points which constituted 2\% of correct responses. The remaining trials were subjected to tests performed with the SPSS software (version 22).

In order to test the research hypotheses, a pair of 3×2×2 ANOVA tests (by subjects and by items) were conducted with RT as the dependent variable and the following independent factors:

\begin{itemize}
\item Number Type (collective singular, unitary singular, plural)
\item Font Size (small, big)
\item Response Hand (left, right)
\end{itemize}

Results of the ANOVA tests are given in \tabref{gul-bla:tab:ANOVA-exp2}. Mean reaction times and accuracy in each condition are given in \tabref{gul-bla:tab:all-conditions-exp2}.

\begin{table}
\sisetup{table-space-text-post = *}
\caption{ANOVA test results for Experiment 2. NT: Number Type; FS: Font Size; RH: Response Hand.}
\label{gul-bla:tab:ANOVA-exp2}
\begin{tabular}{l cc rr *{2}{S[table-format=<1.3,table-align-text-post=false]} rr}
\lsptoprule
Source  & \multicolumn{2}{c}{df} & \multicolumn{2}{c}{F} & \multicolumn{2}{c}{$p$}   & \multicolumn{2}{c}{Partial $\eta^2$} \\
\cmidrule(lr){2-3}\cmidrule(lr){4-5}\cmidrule(lr){6-7}\cmidrule(lr){8-9}
                 & F1      & F2    & \multicolumn{1}{c}{F1}     & \multicolumn{1}{c}{F2}    & {F1}       & {F2}     & \multicolumn{1}{c}{F1}      & \multicolumn{1}{c}{F2}    \\
\midrule
NT &
  2, 44 &
  2, 117 &
  20.31 &
  9.82 &
  <0.001* & 
  <0.001* & 
  0.48 &
  0.14 \\
FS &
  1, 22 &
  1, 117 &
  0.02 &
  0.06 &
  0.893 &
  0.815 &
  0.00 &
  0.00 \\
RH &
  1, 22 &
  1, 117 &
  0.47 &
  1.17 &
  0.499 &
  0.281 &
  0.02 &
  0.01 \\
NT×FS&
  2, 44 &
  2, 117 &
  2.57 &
  1.03 &
  0.088 &
  0.361 &
  0.11 &
  0.02 \\
NT×RH &
  2, 44 &
  2, 117 &
  0.07 &
  0.22 &
  0.932 &
  0.803 &
  0.00 &
  0.00 \\
FS×RH &
  1, 22 &
  1, 117 &
  2.35 &
  1.16 &
  0.140 &
  0.283 &
  0.10 &
  0.01 \\
NT×FS×RH &
  2, 44 &
  2, 117 &
  2.86 &
  1.55 &
  0.068 &
  0.216 &
  0.12 &
  0.03 \\
\lspbottomrule
\end{tabular}
\end{table}

%\input {./tables/table-7.tex}

% Please add the following required packages to your document preamble:
% \usepackage{multirow}
% \begin{table}[]
% \caption{Mean reaction times (ms) and accuracy (percent correct) in all conditions in Experiment 2. Standard errors in parentheses}
% \label{gul-bla:tab:all-conditions-exp2}
% \begin{tabular}{cccccccc}
% \lsptoprule
% \multicolumn{2}{c}{}                               & \multicolumn{6}{c}{Response Hand}                                \\
% \multirow{2}{*}{\begin{tabular}[c]{@{}c@{}}Num\\ Type\end{tabular}} &
%   \multirow{2}{*}{\begin{tabular}[c]{@{}c@{}}Font\\ Size\end{tabular}} &
%   \multicolumn{2}{c}{Left} &
%   \multicolumn{2}{c}{Right} &
%   \multicolumn{2}{c}{\begin{tabular}[c]{@{}c@{}}Congruity\\ (Left-Right)\end{tabular}} \\
%  &
%   &
%   \begin{tabular}[c]{@{}c@{}}RT\\ (ms)\end{tabular} &
%   \begin{tabular}[c]{@{}c@{}}Acc\\ (\% corr)\end{tabular} &
%   \begin{tabular}[c]{@{}c@{}}RT\\ (ms)\end{tabular} &
%   \begin{tabular}[c]{@{}c@{}}Acc\\ (\% corr)\end{tabular} &
%   \begin{tabular}[c]{@{}c@{}}RT\\ (ms)\end{tabular} &
%   \begin{tabular}[c]{@{}c@{}}Acc\\ (\% corr)\end{tabular} \\
%   \midrule
% \multirow{2}{*}{Col Sg}  & Small & $830 (34)$ & $96.2\% (1.1)$ & $834 (39)$ & $97.4\% (7.0)$ & $-4$  & $-1.2\%$ \\
%                                   & Big   & $818 (31)$ & $97.1\% (7.0)$ & $830 (36)$ & $98.6\% (6.0)$ & $-12$ & $-1.5\%$ \\
% \multirow{2}{*}{Unit Sg} & Small & $765 (27)$ & $98.8\% (4.0)$ & $743 (24)$ & $98.7\% (4.0)$ & $22$  & $0.1\%$  \\
%                                   & Big   & $755 (26)$ & $98.1\% (5.0)$ & $776 (28)$ & $98.3\% (5.0)$ & $-21$ & $-0.2\%$ \\
% \multirow{2}{*}{Plural}  & Small & $794 (31)$ & $97.0\% (4.0)$ & $808 (30)$ & $97.5\% (6.0)$ & $-14$ & $-0.5\%$ \\
%                                   & Big   & $798 (32)$ & $96.4\% (5.0)$ & $802 (30)$ & $97.0\% (4.0)$ & $-4$  & $-0.6\%$ \\
% \lspbottomrule
% \end{tabular}
% \end{table}

\begin{table}
\caption{Mean reaction times (ms) and accuracy (percent correct) in all conditions in Experiment 2. Standard errors in parentheses}
\label{gul-bla:tab:all-conditions-exp2}
\begin{tabularx}{.97\textwidth}{X r@{~~~}r r@{~~~}r r@{~~~}r}
\lsptoprule
&\multicolumn{6}{c}{Response Hand}\\\cmidrule(lr){2-7}
&\multicolumn{2}{c}{\multirow{2}{*}{Left}}&\multicolumn{2}{c}{\multirow{2}{*}{Right}}&\multicolumn{2}{c}{Congruity}\\
&&&&&\multicolumn{2}{c}{($\text{Left}-\text{Right}$)}\\\cmidrule(lr){2-3}\cmidrule(lr){4-5}\cmidrule(lr){6-7}
\textit{Num Type}&\multicolumn{1}{c}{RT}&\multicolumn{1}{c}{Acc}&\multicolumn{1}{c}{RT}&\multicolumn{1}{c}{Acc}&\multicolumn{1}{c}{RT}&\multicolumn{1}{c}{Acc}\\
\hspace{6pt}Font Size&\multicolumn{1}{c}{(ms)}&\multicolumn{1}{c}{(\% corr)}&\multicolumn{1}{c}{(ms)}&\multicolumn{1}{c}{(\% corr)}&\multicolumn{1}{c}{(ms)}&\multicolumn{1}{c}{(\% corr)}\\\midrule
\textit{Col Sg}\\
\hspace{6pt}Small& $830 (34)$ & $96.2\% (1.1)$ & $834 (39)$ & $97.4\% (7.0)$ & $-4$  & $-1.2\%$ \\
\hspace{6pt}Big& $818 (31)$ & $97.1\% (7.0)$ & $830 (36)$ & $98.6\% (6.0)$ & $-12$ & $-1.5\%$ \\\tablevspace
\textit{Unit Sg}\\
\hspace{6pt}Small& $765 (27)$ & $98.8\% (4.0)$ & $743 (24)$ & $98.7\% (4.0)$ & $22$  & $0.1\%$  \\
\hspace{6pt}Big& $755 (26)$ & $98.1\% (5.0)$ & $776 (28)$ & $98.3\% (5.0)$ & $-21$ & $-0.2\%$ \\\tablevspace
\textit{Plural}\\
\hspace{6pt}Small&$794 (31)$ & $97.0\% (4.0)$ & $808 (30)$ & $97.5\% (6.0)$ & $-14$ & $-0.5\%$ \\
\hspace{6pt}Big& $798 (32)$ & $96.4\% (5.0)$ & $802 (30)$ & $97.0\% (4.0)$ & $-4$  & $-0.6\%$ \\
\lspbottomrule
\end{tabularx}
\end{table}


\subsubsection{Number Type effect}
The main effect of Number Type was significant. Responses to collective singular nouns were on average longest ($\text{M}=828\text{ms}, \text{SE}=33$), followed by responses to plural nouns ($\text{M}=801\text{ms}, \text{SE}=29$) and to unitary singular nouns ($\text{M}=760\text{ms}, \text{SE}=24$). No other main effect was significant.

\subsubsection{SNARC effect}
The Number Type×Response Hand interaction was not significant either by subjects or by items. There was no statistically valid evidence for a SNARC effect.

\subsubsection{Size congruity effect}
The Number Type×Font Size interaction was not significant either by subjects or by items. There was no statistically valid evidence for a size congruity effect. 

\subsection{Discussion}
Experiment 2 introduced some changes to the design of Experiment 1 as an attempt to strengthen the SNARC effect and elicit a size congruity effect. However, this time both effects were absent. The results showed no interaction of number with either the response side or visual size. 

The main change in Experiment 2 with respect to Experiment 1 was a change in the task. The semantic number judgment task of deciding whether the word named one or more than one thing from Experiment 1 was replaced with the grammatical number judgment task of deciding whether the word was grammatically singular or plural. The change was intended to turn the participants’ attention away from the number ambiguity of collective singulars while keeping the task in the domain of number. However, it is possible that the fact that conceptual number in Experiment 2 was irrelevant for the task meant that it was not extracted fast enough to affect the performance and produce a SNARC effect. This would be in line with the results of \citet{rottgerGrammaticalNumberElicits2015}, who found a SNARC effect for singular and plural German nouns only for the task requiring the processing of semantic number but not for tasks related to other types of information (animacy semantics, lexical status, visual features). In the present study, the SNARC effect remained absent for a task involving paying attention to grammatical number.

\section{General discussion}
The two experiments reported here investigated the numerical representation of the referents of collective singular nouns. The main research problem concerned the question whether language comprehenders construe the entities denoted by collective singular nouns primarily in terms of conceptual singularity (determined by their grammatical number) or conceptual plurality (determined by their lexical semantics). In Experiment 1 collective singular nouns behaved overall like unitary singular nouns and differed from plural nouns in terms of the SNARC effect. Plural nouns received faster responses with the right hand than with the left hand. In contrast, collective and unitary singulars showed a clear preference for the left hand. This fits the hypothesis that the reference of a collective noun is initially construed as a single entity (the whole group), consistent with the grammatical singularity of the word, and the plural interpretation is secondary to this initial singularity, resulting from the highlighting of component parts. 

Some tentative conclusions for models of grammatical number processing can be offered based on our findings. For words with a conflict between the grammatical and lexical number, like collective nouns, the number mismatch seems to be resolved in favor of the grammatical information. The data obtained in the present experiments suggests that such words initially activate numerical concepts consistent with their grammatical number. Comprehenders seem to expect grammatical number to be a reliable cue for the numerosity of the objects under discussion. This is true even if the lexical specification of a noun is at odds with its morphosyntactic marking. This independence of the primary number representation from lexical factors like collectivity suggests that the extraction of grammatical number information is automatic and happens soon after a noun is encountered, possibly before or in parallel to the lexical semantics. This may follow from the status of number as a grammatical category. Electrophysiological studies show the separability of semantic and morphosyntactic processes in the form of separate early ERP components, with signs of interaction between the two types of information visible in relatively late time windows \citep{aFriedericineuralbasisauditory2002}. Effects of semantic manipulations are commonly observed as amplitude modulations of the N400, which is a component peaking around 400ms after stimulus onset \citep{aKutasThirtyyearscounting2011}. Processes that require access to the syntactic category of a word are reflected in the amplitude of the eLAN, an early component peaking around 150--300ms after stimulus onset \citep{aHahneElectrophysiologicalevidencetwo1999}, which has been found for word-category violations even in meaningless “jabberwocky” sentences \citep{aHahneWhatleftif2001}. Manipulations involving specifically grammatical number affect the amplitude of the LAN, a component related to morphosyntactic processes \citep{munteBrainActivityAssociated1997,friedericiTimeCourseSyntactic1995} peaking around the same time as the N400 \citep{aBarberGrammaticalgendernumber2005,aLuckBrainpotentialsmorphologically2006}.\footnotemark{} Thus ERP evidence points to lexical and grammatical information being processed independently at an early stage of comprehension. This is consistent with the present findings.
\footnotetext{Even though the N400 and the LAN are both negative going components peaking around the same time, they can be distinguished by the distributions of the electrodes picking them at the scalp. Whereas the distribution of the N400 is centro-parietal, the LAN is most prominent at the left-anterior sites.}

There was no evidence from either Experiment 1 or Experiment 2 that the conceptual representation of number in language can lead to a size congruity effect. This null result may indicate the limits of mental simulations based on linguistic information  \citep{barsalouPerceptualSymbolSystems1999,zwaanMentalSimulationLanguage2009,patsonConceptualRepresentationNumber2014}. It seems that the size of the denoted set is not a salient property of the conceptual representations of grammatical number. In the original study by \citet{paivioPerceptualComparisonsMind1975} participants had problems comparing real life sizes of depicted objects if the image sizes were incongruent (e.g., the image of a lamp bigger than the image of a zebra). Given the results of Paivio's study, it is possible that participants in the present study focused more on the size of typical individuals constituting a given group than the size of the group itself. The nouns used in the two experiments were not matched for average sizes of the denoted individuals. The items included words naming relatively small objects (e.g., \textit{pasek} ‘belt’) as well as names for bigger things (e.g., \textit{stół} ‘table’). Perhaps a more careful choice of items is necessary to detect a size congruity effect related to grammatical number or collectivity.

From a methodological perspective, the results of Experiment 1 confirm the suitability of the SNARC effect elicited by semantic number judgments as a tool for studying the conceptual representation of number in language. However, the complete absence of the effect in Experiment 2, which used grammatical number judgments, points to the task-sensitive nature of this effect, consistent with the results of \citet{rottgerGrammaticalNumberElicits2015}. The lack of the size congruity effect in both experiments means that more research is needed to determine whether it can be a suitable diagnostic of number interpretation for grammatical number studies.

\section*{Acknowledgements}
Fragments of this work were adapted from a chapter of Piotr Gulgowski’s unpublished PhD dissertation written under the supervision of Joanna Błaszczak. The research presented here was funded by the Polish National Science Center (NCN) (grant number 2013/09/B/HS2/02763).

{\sloppy\printbibliography[heading=subbibliography,notkeyword=this]}

\end{document}
