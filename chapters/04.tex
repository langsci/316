\documentclass[output=paper]{langscibook} 

% local compile: xelatex  -interaction=nonstopmode main.tex
% biber main
% added files: jambox.sty, langsci-basic.sty, langsci-cgloss.sty, langsci-gb4e.sty, langsci-optional.sty, scrpage2.sty
% and reverted back
% regex cleaning: 
%     - \\textstyleDefaultParagraphFont\{(.+?)\} -> $1
%     - \\textrm\{(.+?)\} -> $1
%     - \n\n\n -> \n\n
%

\author{Wiktor Pskit\orcid{0000-0003-2786-4741}\affiliation{University of Lodz}}
\title[Syntactic reduplication and plurality]{Syntactic reduplication and plurality: On some properties of NPN subjects and objects in Polish and English}

\abstract{This paper is concerned with selected properties of noun--preposition--noun (NPN) clausal subjects and objects (e.g. \textit{day after day/dzień po dniu}) in English and Polish. At the descriptive level, the relevant phenomena include NPN subject-verb agreement and the aspectual features of verbs co-occurring with NPN subjects and objects. The phenomena are discussed in the light of the ``internal'' properties of NPN structures derived by the mechanism of iterative (syntactic) reduplication developed in \cite{Travis2001,Travis2003} where a reduplicative head (Q) copies the complement of the preposition. The copy of the noun moves to SpecQP. Both nouns are treated as ``defective'' nominals (\textit{n}Ps) due to the absence of the DP-layer since the presence of determiners is excluded (arguably cross-linguistically). The whole NPN is syntactically singular though semantically it encodes plurality (a sequence or succession of entities or events). In both English and Polish the singular character of NPN subjects is manifested by their co-occurrence with singular rather than plural verbs. Whenever such NPNs are subjects or objects, they only occur with imperfective verbs in Polish. While this is not morphologically marked in English, English clauses with NPN subjects or objects only allow imperfective interpretation too.

\keywords{reduplication, iteration, plurality, agreement, aspect}}

\begin{document}
\SetupAffiliations{mark style=none}
\maketitle

\section{Introduction}

Although the key characteristics of the syntax and semantics of \textsc{noun--pre\-po\-si\-tion--noun} (NPN) structures (e.g. \textit{day after day} in English, \textit{dzień po dniu} in Polish) are discussed in a number of studies (see \citealt{Pi1995,Travis2001,Travis2003,Beck.Stechow2007,Jackendoff2008,Dobaczewski2009,Dobaczewski2018,Rosalska2011,Haik2013,Pskit2015,Pskit2017}), the properties of NPNs functioning as clausal subjects and objects have not yet been investigated.

\sectref{psk:sec:sec2} presents the basic internal properties of NPNs in English and Polish, mainly based on what is reported in earlier studies. It also proposes an account of the mechanism responsible for the derivation of NPNs, which is a revised version of an earlier proposal in \cite{Travis2001,Travis2003}. \sectref{psk:sec:sec3} is concerned with the behaviour of argument NPNs: their status as subjects and objects, NPN subject-verb agreement patterns, and aspectual characteristics of the verb with NPN subject or object in Polish. \sectref{psk:sec:sec4} summarises the discussion, offers some tentative conclusions, and remarks on prospects for further research on the topic.

The current study constitutes but a preliminary look at the relevant problems and the observations made below need to be confronted with data from other languages.

\section{The structure and internal properties of NPN structures}\label{psk:sec:sec2}
\subsection{NPNs and related structures}

What comes to be called NPN in the relevant literature represents a heterogeneous inventory of structures. Thus, there are idiomatic NPNs with a restricted selection of different nouns (e.g. \textit{cheek by jowl}, \textit{hand over fist}) and more regular NPN patterns with several prepositions but without lexically constrained nominals (e.g. \textit{day by day}, \textit{bumper to bumper}, \textit{layer upon layer}). The latter category includes a number of highly lexicalised instances, such as \textit{face to face/twarzą w twarz} ‘face.\textsc{ins} in face.\textsc{acc’}. The productive pattern involves the English prepositions \textit{by}, \textit{for}, \textit{to}, \textit{after} and \textit{upon} (\citealt{Pi1995,Jackendoff2008}) and the Polish prepositions \textit{w} ‘in’, \textit{po} ‘after’, \textit{za} ‘behind/for/after/by’, \textit{przy} ‘next to/close to’ and \textit{obok} ‘next to’ (\citealt{Rosalska2011,Pskit2015,Dobaczewski2018}). Thus understood NPN structures are distinguished from PNPN constructions with identical (e.g. \textit{from cover to cover/od deski do deski} ‘from board.\textsc{gen} to board.\textsc{gen}’, \textit{from door to door}) or different nominals (e.g. \textit{from mother to daughter}, \textit{from shelf to floor, z ojca na syna }‘from father.\textsc{acc} to son.\textsc{acc}’) (cf. \citealt{Zwarts2013}). In particular, (P)NPN with the optional initial \textit{from} in English can give an impression of being NPN, as in \citeauthor{Jackendoff2008}’s (\citeyear{Jackendoff2008}: 12) examples below (cf. also \citealt{Zwarts2013}: 70):

\ea\label{psk:ex:1} \ea Adult coloration is highly variable (from) snake to snake.
\ex (From) situation to situation, conditions change.
\z\z

\noindent An important characteristic of NPN structures with identical nouns is that they seem to involve some combination of the doubling of language form (identical nominals ‘surrounding’ the preposition) and the plurality (or iteration) in terms of interpretation.\footnote{For more on different approaches to the semantics of NPN structures see \cite{Beck.Stechow2007} and \cite{Jackendoff2008}.} As \citet[280]{Quirk.etal1985} observe, in such NPNs “two nouns are placed together in a parallel structure”.

The present paper focuses on the productive subtype of NPNs with the English prepositions \textit{after} and \textit{upon} and the Polish prepositions \textit{po} ‘after’ and \textit{za} ‘after/by’ (lit. ‘behind’), because only such NPNs occur as clausal arguments. As observed in other studies, while some NPNs allow dual (in \citepossalt{Jackendoff2008} terms: the sense of juxtaposition of two entities or matching of two entities or sets of entities) or plural readings (succession in \citealt{Jackendoff2008}), those with \textit{after/upon} in English and with \textit{po/za} in Polish have invariably plural readings.

\subsection{Constraints on NPN-internal nominals}\label{psk:sec:constraints}

In both Polish and English, there are similar constraints on the nominals in NPNs. There is preference for countable singular nouns in  both N\textsubscript{1} and N\textsubscript{2} position in N\textsubscript{1}PN\textsubscript{2}. As a result, uncountable \REF{psk:ex:2} and plural countable nominals \REF{psk:ex:key:3} appear to be ruled out (English data from \citealt{Jackendoff2008}):

\ea \label{psk:ex:2}  \ea[*]{water after water, * dust for dust}\label{psk:ex:key:2a}
\ex[*]{\gll  odzież            za     odzieżą\\
clothes.\textsc{sg.nom} after clothes.\textsc{sg.ins}\\
\glt         Literally: ‘clothes after clothes’}
\z\z

\ea \label{psk:ex:key:3}  \ea[*]{men for men, * books after books, * weeks by weeks}\label{psk:ex:key:3a}
\ex[*]{\gll  książki         za    książkami\\
            books.\textsc{pl.nom} after books.\textsc{pl.ins}\\
\glt         Literally: ‘books after books’}
\ex[*]{\gll tygodnie       po         tygodniach\\
            weeks.\textsc{pl.nom} after/by weeks.\textsc{pl.loc}\\
\glt         Literally: \textsc{‘}weeks by weeks’}
\z
\z

\noindent An obvious counterexample to the ban on mass nouns \REF{psk:ex:key:2a} and plurals \REF{psk:ex:key:3a} is the expression found in the Anglican burial service:

\ea \label{psk:ex:key:4}  … earth to earth, ashes to ashes, dust to dust ...\\
\z

\noindent However, it is an instance of formulaic language and the NPNs \textit{ashes to ashes} and \textit{dust to dust} – whether used separately or together – have attained the status of idiom(s) rather than given rise to a productive pattern. It is also possible to interpret the data in \REF{psk:ex:key:4} as elided versions of their clausal counterparts. The English NPNs with the preposition \textit{upon} provide further problems with regard to the aforementioned constraint on nominals. What turns out to be relatively productive is the occurrence of mass nouns that undergo the well-known process of semantic recategorisation (mass / uncountable → countable):

\ea \label{psk:ex:key:5}  Absurdity upon absurdity.\hfill (Internet)\\
\z

\noindent Its Polish counterpart (though unattested) would definitely have a countable reading (‘a number of instances of absurdity following one another’):

\ea\gll\label{psk:ex:key:6}absurd                za             absurdem\\
       absurdity\textsc{.sg.nom} after/upon absurdity.\textsc{sg.ins}\\
\glt   ‘absurdity upon absurdity’\z

\noindent A semantically related and well-attested clausal counterpart also involves the doubling of the nominal that is countable, but such clausal structures are beyond the scope of the present analysis:

\ea\gll\label{psk:ex:key:7}Absurd                 goni              absurd.\\
       absurdity.\textsc{3sg.nom} chase.\textsc{3sg.prs} absurdity.\textsc{3sg.acc}\\
\glt ‘It is absurdity upon absurdity.’
\z

\noindent The English \textit{upon} turns out to be a ``troublemaker'' in the context of NPNs that permit plurals such as \textit{millions} below:

\ea \label{psk:ex:key:8}  … there are millions upon millions who support your decision …\vspace{-12pt}\\\null\hfill (Internet)\\
\z

\noindent While \textit{millions} has morphological plural marking, its plural sense is non-specific: a very large but non-specific number/amount. One way to account for this apparent exception to the ban on plural nominals in NPNs is to rely on \citeposst{Acquaviva2008} notion of lexical plurals. In spite of their plural inflectional marking, the English \textit{hundreds}, \textit{thousands} or \textit{millions} are instances of number neutralisation, in the sense of neutralisation of the singular-plural opposition (\citealt{Acquaviva2008}: 23, 26), or in \citeauthor{Link1998}'s (\citeyear{Link1998}: 221) wording they “have the \textit{form} of a plural, but their reference is \textit{transnumeral}” (emphasis in original). Then the ban on mass nouns and plurals should perhaps be rephrased in terms of number-neutrality or in terms of an unvalued number feature: bare nominals occur as N\textsubscript{1} and N\textsubscript{2}, because they are number-neutral or their number features are unvalued.\footnote{As pointed out by an anonymous reviewer, the notion of unvalued feature seems to be more appropriate than that of number-neutrality, esp. if the latter is understood as general number.} The doubling of the nominals is responsible for the plural interpretation. This makes the presence of \textit{millions} in \REF{psk:ex:key:8} somewhat redundant from a semantic point of view.

The ‘bareness’ of N\textsubscript{1} and N\textsubscript{2} is also reflected by the absence of any kind of determinative material: articles (in English), demonstratives and indefinite determiners (in Polish and English):

\ea\label{psk:ex:key:9}\ea[*]{the man for the man, * a day after a day}
\ex[*]{some inch by some inch\hfill \citep[9]{Jackendoff2008}}
\ex[*]{\gll ten dzień                    po    tym / tamtym dniu\\
            this.\textsc{sg.nom} day.\textsc{sg.nom} after this {} that.\textsc{sg.loc} day.\textsc{sg.loc}\\
\glt         Literally: ‘this day after this/that day’}
\ex[*]{\gll  jakiś dzień                     po    jakimś dniu\\
            some.\textsc{sg.nom} day.\textsc{sg.nom} after some.\textsc{sg.loc} day.\textsc{sg.loc}\\
\glt         Literally: ‘some day after some day’}
\z
\z

\noindent All in all, the doubling of the nominals seems to yield the meaning of plural. Obviously, the identical nominals – though with different morphological case markings in Polish – capture identity of sense rather than identity of reference.

\subsection{Modification of NPN-internal nominals}\label{psk:sec:sec2-3}

Usually the nominals cannot be modified \REF{psk:ex:key:10} (examples from \citealt{Jackendoff2008}), although \textit{after} and \textit{upon} allow premodification and postmodification \REF{psk:ex:key:11} (examples from \citealt{Jackendoff2008} and \citealt{Haik2013}). Interestingly, in English both premodifiers and postmodifiers occur either on both N\textsubscript{1} and N\textsubscript{2} \REF{psk:ex:key:11-a} or just on N\textsubscript{2} \REF{psk:ex:key:11-b}--\REF{psk:ex:key:11-c}. Moreover, both \textit{after} and \textit{upon} allow iteration \REF{psk:ex:key:11-e}.

\ea \label{psk:ex:key:10}  \ea * father of a soldier for father of a soldier\\
\ex   * day of rain to day of rain\\
\z\z

\ea \label{psk:ex:key:11}  \ea\label{psk:ex:key:11-a} tall boy after tall boy\\
\ex\label{psk:ex:key:11-b} day after miserable day\\
\ex\label{psk:ex:key:11-c}   day after day of rain\\
\ex   layer upon layer of mud\\
\ex\label{psk:ex:key:11-e}   day after day after day of unending rain\\
\z\z

\noindent By contrast, Polish NPNs with relatively productive \textit{po} ‘after’ and \textit{za} ‘af\-ter/up\-on/be\-hind’ exhibit lower acceptability of modification \REF{psk:ex:key:12}, and if modification is marginally acceptable, which is more likely in the context of premodification, then it is found on either both N\textsubscript{1} and N\textsubscript{2}, as in English, or only on N\textsubscript{1}, as opposed to the English data in \REF{psk:ex:key:11}.

\ea \label{psk:ex:key:12}  \ea[?]{\gll deszczowy dzień           za    deszczowym dniem\\
             rainy.\textsc{sg.nom} day.\textsc{sg.nom} after rainy.\textsc{sg.ins} day.\textsc{sg.ins}\\
\glt         Literally: ‘rainy day after/upon rainy day’}
\ex[?]{\gll   deszczowy dzień        po    dniu\\
          rainy.\textsc{sg.nom} day.\textsc{sg.nom} after day.\textsc{sg.loc}\\
\glt      Literally: ‘rainy day after day’}
\ex[??]{\gll   dzień deszczu  za    dniem deszczu\\
            day.\textsc{nom} rain.\textsc{gen} after day.\textsc{ins} rain.\textsc{gen}\\
\glt        Literally: ‘day of rain after day of rain’}
\ex[*]{\gll   dzień deszczu    za   dniem\\
            day.\textsc{nom} rain.\textsc{gen} after day.\textsc{ins}\\
\glt        Literally: ‘day of rain after day’}
\ex[*]{\gll   dzień    za    dniem deszczu\\
            day.\textsc{nom} after day.\textsc{ins} rain.\textsc{gen}\\
\glt        Literally: ‘day after day of rain’}
\z
\z

\noindent While the availability of modification does not seem to directly affect the issue of number in NPNs, the nominal concord involving morphological marking of number, gender and case on the noun and its premodifier in Polish does have implications for the account of the structure and derivation of NPNs, as is made clear in \sectref{psk:sec:sec-2-4} below.

\subsection{The structure of NPN via syntactic reduplication}\label{psk:sec:sec-2-4}

Following \cite{Travis2001,Travis2003}, I assume that NPNs are derived by the mechanism of iterative (syntactic) reduplication, where a reduplicative head (Q) copies the complement of the preposition. The copy of the noun moves to SpecQP as in \figref{psk:fig:fig1} below.

\begin{figure}

\begin{forest}for tree={s sep=5mm,inner sep=0, l=0}
    [QP [Spec,name=target[COPY,name=source]] [Q' [Q ] [XP [X ] [ZP ] ] ] ]
    \draw[->](source) to[out=north, in=south] (target);
\end{forest}

\caption{Syntactic (iterative) reduplication (\citealt{Travis2001,Travis2003})}
\label{psk:fig:fig1}

\end{figure}

Importantly, the mechanism of iterative reduplication developed by \cite{Travis2001,Travis2003} permits some subdomains to be copied into specifier positions. The kind of copying in question substantially differs from the copying in the ``classical'' movement since in the case of syntactic reduplication it is copying without deletion. Given the  modification patterns in \REF{psk:ex:key:11}--\REF{psk:ex:key:12}, and in particular considerable variation concerning the presence of modifiers on both nominals or only N\textsubscript{1} or only N\textsubscript{2}, Travis’s approach needs to be reconsidered: the whole \textit{n}P is copied, and modifiers can undergo PF deletion on either N\textsubscript{1} (in English) or N\textsubscript{2} (in Polish). The distribution of modifiers in NPNs could be regulated by \citeposst{Fanselow.Cavar2002} distributed deletion mechanism, but it is not to be elaborated on here.

\cite{Travis2001,Travis2003} does not take it to be a settled matter whether the Q head selects a PP as its complement, or it is lexically realised as the preposition. In the latter case, the preposition would be an overt realisation (or at least the guise) of the reduplicative head. As a result, there are two possible structures for NPNs derived via syntactic reduplication: see \figref{psk:fig:fig2} and \figref{psk:fig:fig3}.

\begin{figure}

\begin{forest}for tree={s sep=5mm,inner sep=0, l=0}
    [QP [\textit{n}P [day,name=target ] ] [Q$'$ [Q ] [PP [P [after ] ] [\textit{n}P [day,name=source] ] ] ] ]
    \draw[->](source) to[out=south west, in=south, looseness=1.3] (target);
\end{forest}

\caption{A variant of syntactic reduplication where the Q head selects a PP as its complement}
\label{psk:fig:fig2}

\end{figure}

%\ea%14
%    \label{psk:ex:key:14}
% [\textsubscript{QP [\textsubscript{nP} day ] [\textsubscript{Q’} Q %[\textsubscript{PP} after [\textsubscript{nP} day ] ] ] ]}
%\z

\begin{figure}

\begin{forest}
    [QP [\textit{n}P [day,name=target]] [Q$'$ [Q ] [\textit{n}P [day,name=source]] ] ]
    \draw[->](source) to[out=south west, in=south] (target);
\end{forest}

\caption{A variant of syntactic reduplication where the Q head is morpho-phonologically realized as a preposition in languages such as English or Polish}
\label{psk:fig:fig3}

\end{figure}

%\ea\label{psk:ex:key:15}[\textsubscript{QP [\textsubscript{nP} day ] %[\textsubscript{Q’} after [\textsubscript{nP} day ] ] ]}
%\z

The structure in \figref{psk:fig:fig2} has a somewhat un-Minimalist flavour as it is based on a head (Q) that would probably be morpho-phonologically empty in all languages. Apart from this, the mechanism involving movement of a nominal complement out of a PP in non-P-stran\-ding languages such as Polish poses another difficulty. If \citet{Abels2003} is right regarding the phasal status of P in non-P-stranding languages, then \figref{psk:fig:fig3} would involve the crossing of a phase boundary.

The configuration in \figref{psk:fig:fig3} seems to capture the facts from languages where NPNs have no preposition, as illustrated for Kazakh (Turkic) in \REF{psk:ex:key:16} (Turkish would follow the same pattern, Dilek Uygun Gokmen p.c.):\footnote{The Kazakh examples were provided by native speakers of the language who participated in comparative morphosyntax seminars I taught at the University of Lodz (Poland) 2016--2019.}

\ea\label{psk:ex:key:16}\ea \gll kunen kunge\\
      day.\textsc{abl}   day.\textsc{dat}\\
\glt       ‘day by day’
\ex\gll  elden elge\\
      country.\textsc{abl}  country.\textsc{dat}\\
\glt      ‘country by country’
\ex\gll   sureten suretke\\
      picture.\textsc{abl}  picture.\textsc{dat}\\
\glt      ‘picture after picture’\hfill (Kazakh)
\z
\z

\noindent The major theoretical disadvantage of the structure in \figref{psk:fig:fig3} is that – by allowing the copying of the content of the complement of Q into its specifier – it violates anti-locality (\citealt{Abels2003,grohmann2003prolific}): the movement is too local. In particular, \citet{Abels2003} argues against movement from the complement to the specifier of the same head.\footnote{According to an anonymous reviewer, the only solution to the problem of anti-locality in the case of NPN structures would be to treat this kind of movement as a non-syntactic operation. I leave it for further research to decide whether the original idea of syntactic reduplication in \cite{Travis2001,Travis2003} can be maintained.} This analysis can be saved by stipulating that the syntactic reduplication is distinct from the ``classical'' movement: copying without deletion – licensed by the reduplicative head – is allowed to be that local.\footnote{Another problem pointed out by an anonymous reviewer with respect to movement without deletion is that this kind of operation overgenerates. However, if we assume that this sort of movement is only triggered by the reduplicative head that has some selectional restrictions (as illustrated in \sectref{psk:sec:constraints} above), the operation becomes restricted, though obviously by stipulation.}

For languages like Kazakh or Turkish, the structure in \figref{psk:fig:fig2} would entail the presence of two empty heads: the Q head triggering reduplication, and the adposition-like case assigner heading the complement of Q, which is quite an unwelcome result. According to the structure in \figref{psk:fig:fig3}, the Q head would be morpho-phonologically realised as a preposition in languages such as English or Polish, and it would be phonologically null in languages such as Kazakh.\footnote{This needs to be corroborated by analysing the behaviour of NPNs in clauses in Kazakh or Turkish.}

As regards case assignment in Polish or Kazakh NPNs (and possibly in other languages with a rich system of morphological case), it would have to take place after the reduplication occurs. The nominal following the preposition is copied before it is assigned case by P: in Polish the case-marking of N\textsubscript{2} is determined by the preposition. This would involve post-syntactic realisation of case inflection \citep{Sigurðsson2012} or delayed movement to the appropriate position in KP as in \citet{Caha2009}. The details of case assignment are not going to be elaborated on here, however.

Based on the idea of cross-categorial symmetry between the nominal and the verbal/clausal domains, there has been a long-standing tradition of assuming the presence of an outer \textit{n}P shell headed by a light noun and serving as the complement for some other higher functional heads (cf. \citealt{Radford2000,Radford2009}; \citealt{Alexiadou.etal2007}) as a nominal counterpart of the vP projection in the clausal domain. Following this tradition, I assume that the bare nominals in NPNs are ``defective'' in the sense that they lack the DP-layer in English (and other languages with articles) and in Polish if one assumes the universality of DP (see e.g. \citealt{Progovac1998}, \citealt{Willim1998}, \citealt{Pereltsvaig2007}, \citealt{Jeong2016}). The NPN-internal nominals also lack projections hosting demonstratives and other determinative heads in both English and Polish, which I expect to be valid cross-linguistically, but it obviously remains a tentative hypothesis to be tested in the course of further research. They resemble \citeposst{Pereltsvaig2006} small nominals, as argued for in \citet{Pskit2017}. Alternatively, the ``defective''/small nominals inside NPNs can also be viewed as \textit{n}Ps in the sense of roots with a categorising \textit{n} head, as in Distributed Morphology (cf. \citealt{Halle.Marantz1993,harley1999distributed,Acquaviva2008}). Whether there are any higher functional projections dominating \textit{n}P is a questionable issue. Given the number-neutral status of N\textsubscript{1} and N\textsubscript{2}, they most probably do not include NumP, though this may seem problematic from the point of view of subject-verb agreement facts discussed in 3 below, and is perhaps even more controversial in the context of plural agreement as in \REF{psk:ex:key:8} above, reproduced in \REF{psk:ex:key:16001} below for convenience:

\ea \label{psk:ex:key:16001}  … there are millions upon millions who support your decision   …\vspace{-12pt}\\ \null\hfill(Internet)\\
\z

\noindent \citet{Acquaviva2008} argues that plurality that is inherent in nouns such as \textit{hundreds}, \textit{thousands} or \textit{millions} is encoded in the categorising \textit{n} head, making the nouns in question [ \textit{n} [ \textsc{root}] ] complexes in the spirit of Distributed Morphology. If NumP is absent, the fact that the case endings on N\textsubscript{1} and N\textsubscript{2} in Polish are for the singular results from the treatment of these number-neutral bare nominals as singular by default. The same ``singular-by-default'' explanation would have to work in the context of premodifiers of the bare nominals, if they are found licit in Polish (cf. the data in \REF{psk:ex:key:12} above), as such premodifiers necessarily agree with the head noun in terms of number, gender and case. As regards gender, the absence of the relevant functional head could be explained based on the assumption in \citet{Alexiadou.etal2007}: gender is an inherent part of the lexical entry of each noun rather than the matter of a dedicated functional head in the syntax.\footnote{As an anonymous reviewer aptly observes, this may mean that both plurality and gender are encoded in the categoriser. An alternative would be to assume that – given data such as \REF{psk:ex:key:16} – the NPN-internal nominals contain the NumP projection, which requires investigating more cross-linguistic data on NPN subjects and objects.}

If NPNs are actually QPs, it naturally follows that the properties – including the quantificational properties – of the whole NPN are determined by the Q head.

\section{The external properties of NPN subjects and objects}\label{psk:sec:sec3}

In both English and Polish, NPNs with all the prepositions in question can occur as adjuncts in typical adjunct positions in the clausal architecture. Consider the English data in \REF{psk:ex:key:17} (from \citealt{Jackendoff2008} and \citealt{Huddleston.Pullum2002}) and the Polish examples in \REF{psk:ex:key:12345}:

\ea \label{psk:ex:key:17}\ea Page for page, this is the best-looking book I’ve ever   bought.
\ex John and Bill, arm in arm, strolled through the park.
\ex We went through the garden inch by inch.
\ex She worked on it day after day.\z
\z

\ea\label{psk:ex:key:12345} \ea \gll \label{psk:ex:key:18}Szli łeb w łeb.\\
      go.\textsc{3pl.pst}  head.\textsc{sg.nom} in head.\textsc{sg.acc}\\
\glt     ‘They went/ran neck and neck.’
\ex \gll  Dzień          po     dniu            zbliżaliśmy się     do celu.\\
      day.\textsc{sg.nom} after day.\textsc{sg.loc} approach.\textsc{1pl.pst} to goal\\
\glt     ‘Day after day we were approaching our goal.’
\ex\gll  Wertował                  książkę kartka           po     kartce.\\
      leaf.\textsc{3sg.pst}.through book page.\textsc{sg.nom} after page.\textsc{sg.loc}\\
\glt     ‘He leafed through a book page after page.’\hfill\citep[249]{Dobaczewski2018}\z
\z

\noindent English NPNs can also be DP-internal premodifiers \REF{psk:ex:key:19a}, and those with \textit{after} and \textit{upon} can function as complements of prepositions \REF{psk:ex:key:19b}  or possessive determiners \REF{psk:ex:key:19c} \citep[19]{Jackendoff2008}, though such patterns are not available in Polish:

\ea \label{psk:ex:key:19}  \ea\label{psk:ex:key:19a} Your day-to-day progress is astounding.
\ex\label{psk:ex:key:19b} We looked for dog after dog.
\ex\label{psk:ex:key:19c} Student after student’s parents objected.
\z\z

\noindent A selected set of NPNs – with \textit{after} and \textit{upon} in English and with \textit{po} and \textit{za} in Polish – can become clausal subjects or objects.

\ea \label{psk:ex:key:20}  \ea Day after day passed.\\
\ex I drank cup after cup (of coffee).\\
\z \z

\ea \label{psk:ex:key:21}  \ea \gll Mijał dzień za dniem.\\
      pass.\textsc{3sg.pst} day.\textsc{sg.nom} after day.\textsc{sg.ins}\\
\glt      ‘Day after day passed.’
\ex \gll   Czytał wiersz za wierszem.\\
      read.\textsc{3sg.pst} poem.\textsc{sg.acc} after poem.\textsc{sg.ins}\\
\glt      ‘He read poem after poem.’
\ex \gll   Mówiła studentowi za studentem …\\
      tell.\textsc{3sg.pst} student.\textsc{sg.dat} after student.\textsc{sg.ins}\\
\glt      ‘She told student after student …’\z
\z

\noindent An interesting subject-verb agreement pattern emerges from the data in \REF{psk:ex:key:20}--\REF{psk:ex:key:21}: in both English and Polish the verb is invariably singular in spite of the plural semantics of the whole NPN, which is corroborated by \REF{psk:ex:key:22} below:

\ea \label{psk:ex:key:22}  \ea[]{Day after day passes …}
\ex[*]{Day after day pass …}
\ex[]{\gll Mija dzień za dniem.\\
      pass.\textsc{3sg.prs} day.\textsc{sg.nom} after day.\textsc{sg.ins}\\
\glt     ‘Day after day passes.’}
\ex[*]{\gll  Mijają        dzień            za     dniem\\
      pass.\textsc{3pl.prs} day.\textsc{sg.nom} after day.\textsc{sg.ins}\\
\glt      Intended: ‘Day after day passes.’}
\z
\z

\noindent Given the derivation of NPNs as QPs via syntactic (iterative) reduplication, I assume – as suggested in \sectref{psk:sec:sec-2-4} above – that the quantificational properties of NPNs are determined by the Q head. The agreement data prove that subject NPNs are syntactically singular. In addition, Polish NPN subjects agree with the verb also in terms of grammatical gender; see \REF{psk:ex:23:a} vs. \REF{psk:ex:23:b}:

\ea \label{psk:ex:23}\ea \gll Mijał dzień za dniem.\\
      pass.\textsc{3sg.m.pst} day.\textsc{sg.m.nom} after day.\textsc{sg.m.ins}\\
\glt      ‘Day after day passed.’\label{psk:ex:23:a}
\ex \gll  Mijała noc za    nocą.\\
      pass.\textsc{3sg.f.pst} night.\textsc{sg.f.nom} after night.\textsc{sg.f.ins}\\
\glt      ‘Night after night passed.’\label{psk:ex:23:b}\z
\z

\noindent The data in \REF{psk:ex:key:22} and \REF{psk:ex:23} suggest that the relevant agreement relation is established in one of the two ways: either the T head may look into the features of N\textsubscript{1} or the feature valuation takes place between T and Q, with the Q head inheriting the phi-features of N\textsubscript{1}.

Whenever NPNs are subjects or objects, they only occur with imperfective verbs in Polish as in \REF{psk:ex:key:24}. While this is not morphologically marked in English, English clauses with NPN subjects or objects would only allow imperfective interpretation too. Note that morphologically perfective verbs in Polish are fine with non-NPN plural objects \REF{psk:ex:key:24c}:

\ea \label{psk:ex:key:24} \ea[]{\gll  Strzelał  bramkę  za    bramką.\\
      score.\textsc{3sg.m.pst.ipfv} goal      after goal\\
\glt       ‘He scored goal after goal.’}
\ex[*]{\gll   Strzelił                     bramkę za    bramką.\\
      score.\textsc{3sg.m.pst.pfv} goal     after goal\\
\glt       Literally: ‘He has scored goal after goal.’}
\ex[]{\gll   Strzelił wiele bramek.\\
      score.\textsc{3sg.m.pst.pfv} a.lot.of goals\\
\glt       ‘He has scored a lot of goals.’\label{psk:ex:key:24c}}
\z
\z

\noindent One possible – though stipulative – account of the co-occurrence of imperfective verbs with NPN objects and subjects is based on the mechanism of valuation of the relevant feature of the Asp head in the extended verbal projection and the Q head of the NPN. An alternative is to relegate the issue to the level of LF interface as this property of NPN subjects and objects is shared with NPN adjuncts. Indeed, irrespective of the grammatical function of NPNs, their plural semantics (iteration of entities or events) seems to match the morphological manifestation of the outer (grammatical) aspect in the verbal domain. The lack of such morphological aspectual marking in English points to the semantic licensing of the phenomenon.

\section{Conclusion}\label{psk:sec:sec4}

The aim of the paper was to discuss the properties of subject and object NPNs in the light of the internal characteristics of NPN structures derived via a revised version of syntactic reduplication, originally proposed in \cite{Travis2001,Travis2003}.

The investigation is preliminary in nature and awaits corroboration by further research on NPNs in English, Polish and beyond.

The singular syntax of NPNs in both languages is reflected by the singular subject-verb agreement, whereas the plural semantics of NPNs corresponds to the imperfective characteristics of the verb with all types of NPNs.

The modification data discussed in \sectref{psk:sec:sec2-3} above suggest the following hypothesis with possible typological implications. While they encode the plurality of entities or events, NPNs are structures that are formally ``abbreviatory'': the mechanism of syntactic (iterative) reduplication yields expressions with minimal structure. The NPN is a structure with as little material (both in terms of ``surface'' morpho-phonological material and in terms of the articulation of the underlying syntactic structure) as possible. Ideally, there are two bare nominals ``linked'' by a preposition. Hence, in a language such as Polish, with rich nominal-internal agreement between the head noun and its modifiers, the amount of the morpho-phonological material resulting from establishing the agreement makes it too ``heavy'' for the Q head to accept modification within the NPN. But this remains a hypothesis to be tested empirically in other languages, especially beyond Germanic and Slavic and indeed beyond Indo-European, and also to be further pursued on theoretical grounds.

If the internal and external properties of NPNs discussed above turn out to be cross-linguistically valid, as expected based on fragmentary data from other languages, the lines of reasoning suggested above may gain further empirical support.

\section*{Abbreviations}

\begin{tabularx}{.5\textwidth}{@{}lX@{}}
\textsc{3} & third person\\
\textsc{abl} & ablative\\
\textsc{acc} & accusative\\
\textsc{dat} & dative\\
\textsc{f} & feminine\\
\textsc{ins} & instrumental\\
\textsc{ipfv} & imperfective\\
\textsc{loc} & locative\\
\end{tabularx}%
\begin{tabularx}{.5\textwidth}{@{}lX@{}}
      \textsc{m} & masculine\\
      \textsc{nom} & nominative\\
      \textsc{pfv} & perfective\\
      \textsc{pl} & plural\\
      \textsc{prs} & present\\
      \textsc{pst} & past\\
      \textsc{sg} & singular\\
      &\\
\end{tabularx}





\section*{Acknowledgements}

I would like to thank the participants of SinFonIJA 8 (2015), Poznań Linguistic Meeting (PLM 2017, Poznań), Linguistic Beyond and Within (LinBaW2017, Lublin), SinFonIJA 12 (Brno, 2019) for valuable feedback on some of the portions of the material presented above. I am particularly indebted to Przemysław Tajsner, Andreas Blümel, Piotr Cegłowski, Anna Bondaruk, and Hana Filip for insightful comments and suggestions that inspired my thinking about NPN structures. I would like to thank two anonymous reviewers whose remarks helped me improve the paper. The reviewers also provided valuable suggestions concerning further research on NPN structures. Needless to say, any remaining errors or misconceptions are my own responsibility.

{\sloppy\printbibliography[heading=subbibliography,notkeyword=this]}

\end{document}
