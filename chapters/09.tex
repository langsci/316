\documentclass[output=paper]{langscibook} 
\ChapterDOI{10.5281/zenodo.5082466}

\author{Magdalena Roszkowski\affiliation{Central European University}}
\title{Conjunction particles and collective predication}  
\abstract{This paper is concerned with Polish $e$-type conjunctions that involve conjunction particles and their semantic properties. The possible interpretations of such conjunctions and the restrictions on the type of predicate they may combine with do not only pose problems for standard assumptions about distributivity and collectivity but also grant insight into the structure of plural predicates in general. The discussion thereof will bear on the observations that have been made with respect to the behavior of the determiner \textit{all} in English (cf. \citealt{Dowty:1987}). Moreover, additional requirements on the context that arise in combination with collective predicates will be taken to suggest an analysis of conjunction particles in terms of focus particles ranging over subpluralities.

\keywords{plural predication, conjunction particles, collectivity, distributivity}}


\begin{document}
\SetupAffiliations{mark style=none}
\maketitle

\section{Introduction}\label{ros:sec:1}


Polish exhibits, in addition to a ``simple'' conjunction strategy which may be used to conjoin two or more individual-denoting expressions \REF{ros:ex1}, a ``marked'' conjunction strategy in which the marker \textit{i} occurs before each conjunct \REF{ros:ex2}.

\ea\label{ros:ex1} \gll Ewa \minsp{(} i) Karol i Iza palili w kuchni. \\
Ewa.\textsc{nom} {} and Karol.\textsc{nom} and Iza.\textsc{nom} smoke.\textsc{pst.3pl} in kitchen.\textsc{loc} \\
\glt `Ewa, Karol and Iza were smoking in the kitchen.'
\ex\label{ros:ex2} \gll I Ewa i Karol i Iza palili w kuchni. \\
and Ewa.\textsc{nom} and Karol.\textsc{nom} and Iza.\textsc{nom} smoke.\textsc{pst.3pl} in kitchen.\textsc{loc} \\
\glt `Ewa as well as Karol as well as Iza were smoking in the kitchen.'
\z 

\noindent Structurally similar iterative $e$-type conjunction strategies which involve conjunction particles, i.e.~particles that occur on each conjunct, have been attested in several other languages, e.g.~Turkish \REF{ros:tr}, Bosnian/Croatian/Montenegrin\slash Ser\-bi\-an (BCMS) \REF{ros:ser}, Japanese \REF{ros:jap} and Hungarian \REF{ros:hu} and are usually associated with distributivity (see \citealt{Flor:2017Cross, Mitrovic:2014,Szabolcsi:2015}).

\ea \label{ros:cross}
\ea \label{ros:pl} [i A i B i C]  \hfill Polish
\ex \label{ros:tr} [A dA (ve) B dA (ve) C dA] \hfill Turkish
\ex \label{ros:ser} [i A i B i C] \hfill BCMS
\ex \label{ros:jap} [A-mo B-mo C-mo] \hfill Japanese
\ex \label{ros:hu} [A is (és) B  is (és) C is] \hfill Hungarian
\z\z

% \ea \label{ros:cross}
% \ea \label{ros:pl} [i A i B i C]  \phantom{.}\hfill Polish
% \ex \label{ros:tr} [A dA (ve) B dA (ve) C dA] \phantom{.}\hfill Turkish
% \ex \label{ros:ser} [i A i B i C] \phantom{.}\hfill BCMS
% \ex \label{ros:jap} [A-mo B-mo C-mo] \phantom{.}\hfill Japanese
% \ex \label{ros:hu} [A is (és) B  is (és) C is] \phantom{.}\hfill Hungarian
% \z\z

\noindent Polish seems to pattern with these languages insofar as conjunction particles enforce distributive interpretations in sentences in which an individual conjunction combines with an ambiguous predicate like \textit{earn 100 euros}. While a sentence that contains a simple conjunction like \REF{ros:amb1} allows for both a distributive and a non-distributive interpretation, and thus may be judged true in \textit{Situation 1} and in \textit{Situation 2}, sentences containing the marked conjunction only allow for a distributive interpretation, i.e.~\REF{ros:amb2} is only true in \textit{Situation 1}.\footnote{Acceptability judgements in this paper reflect my own intuitions as well as judgements provided by five native speakers of Polish via an informal questionnaire.}

\ea\label{ros:amb1} \gll Ewa \minsp{(} i) Karol i Iza zarobili 100 euro. \\
Ewa.\textsc{nom} {} and Karol.\textsc{nom} and Iza.\textsc{nom} earn.\textsc{pst.3pl} 100 euros \\
\glt `Ewa, Karol and Iza earned 100 euros.'
\ex\label{ros:amb2} \gll I Ewa i Karol i Iza zarobili 100 euro. \\
and Ewa.\textsc{nom} and Karol.\textsc{nom} and Iza.\textsc{nom} earn.\textsc{pst.3pl} 100 euros \\
\glt `Ewa, Karol and Iza earned 100 euros each.'
\ex
\ea \textit{Situation 1:} Ewa earned 100 euros. Karol earned 100 euros. Iza earned 100 euros.
\ex \textit{Situation 2:} Ewa earned 30 euros. Karol earned 10 euros. Iza earned 60 euros.
\z\z

\noindent This would suggest that marked structures are always distributive; however, as illustrated in \REF{ros:coll}, in Polish they may also combine with collective predicates.\largerpage

\ea\label{ros:coll} \gll I Ewa i Karol i Iza spotkali się wczoraj o 11. \\
and Ewa.\textsc{nom} and Karol.\textsc{nom} and Iza.\textsc{nom} meet.\textsc{pst.3pl} \textsc{refl} yesterday at 11 \\
\glt `Ewa, Karol and Iza met yesterday at 11.'
\z 

\noindent This pattern on the one hand challenges some common assumptions about how distributive, cumulative and collective interpretations are derived and related, but may on the other hand, as will be shown below, also provide new insights on the semantics of plural predicates in general (cf. \citealt{Dowty:1987, Schein:1993, Schein:2017, Winter:2002, Hackl:2002, Champollion:2010} a.o.).


\section{Theories of conjunction}\label{ros:sec:2}

The dichotomy observed in Polish is not straightforwardly accounted for by most semantic theories which are concerned with distributive and non-distributive interpretations of \textit{e}-type conjunctions (e.g.~\citealt{Link:1983, Partee:1983, Landman:1989, Krifka:1990, Schein:1993, Schein:2017, Schwarzschild:1996}).\footnote{The following discussion focuses only on analyses that are relevant for the phenomenon at hand, since it is beyond the scope of the present paper to provide an exhaustive overview of theories of conjunction. I thank a reviewer for asking to clarify the selective view in this section.} For instance, \citet{Link:1983}, in order to capture the denotations of plural expressions such as \textit{the girls} or \textit{Mary, Sue and Ann}, assumes that $D_{e}$ is closed under sum ($\oplus$). This allows us to distinguish and model three types of predicates: collective predicates like \textit{meet} primitively denote properties of pluralities. Distributive predicates like \textit{smoke} -- which obligatorily give rise to distributive entailments -- must be affixed or lexically supplemented with a distributivity operator and are only true of atomic individuals. The distributive interpretation of ambiguous predicates like \textit{earn 100 euros}, which may receive a distributive and a non-distributive (i.e. collective or cumulative) interpretation, results from affixing the VP with \textsc{D\textsubscript{pred}}, which requires the predicate to hold of each atomic individual (cf. \citealt{Link:1987} a.o.).

\ea\label{ros:d1}
\sib{\textsc{D\textsubscript{pred}}} = $\lambda P_{\langle e,t \rangle}. \lambda x_{e}. \forall y \le_{\cnst{at}} x.P(y) = 1$ 
\z

\noindent In principle, one could assume that \textsc{D\textsubscript{pred}} is optional in sentences like \REF{ros:amb1}, which contain the simple strategy and allow for both interpretations, whereas it is obligatory in sentences like \REF{ros:amb2}, forcing a distributive interpretation. This would make the correct predictions for sentences with ambiguous predicates, but collective interpretations of sentences containing the marked strategy would remain unexplained. On the other hand, the morphological properties of the marked strategy suggest that the lack of a non-distributive interpretation should be accounted for in the DP semantics.\footnote{Distributivity of ambiguous sentences like \REF{ros:amb1} may also be enforced by adding the marker \textit{po} before the measure phrase. However, to take \textit{po} to be the overt realization of \textsc{D\textsubscript{pred}} seems problematic, especially since the marker has been shown to distribute not only over atomic individuals but also over spatial and temporal intervals \citep{Przepiorkowski:2014, Champollion:2016Overt}.} For instance, one could assume that the distributive interpretation is due to an operator like \REF{ros:dist2}, which applies to the subject DP.

\ea\label{ros:dist2}
\sib{\textsc{D\textsubscript{conj}}} = $\lambda x_{e}. \lambda P_{\langle e,t \rangle}.\forall y \le_{\cnst{at}} x.P(y) = 1$ 
\z

\begin{sloppypar}
\noindent However, the fact that the marked strategy is compatible with collective predicates is also inconsistent with this assumption. As introduced above, conjunctions that involve conjunction particles exist in several other, typologically diverse languages (see \citealt{Mitrovic:2014, Szabolcsi:2015, Flor:2017Cross}) and recent accounts propose analyzing them in terms of focus \citep{Arsenijevic:2011}, type-shifts \citep{Mitrovic:2014} or postsuppositions \citep{Szabolcsi:2015}. Without further assumptions, these analyses predict that such constructions will receive a distributive interpretation in all environments and do not consider the possibility of collective interpretations. Though it is an open empirical question whether conjunction particles can be analyzed cross-linguistically in a uniform way or whether we find distributional and interpretational differences across languages, the behavior of conjunction particles in Polish cannot be captured by existing proposals.
\end{sloppypar}
 
A slightly different distinction, which is proposed in \citet{Landman:1989} (see also \citealt{Link:1983}), is to enrich the ontology with intransparent groups which are formed via a group forming operation $\uparrow$ that maps sums of individuals onto atomic group individuals. 

\ea $\uparrow$ is a one-one function from SUM into ATOM such that:
\ea $ \forall d \in$ SUM-IND: $\uparrow(d) \in$ GROUP
\ex $ \forall d \in$ IND: $\uparrow(d) = d$
\z\ex $\downarrow$ is a function from ATOM onto SUM such that:
\ea $ \forall d \in$ SUM: $\downarrow(\uparrow(d)) = d$
\ex $ \forall d \in$ IND: $\downarrow(d) = d$
\z\z

\noindent The operation $\uparrow$ maps sums of individuals to group individuals that count as atomic and the operation $\downarrow$ maps any group to the sum of its members, which is a non-atomic individual unless the group has only one member. For instance, in addition to the sum $m \oplus s \oplus a$, there is an individual $\uparrow$($m \oplus s \oplus a$), which counts as atomic and can itself be part of a sum. 

\ea
\ea \sib{Mary \textsc{coord} [Sue \textsc{coord} Ann]} = $m \oplus s \oplus a$
\ex \sib{$\uparrow$ [Mary \textsc{coord} [Sue \textsc{coord} Ann]]} = $\uparrow(m \oplus s \oplus a)$
\z\z

\noindent While distributive predicates are primitively true of singular individuals, collective predicates are true of groups and ambiguous predicates of both singular individuals and groups. Non-distributive  interpretations involve applying a collective predicate or an ambiguous predicate to an atomic group individual (not to a sum). Ambiguous predicates distribute down to the parts of a sum, but not to the parts of a group, since the group counts as an atomic individual. In this way it is also possible to modulate partly distributive readings, e.g. the reading of \REF{ros:inter} on which the predicate \textit{earn 100 euros} distributes down to the atomic singular individual Mary on the one hand, and to the group individual consisting of Sue and Ann on the other hand. 

\ea
\ea\label{ros:inter} Mary and Sue and Ann earned 100 euros.
\ex\label{ros:land-d} [[Mary \textsc{coord} [Sue \textsc{coord} Ann]] [\textsc{D\textsubscript{pred}} [earned 100 euros]]] 
\ex \sib{Mary \textsc{coord} $\uparrow$ [Sue \textsc{coord} Ann]]} = $m\ \oplus \uparrow(s \oplus a)$

\ex \sib{\textsc{D\textsubscript{pred}} [earned 100 euros]} = $\lambda x_{e}. \forall y \le_{\cnst{at}} x.$\sib{earned 100 euros}$(y) = 1$

\ex \sib{\REF{ros:land-d}} = 1 iff $\forall y \le_{\cnst{at}} m\ \oplus \uparrow(s \oplus a).$\sib{earned 100 euros}$(y) = 1$
\z\z

\noindent Both strategies in Polish allow for such interpretations, i.e.~\REF{ros:un-part} and \REF{ros:mark-part} can be used to describe the mixed scenario in \REF{ros:sc3}.

\ea
\ea\label{ros:un-part} \gll Ewa || i Karol | i Iza zarobili 100 euro. \\
Ewa.\textsc{nom} {} and Karol.\textsc{nom} {} and Iza.\textsc{nom} earn.\textsc{pst.3pl} 100 euros \\
\glt `Ewa and Karol and Iza earned 100 euros.' 

\ex\label{ros:mark-part} \gll I Ewa || i Karol | i Iza zarobili 100 euro. \\
and Ewa.\textsc{nom} {} and Karol.\textsc{nom} {} and Iza.\textsc{nom} earn.\textsc{pst.3pl} 100 euros \\
\glt `Ewa and Karol and Iza earned 100 euros.' 
\z\ex\label{ros:sc3}
\textit{Situation 3}: Ewa earned 100 euros. Karol earned 50 euros. Iza earned 50 euros.
\z

\noindent Like in English, this kind of interpretation for \REF{ros:un-part} becomes available when the first coordinator is realized overtly.\footnote{Both strategies also allow for the introduction of further conjuncts, whereby additional group readings potentially become available.} Furthermore, there is a prosodic boundary after the first conjunct in \REF{ros:un-part} and in \REF{ros:mark-part} (cf. \citealt{Winter:2002, Wagner:2010}).\footnote{Prosodic boundaries are indicated by the pipe symbol with the number of pipes marking their relative strength (cf. \citealt{Wagner:2010}).} So it seems that groups or equivalent higher-order pluralities are needed anyway for the analysis of all possible interpretations of both coordination strategies in Polish. According to Landman's account, only group-denoting expressions may combine with collective predicates, and, in general, these expressions should allow for non-distributive interpretations when combined with ambiguous predicates. But this is, of course, not what we find in Polish when looking at the marked strategy, as the examples above illustrated. The question then is why, given that the marked conjunction can be combined with collective predicates, a partly distributive interpretation that involves groups is available for \REF{ros:mark-part}, but the group interpretation for the entire conjunction is generally excluded.  



\section{Compatibility with collective predicates}\label{ros:sec:3}

A closer inspection reveals that only a subclass of collective predicates is compatible with conjunction particles. This class includes predicates like \textit{meet}, \textit{hold hands} and \textit{be similar} (corresponding to \textit{gather}-type predicates in \citealt{Champollion:2010}, set predicates in \citealt{Winter:2002} and essentially plural predicates in \citealt{Hackl:2002}). 

\ea\label{ros:meet} \gll  I Ewa i Karol i Iza spotkali się wczoraj. \\
and Ewa and Karol and Iza  met \textsc{refl} yesterday \\
\glt `Ewa, Karol and Iza met yesterday.' 
\ex\label{ros:hand} \gll I Ewa i Karol i Iza trzymali się za ręce. \\
and Ewa and Karol and Iza held \textsc{refl} \textsc{prep} hands \\
\glt `Ewa, Karol and Iza were holding hands.' 
\ex\label{ros:similar} \gll  I Ewa i Karol i Iza są podobni do siebie. \\
and Ewa and Karol and Iza are similar to \textsc{refl} \\
\glt `Ewa, Karol and Iza are similar to each other.' 
\z

\noindent To a certain degree, \textit{gather}-type predicates allow for distributive subentailments about the members of their plural subject (\citealt{Dowty:1987, Winter:2002, Hackl:2002, Champollion:2010} a.o.). For instance, if Ewa, Karol and Iza met, then one may conclude that it is the case that Ewa and Karol, Karol and Iza, and Ewa and Iza met. Other collective predicates, like e.g. \textit{be numerous}, \textit{be a couple} and \textit{constitute a majority}, do not allow for such entailments. This class (roughly corresponding to pure cardinality predicates in \citealt{Dowty:1987}{,} \textit{numerous}-type predicates in \citealt{Champollion:2010} and genuine collective predicates in \citealt{Hackl:2002}) yields unacceptable sentences when combined with the marked conjunction.

{\judgewidth{\#}%
\ea[\#]{\gll I Ewa i Karol i Iza byli liczni. \\
and Ewa and Karol and Iza were numerous \\
\glt }\label{ros:num}
\ex[\#]{\gll I Ewa i Karol są parą. \\
and Ewa and Karol are couple \\
\glt Intended: `Ewa and Karol are a couple.'}\label{ros:couple} 
\ex[\#]{\gll I Ewa i Karol i Iza stanowili większość. \\
and Ewa and Karol and Iza  constituted majority \\
\glt Intended: `Ewa, Karol and Iza constituted the majority.'}\label{ros:majority} 
\z}

\noindent The former class of predicates is compatible with the plural determiner \textit{wszyscy} `all' \REF{ros:all1}, whereas the latter usually is not \REF{ros:all2} (cf. \citealt{Dowty:1987}).

{\judgewidth{\#}\ea[]{\label{ros:all1} \gll  Wszyscy studenci spotkali się / trzymali się za ręce / są podobni do siebie. \\
all students  met \textsc{refl} {} held \textsc{refl} \textsc{prep} hands {} are similar to \textsc{refl}  \\
\glt `All students met / were holding hands / are similar to each other.'}
\ex[\#]{\label{ros:all2} \gll Wszyscy studenci byli liczni \text{/} są parą / stanowili większość. \\
all students were numerous {} are couple {} constituted majority \\
\glt}
\z}

\noindent But \textit{all} can -- in contrast to the marked conjunction -- receive a non-distributive interpretation when combined with an ambiguous predicate as in \REF{ros:all3}.

\ea\label{ros:all3} \gll Wszyscy studenci zarobili 100 euro. \\
all students earned 100 euros \\
\glt `All students earned 100 euros.' (distributive or non-distributive) 
\z


\noindent Thus, the status of the marked conjunction is ambivalent: on the one hand, this strategy and the determiner \textit{all} are alike in that they are compatible only with \textit{gather}-type predicates and stress the fact that every member of the plural subject takes part in the action expressed by the predicate. They also share the property of being distributive with inherently distributive predicates like \textit{smoke}, but being collective with collective predicates like \textit{meet} (cf.~\citealt{Dowty:1987} for a discussion on the status of \textit{all}). On the other hand, their behavior differs with respect to ambiguous predicates -- in such environments the marked conjunction only allows for distributive interpretations, whereas \textit{all} is also compatible with non-distributive ones. There, the marked strategy seems to pattern with the determiner \textit{every} in that it forces a distributive reading. 


\section{Further restrictions}\label{ros:sec:4}

In addition to the collective predicate type that matters for conjunction particles, further limitations may be observed with respect to the possible situations they may appear in. Whereas \REF{ros:context-ex1} is felicitous in \textit{Situation 1}, without any further assumptions it does not fit a situation like \textit{Situation 2}.

\eanoraggedright
\eanoraggedright\label{ros:c1} \textit{Situation 1:} Ewa, Karol and Iza are organizing a party together. They have tried to set up meetings once a week, but it has never worked out for all of them. Two weeks ago, only Karol and Iza met. Last week, only Ewa and Iza met.
\ex \textit{Situation 2:} Ewa, Karol and Iza are organizing a party together. They have tried to set up a meetings once a week and, surprisingly, it has always worked out for all of them.
\z\ex\label{ros:context-ex1} \gll Wczoraj i Ewa i Karol i Iza spotkali się. \\
yesterday and Ewa and Karol and Iza  met \textsc{refl} \\
\glt `Yesterday Ewa, Karol and Iza met.' 
\z

\noindent Intuitively, \REF{ros:context-ex1} means `not only Ewa and Karol, but also Iza met' and the situation in \REF{ros:c1} suggests that a meeting in which all of them take part was unexpected in a way. Indeed, such sentences even improve when the quantifier \textit{wszyscy} `everybody' is introduced as in \REF{ros:wsz}.\footnote{I would like to thank an anonymous reviewer for pointing this out to me.}

\ea\label{ros:wsz} \gll Wczoraj wszyscy, i Ewa i Karol i Iza, się spotkali. \\
yesterday everybody and Ewa and Karol and Iza \textsc{refl}  met \\
\glt `Yesterday everyone, Ewa, Karol and Iza, met.' 
\z


\noindent This relates to the requirement on the number of individuals involved: a sentence that contains only two conjuncts seems to be not interpretable at all \REF{ros:context-ex2}. 

\ea[?]{\label{ros:context-ex2} \gll I Ewa i Karol spotkali się. \\
and Ewa and Karol met \textsc{refl} \\
\glt Intended: `Ewa and Karol met.'}
\z

\noindent Informally speaking \REF{ros:context-ex2} should mean something like `not only Ewa, but also Karol met', which is odd for several reasons. Hence, conjunction particles may not only enforce that a predicate holds of each atomic individual as in sentences with ambiguous predicates, with collective predicates they also seem to emphasize that the predicate holds of each member of the subject plurality, but only in cases where the number of individuals is greater than two.


\section{Reciprocal predicates}\label{ros:sec:5}

A theory of conjunction particles thus relies on an analysis of collective predicates which allows us to account for their occurrence in such environments. Following \citet{Hackl:2002}, I therefore propose to treat \textit{gather}-type predicates in Polish as inherently reciprocal predicates, i.e.~containing a silent \textit{each other}, and to derive them from reflexive predicates bearing a non-identity presupposition. This way, the sentence below is true if each individual stands in the relation expressed by the predicate to another individual that is part of the subject plurality.

\eanoraggedright\label{ros:hackl1}
\sib{\text{Ewa, Karol and Iza met}}${ }
= 1 $ iff for each individual that is part of the plural individual Ewa, Karol and Iza there is at least one other individual in Ewa, Karol and Iza who stands in the \textit{meet with each other} relation to him or her
\z

\noindent Though it is an open empirical question whether these truth-conditions might be too weak and further (pragmatic) strengthening is needed, interestingly, most (if not all) collective predicates of the \textit{gather}-sort in Polish do include a reflexive (\ref{ros:meet}--\ref{ros:similar}). This could be just the overt realization of the assumed covert reciprocal, which in languages like English is not spelled out.\footnote{It is not clear to what extent alternative analyses, for instance in terms of apposition to a silent plural pronoun (cf. \citealt{Dikken:2001, Citko:2004}), as has been suggested by a reviewer, could account in the same way for the occurrence of reflexives.} What may be proposed for such predicates is that, in contrast to \textit{numerous}-type predicates, which seem to require groups as their arguments, they only can be satisfied by pluralities, i.e~sums, and denote a relation between non-identical individual parts of their subject plurality (following \citealt{Hackl:2002}, also \citealt{Krifka:1986, Sternefeld:1998, Beck:1999, Beck:2001}). The function of the conjunction particles in such a construction is then to introduce focus alternatives (cf. \citealt{Rooth:1992}). The requirement on number of conjuncts suggests that these have to include alternatives which can be arguments of a \textit{gather}-type predicate, i.e.~pluralities. In consequence, it is predicted that sentences like \REF{ros:context-ex2} will not be felicitous since they do not allow for deriving the ``right'' sort of alternatives, whereas a sentence that contains three conjuncts like \REF{ros:context-ex1} allows for alternatives that include subpluralities such as \sib{\text{Ewa and Karol}} and \sib{\text{Karol and Iza}}.  

\section{Conclusion}\label{ros:sec:6}

A close examination of the Polish data has shown that Polish conjunction particles force distributive interpretations with respect to ambiguous predicates, but allow for collective interpretations with \textit{gather}-type predicates whereby their presence in collective contexts requires the number of conjuncts to be greater than two and the conjunction of them to be ``unexpected''. I have argued that the ambivalent behavior of conjunction particles can be best understood if a distinction is made between cumulative, genuine collective predicates and plural collective predicates (\citealt{Dowty:1987, Winter:2002, Hackl:2002, Champollion:2010}), plural collectives are treated in terms of reciprocal predicates, and conjunction particles are analyzed in terms of focus particles ranging over subpluralities when combined with plural collectives. This provides further evidence that cumulative and collective interpretations have to be kept apart and the class of collective predicates is indeed heterogenous. Open questions remain whether the behavior of Polish conjunction particles parallels the behavior of such particles in other languages, i.e.~whether conjunction particles may be analyzed in a uniform way across languages, and if not, to what extent the patterns diverge from each other.

\section*{Abbreviations}
% \begin{tabularx}{1\textwidth}{lQ}
% \end{tabularx}

\begin{tabularx}{.5\textwidth}{@{}lX@{}}
    % BCMS & Bosnian/Croatian/Mon\-tenegrin/Serbian \\
    % DP & determiner phrase \\
    \textsc{loc} & locative\\
    \textsc{nom} & nominative \\
    \textsc{pl} & plural \\
        \end{tabularx}%
    \begin{tabularx}{.5\textwidth}{@{}lX@{}}
        \textsc{prep} & preposition \\
        \textsc{pst} & past tense \\
        \textsc{refl} & reflexive \\
        % VP & verb phrase \\
\end{tabularx}
    

\section*{Acknowledgements}

I thank our consultants as well as Mojmír Dočekal, Jovana Gajić, Nina Haslinger, Eva Rosina, Viola Schmitt, Marcin Wągiel and Kazuko Yatsushiro for helpful comments and discussion. I would also like to thank two anonymous reviewers for their detailed comments. This research was funded by the Austrian Science Fund (FWF), project Projekt: P 29240-G23, ``Conjunction and disjunction from a typological perspective". 

{\sloppy\printbibliography[heading=subbibliography,notkeyword=this]}

\end{document}
