\documentclass[output=paper]{langscibook} 

\author{Brigitta R. Schvarcz \affiliation{Bar-Ilan University; Afeka College of Engineering} and  Borbála Nemes\affiliation{Babeș-Bolyai University}}
\title[Classifiers make a difference]{Classifiers make a difference: Kind interpretation and plurality in Hungarian}  
\abstract{This paper provides an analysis of Hungarian sortal classifiers, shedding light on the complex interplay between classifiers, plurality and kind interpretation in the language. We build on \citeposst{schvarcz-rothstein-17} approach to the mass/count distinction, providing further evidence for noun flexibility. We show that Num+N and Num+CL+N constructions have different interpretations; in particular, kind interpretation tells the two apart. We provide evidence against plural-as-a-classifier \citep{dekany-11} and number-neutrality \citep{erbach-etal-19} views and argue that classifier optionality can be accounted for by the predictions the Nominal Mapping Parameter \citep{chierchia-98b} makes with respect to bare singular nouns. We claim that Hungarian nominals are born as kind-denoting expressions which then can undergo a kind-to-predicate shift explicitly triggered by a sortal individuating classifier. We analyze classifiers in Hungarian as functional operators on kinds of type $\stb{k,\stb{e,t}}$, which apply to kind denoting terms generating instantiations of that kind. 

\keywords{classifier optionality, plurality, noun flexibility, bare nominal denotation, kind interpretation, Hungarian}}

\begin{document}
\maketitle

\section{Introduction}

Why does a numeral expression allow sortal classifiers in a mass/count language? It has been widely assumed that classifiers serve as mediating elements between numerals and nouns and perform an individuating or portioning out function, allowing mass-denoting nouns to be modified by numerals.  Classifiers are obligatorily used in classifier languages since all nouns in these languages have mass denotations. In mass/count languages, on the other hand, count nouns can be directly modified by numericals. If a language exhibits wide mass/count phenomena, we do not expect it to have a functional category of classifiers. Hungarian, however, seems to contradict this paradigm.

Even though Hungarian has been categorized as a mass/count language  \linebreak\citep{schvarcz-14, schvarcz-rothstein-17}, counting in this language allows an apparently optional classifier: numerical constructions involving a notionally count noun can be realised with a construction of direct modification by a numerical (henceforth NUM+N) \REF{schv-nem:ex:1a}, as well as with a construction involving a sortal classifier (henceforth NUM+CL+N) \REF{schv-nem:ex:1b}. 

\ea \label{schv-nem:ex:1}
\ea \label{schv-nem:ex:1a}
\gll három  újság\\  
     three newspaper\\ 
\glt `three newspapers'
\ex \label{schv-nem:ex:1b}
\gll három darab újság\\
    three \textsc{cl\textsubscript{general}} newspaper\\
\glt `three newspapers'
\z
\z

\noindent As examples \REF{schv-nem:ex:1a} and \REF{schv-nem:ex:1b} illustrate, numerals in Hungarian combine with singular nouns, despite the existence of a genuine plural marker \citep{schvarcz-rothstein-17}.  In addition, the language exhibits unique bare nominal phenomena. This combination of properties poses interesting questions about the category of number. There is a complex interplay between the various grammatical devices linked to the cognitive notion of number, including numerals, classifiers, plural-marking and bare noun denotations. Investigating the category of classifiers can help us gain a better understanding of the function performed by the above-mentioned devices as well as of the category of number in Hungarian and beyond.

The aim of this paper is to provide an explanation for the optional use of sortal classifiers in Hungarian, with special focus on the general classifier \textit{darab}. Relying on novel linguistic data, we provide evidence that the presence of a classifier inside a numerical construction restricts the interpretation of the phrase: while NUM+N can have a plurality of kinds, of sub-kinds and of  individuals reading, NUM+CL+N can only refer to a set of plural individuals. This interpretational difference raises questions about the denotation of the nominal and the semantic significance of the classifier.

The structure of this paper is as follows. In the remainder of this section, we will discuss various approaches to the Hungarian mass/count and classifier phenomena proposed in the literature. In \sectref{schv-nem:sec:2}, we present a range of tests where sortal classifiers make a difference in the interpretation of numerical constructions. In \sectref{schv-nem:sec:3}, we discuss evidence against treating plurality as a classifier as well as data that questions number-neutrality. In  \sectref{schv-nem:sec:4}, we explore the problem of kind interpretation in Hungarian and try to place the language in \citeposst{chierchia-98b} %the \citeauthor{chierchia-98b}n (\citeyear{chierchia-98b}) 
typology of nominal denotation. In \sectref{schv-nem:sec:5}, we provide a semantic analysis of sortal classifiers. In \sectref{schv-nem:sec:6}, we draw some conclusions and discuss implications.

We begin by providing some general background on classifiers in Hungarian, followed by a review of the existing analyses of Hungarian classifier phenomena in the literature. The phenomenon of classifiers has been often noted in the literature on Hungarian classifiers \citep{beckwith-92, beckwith-07, csirmaz-dekany-14, schvarcz-rothstein-17, szabo-toth-18, schvarcz-wohlmuth-20}. Some of the canonical examples include:  

\ea \label{schv-nem:ex:2}
\ea \label{schv-nem:ex:2a}
\gll két \minsp{(} fej) hagyma\\  
     two {} \textsc{cl\textsubscript{head}} onion\\ 
\glt `two heads of onion'
\ex \label{schv-nem:ex:2b}
\gll három \minsp{(} szál) rózsa\\
    three {} \textsc{cl\textsubscript{thread}} rose\\
\glt `three (threads of) roses'
\ex \label{schv-nem:ex:2c}
\gll három darab könyv\\
    three \textsc{cl\textsubscript{general}} book\\
\glt `three books'
\z
\z

\noindent Crucially, the classifiers in \REF{schv-nem:ex:2} are optional and look like sortal classifiers. While some select nouns according to shape and size (e.g. \textit{fej} `head' takes nouns denoting large round objects and \textit{szál} `thread' combines with nouns denoting long thin objects), the general classifier  \textit{darab} combines with any notionally countable noun. The construction without the classifier has the same meaning as its classifier counterpart. \REF{schv-nem:ex:3a} and \REF{schv-nem:ex:3b} have the same meaning, while \REF{schv-nem:ex:3c} contrasts with \REF{schv-nem:ex:3b}. This is due to the fact that \textit{köteg} `bunch' is a so called `group or collective classifier' which groups roses into a higher order entity. 

\ea \label{schv-nem:ex:3}
\ea \label{schv-nem:ex:3a}
\gll három rózsa\\
    three  rose\\
\glt `three roses'
\ex \label{schv-nem:ex:3b}
\gll három szál rózsa\\
    three \textsc{cl\textsubscript{thread}} rose\\
\glt `three threads of roses'
\ex \label{schv-nem:ex:3c}
\gll három köteg rózsa\\
    three \textsc{cl\textsubscript{bunch}} rose\\
\glt `three bunches of roses'
\z
\z

\noindent These facts about Hungarian pose a problem for the traditional categorizations, which define two major systems of making nouns countable \citep{greenberg-74, chierchia-98a}. On the one hand, we find languages such as Mandarin Chinese and Japanese, which lack a genuine plural marker; have no distinction between count and mass nouns on the nominal level; and where bare nouns can occur as arguments of kind-taking predicates. On the other hand, there are languages, such as English, French or Dutch, where nouns are categorised as count or mass; count nouns are directly modified by numerals. These languages have a genuine morphological marker of plurality; and do not allow bare singular arguments. Hungarian exhibits both typical classifier language traits and mass/count language traits. While it has a rich classifier system and uses bare singular arguments \REF{schv-nem:ex:4}, it also manifests a genuine mass/count distinction \citep{schvarcz-14} and morphologically marks plurality \REF{schv-nem:ex:5}.\footnote{While examples such as  \REF{schv-nem:ex:4}  are limited and not highly productive in the language, they highlight the availability of bare singular arguments in Hungarian. The few context and constructions in which the use of the bare singular is possible are discussed in detail under \sectref{schv-nem:sec:4}.}

\ea \label{schv-nem:ex:4}
\gll Sas nem kapkod legyek után.\\
    eagle not fluster.\textsc{pres}.\textsc{3sg} fly.\textsc{pl}  after\\
\glt `Eagles do not fluster after flies.'
\ex \label{schv-nem:ex:5}
\gll újság / újság-ok\\
   newspaper {} newspaper-\textsc{pl}\\
\z

\noindent To account for the occurrence of sortal classifiers, three approaches have been proposed in the literature.

First, \citet{csirmaz-dekany-14} suggest treating Hungarian as a classifier language, in which ``bare nominals [...] are non-atomic, they denote an undifferentiated mass'' (p. 142), and hence counting requires either an explicit lexical classifier (e.g. \textit{darab}) or a null sortal classifier:

\ea \label{schv-nem:ex:6}
\ea \label{schv-nem:ex:6a}
\gll három darab újság\\  
     three \textsc{cl\textsubscript{general}} newspaper\\ 
\ex \label{schv-nem:ex:6b}
\gll három $\emptyset_{\textsc{cl}}$  újság\\
   three \textsc{cl$_{\emptyset}$} newspaper\\
\z
\z

\noindent Treating Hungarian as a classifier language is supported by the absence of plural marking on the nominal upon combining with numerals greater than ‘one’. To address this issue, \citet{dekany-11} suggests treating the Hungarian morphological plural marker \textit{-k} as a type of plural classifier which spans in two positions: Number and Classifier.

While \citeauthor{csirmaz-dekany-14}'s analysis accounts for the use of sortal classifiers, it cannot account for the facts about plurality and mass/count phenomena manifested in the language. Other analyses suggest that the issue of plural is orthogonal and that the absence of plural marking is not related to countability \citep{borer-05, schvarcz-rothstein-17}.

Second, regarding mass/count phenomena, \citet{schvarcz-rothstein-17} argue that  Hungarian has purely mass nouns, a few purely count nouns and a wide range of flexible nouns. Nouns like \textit{újság} `newspaper' can be used in counting contexts either as a count noun or in a sortal classifier construction. They observe that the patterns of classifier use is due to the ambiguity between a count and mass interpretation of a flexible noun. This is illustrated below, where in \REF{schv-nem:ex:7a} the mass counterpart of the flexible noun \textit{újság} `newspaper’ obligatorily takes the general classifier \textit{darab}, while the count counterpart of the same flexible noun \REF{schv-nem:ex:7b} can be counted without any classifier and even bars the use of one. 

\ea \label{schv-nem:ex:7}
\ea \label{schv-nem:ex:7a}
\gll három \minsp{*(} darab) újság\\ %$_{\textsc{mass}}$ \\
three {} \textsc{cl\textsubscript{general}} newspaper.\textsc{mass}\\
\glt `three copies of newspapers' 
\ex \label{schv-nem:ex:7b}
\gll három  \minsp{(*} darab) újság\\ %$_{\textsc{count}}$ \\
three {} \textsc{cl\textsubscript{general}}  newspaper.\textsc{count}\\
\glt `three copies of newspapers /  three titles of newspapers'
\z
\z

\noindent Third, \citet{erbach-etal-19} argue that notionally count nouns are semantically number neutral, in the sense of \citet{farkas-deswart-10}, denoting both atomic entities and sums thereof. Under their analysis, classifiers are required by the numeral semantics and not by the nominals \citep{krifka-95, sudo-17}. However, their analysis does not address classifier optionality per se.

In this paper, we will defend the noun-flexibility analysis. We base our analysis on observations emerging from the interpretations of the two structures, and show that neither plural-as-a-classifier nor number-neutrality fully explains the data on kind interpretation and classifier optionality.  Our data show that the availability of a kind interpretation tells apart the two structures in \REF{schv-nem:ex:1}: while the NUM+N construction \REF{schv-nem:ex:1a} can either refer to a set of individuals or subkinds, the classifier construction \REF{schv-nem:ex:1a} can only refer to a set of individuals. In addition to identifying the function and interpretation of the classifier, we observe that plural marked nouns can freely get a subkind interpretation while classifier phrases can not. Based on this semantic difference, we maintain and provide further evidence for the claims of \citet{schvarcz-rothstein-17} that plural marking cannot be treated as a classifier \citep[contra][]{dekany-11}, showing that the two elements fulfill different functions.  We will also discuss several cases which contradict the assumptions made by  \citet{erbach-etal-19} and  rule out a number-neutrality analysis for Hungarian. 

\section{The semantic effect of \textit{darab} on kind and subkind reading} \label{schv-nem:sec:2}

While \citet{schvarcz-rothstein-17} suggest that there is no significant interpretational difference between numerical constructions involving an overt classifier and a covert one, we observe that there is, in fact, an important semantic contrast between the two structures. For example, NUM+N in \REF{schv-nem:ex:8a} may refer to a plurality of newspaper copies; to a plurality of sub-kinds of newspapers (daily, monthly, weekly); or to a plurality of newspaper titles (\textit{The Herald Tribune}, \textit{The New York Times}, \textit{The Economist}).\footnote{An anonymous reviewer points out that while the plurality of sub-kinds reading exists, it is rather unnatural in the case of the noun \textit{újság}. More natural examples include: 

\ea[?]{\gll Három állat rak tojás-ok-at: a hal, a hüllő és a madár.\\
three animal lay.\textsc{pres}.\textsc{3sg} egg-\textsc{pl}-\textsc{acc} the fish, the reptile and the bird  \\
\glt `Three animals lay eggs: the fish, the reptile and the bird.'}
\ex[]{\gll Két madár nem tud repülni: a strucc és a pingvin.\\
two bird not can  fly.\textsc{inf} the ostrich and the penguin   \\
\glt `Two birds cannot fly: the ostrich and the penguin.'}
\z} In contrast, NUM+CL+N in \REF{schv-nem:ex:8b} can only have a plurality of individuals interpretation under which it can only refer to a plurality of newspaper objects, namely three copies.\footnote{``Sk.'' stands for the subkind interpretation, ``Pl.'' stands for the plurality of individuals interpretation, ``K.'' stands for the kind interpretation.}

\ea \label{schv-nem:ex:8}
\ea \label{schv-nem:ex:8a}
\gll Három újság-ot árul ez az újságárus.\\
three newspaper-\textsc{acc} sell.\textsc{pres}.\textsc{3sg} this the news.vendor\\
\glt `This newspaper vendor sells three newspapers.'  \hfill \raisebox{2.3\baselineskip}[0pt][0pt]{[Sk./Pl.]}
\ex \label{schv-nem:ex:8b}
\gll Három darab újság-ot árul ez az újságárus.\\
three \textsc{cl\textsubscript{general}} newspaper-\textsc{acc} sell.\textsc{pres}.\textsc{3sg} this the news.vendor\\
\glt `This newspaper vendor sells three newspapers.'   \hfill \raisebox{2.3\baselineskip}[0pt][0pt]{[Pl.]}
\z
\z

\noindent The ambiguity between an existential and a subkind interpretation of NUM+N constructions can also be observed in English \REF{schv-nem:ex:9}. English, however, does not have a mechanism parallel to the Hungarian general classifier to disambiguate the two readings in favor of an individuating one. 

\ea \label{schv-nem:ex:9}
\ea \label{schv-nem:ex:9a}
This newsvendor sells three newspapers: \textit{The New York Times, The Herald Tribune} and \textit{The Economist}. He has 50 copies delivered of each.
\ex \label{schv-nem:ex:9b}
This newsvendor sells three newspapers: he only has one copy left of \textit{The New York Times, The Herald Tribune} and \textit{The Economist}.
\z
\z

\noindent Based on the contrast in \REF{schv-nem:ex:8}, we suggest that the role of sortal numeral classifiers in Hungarian is that of restricting subkind reading, thereby eliminating the ambiguity found in numerical expressions. A number of tests and contexts confirm our prediction for Hungarian. In order to test our hypothesis, we carefully selected structures and contexts that disallow the existential interpretation to occur. In these cases, we expected the use of a sortal individual classifier to be infelicitous.

First, kind-reference generic sentences express properties true of kinds, species or classes of objects, but not of individual objects \citep{krifka-etal-95}, hence they should be incompatible with the general classifier \textit{darab}. The use of the kind classifier, \textit{fajta} `kind of/type of', is felicitous in such contexts.\footnote{An anonymous reviewer brings our attention to an alternative interpretation: \textit{újság} may have a title reading. In that case \textit{darab} can refer to newspaper titles, suggesting that the titles reading may be individual-denoting. This ambiguity in the interpretation can be attributed to a polysemy between physical object and informational object senses \citep{puste95, asher-11}. It has been discussed by \citet{schvarcz-wohlmuth-20} if such polysemous  nouns occur in classifier expressions, the numeral can only count physical objects. Nevertheless, with nouns that do not exhibit such polysemy, the classifier expression is ruled out: 

\ea[*]{\gll Két darab madár a kihalás szélén áll: a strucc és a pingvin.\\
two \textsc{cl\textsubscript{general}} bird the extinction verge.\textsc{sup} stand.\textsc{pres}.\textsc{3sg} the ostrich and the penguin   \\
\glt `Two birds stand on the verge of extinction: the ostrich and the penguin.'}
\z} 

\ea \label{schv-nem:ex:10}
\gll Három \minsp{\{(*} darab) / \minsp{(} fajta)\} újság a megszűnés szél-én áll. \\
three {} \textsc{cl\textsubscript{general}} {} {} \textsc{cl\textsubscript{kind}} newspaper the ceasing.to.exist verge-\textsc{sup} stand.\textsc{pres}.\textsc{3sg}\\
\glt `Three newspapers are on the verge of ceasing to exist.' / `Three kinds of newspapers are on the verge of ceasing to exist.'
\ex \label{schv-nem:ex:11}
\gll Ez az újságárus árulja a három \minsp{\{(*} darab) / \minsp{(} fajta)\}  betiltott újságo-t, bár tudja, hogy azok feketelistá-ra kerültek. \\
this the newsvendor sell.\textsc{pres}.\textsc{3sg}  the  three  {} \textsc{cl\textsubscript{general}} {} {} \textsc{cl\textsubscript{kind}} banned newspaper-\textsc{acc} although know.\textsc{pres}.\textsc{3sg} that those blacklist-\textsc{sbl} get.\textsc{past}.\textsc{3sg}\\
\glt `This newsvendor sells the three banned newspapers, although he knows that those have been backlisted.' / `This newsvendor sells the three kinds of banned newspapers, although he knows that those have been backlisted.'
\ex \label{schv-nem:ex:12}
\gll János három \minsp{\{(*} darab) / \minsp{(} fajta)\} marhá-t tenyészt: Holstein marhá-t, Angus-t, és barna svájci marhá-t \\
 John three {} \textsc{cl\textsubscript{general}} {} {} \textsc{cl\textsubscript{kind}} cattle-\textsc{acc} breed.\textsc{pres}.\textsc{3sg}  Holstein cattle-\textsc{acc} Angus-\textsc{acc} and brown Swiss cattle-\textsc{acc} \\
\glt `John breeds three cows: Holstein, Angus and brown Swiss cow.' / `John breeds three kinds of cows. Holstein, Angus and brown Swiss cow.'
\z


\noindent Second, distributive operators and reciprocals require plural atomic antecedents \citep{link-83, rotstein-09, schwarzschild-11, schvarcz-14}. The verb \textit{kiadni} `to publish' in \REF{schv-nem:ex:13}, and the adverb \textit{gyakran} `often' in \REF{schv-nem:ex:15} rule out a plurality of individuals interpretation.  In contexts where only  a plurality of subkinds interpretation is possible, the interplay between these verbs and distributive or reciprocal phrases results in the impossibility of the use of the classifier.  In \REF{schv-nem:ex:13}, for example, the context refers to a multiplicity of copies of different newspapers, as we expect news agencies to publish a large number of  various newspaper editions. Therefore, a plurality of individuals reading induced by the classifier is ruled out.

\ea \label{schv-nem:ex:13}
\gll  A Magyar Távirati Iroda három \minsp{(*} darab) újság-ot egymás után adott ki. \\
the Hungarian news agency three {} \textsc{cl\textsubscript{general}} newspaper-\textsc{acc} each.other after publish.\textsc{past}.\textsc{3sg} \textsc{vm}\\
\glt `MTI [Hungarian news agency] %The Hungarian Newsagency 
published three newspapers one after another.'
\ex \label{schv-nem:ex:14}
\gll  Susan Rothstein öt \minsp{(*} darab) könyv-ét egymás után adták ki. \\
 Susan Rothstein five {} \textsc{cl\textsubscript{general}} book-\textsc{acc} each.other after publish.\textsc{past}.\textsc{3sg} \textsc{vm} \\
\glt `Susan Rothstein’s five books were published one after the other.'
\ex \label{schv-nem:ex:15}
\gll János három \minsp{(*} darab) újság-ot vesz keddenként, ez-ek-ből folyamatosan tud tájékozódni arról, hogy mi történik a nagyvilág-ban. \\
John three {} \textsc{cl\textsubscript{general}} newspaper-\textsc{acc} buy.\textsc{pres}.\textsc{3sg} on.Tuesdays   \textsc{dem}-\textsc{pl}-\textsc{ela} successively can.\textsc{pres}.\textsc{3sg} about that what happen \textsc{pres}.\textsc{3sg} the big.world-\textsc{iness}\\
\glt `John buys three newspapers on Tuesdays. He learns from these what goes on in the world all the time.'
\z

\noindent The third test involves expressions which refer to multiple instantiations of a noun. Contexts which indicate multiple instantiations of kinds are not compatible with the structure involving a sortal individual classifier. The classifier can be used in the second sentence to mark the contrast between the two interpretations of the noun: \REF{schv-nem:ex:17c} is a constellation in which \textit{újság} in the first sentence must have a sub-kind reading, while in the second sentence, it can only be interpreted as a plurality of newspapers. 

\ea \label{schv-nem:ex:16}
\gll Mari három \minsp{(*} tő) rózsá-t ültetett: angol-, futó- és teahibrid  rózsá-t. Összesen ötvenhárm-at. \\
 Mari three {} \textsc{cl\textsubscript{root}} rose-\textsc{acc} plant.\textsc{past}.\textsc{3sg} English  rambler and  hybrid  rose-\textsc{acc}  in.total    fifty.three-\textsc{acc}\\
\glt `Mary planted three roses: English roses, rambler roses and hybrid roses. In total 53. '
\ex \label{schv-nem:ex:17} \textsc{Context}: John buys newspapers for all ten workers in his office, one from each kind.
\ea \label{schv-nem:ex:17b}
\gll János három \minsp{(*} darab) újság-ot vett. Összesen  harminc-at. \\
John  three {} \textsc{cl\textsubscript{general}} newspaper-\textsc{acc} buy.\textsc{past}.\textsc{3sg} in.total thirty-\textsc{acc}\\
\glt `John bought three newspapers. In total 30.'
\ex \label{schv-nem:ex:17c} 
\gll János három újság-ot vett. Összesen harminc darab újság-ot. \\
John  three  newspaper-\textsc{acc} buy.\textsc{past}.\textsc{3sg} in.total thirty \textsc{cl\textsubscript{general}} newspaper-\textsc{acc}\\
\glt `John bought three newspapers. In total thirty newspapers.'
\z
\z

\noindent Fourth, kind-referring anaphoric expressions, such as \textit{ezek a fajta} `these kinds', are not compatible with the NUM+CL+N construction. Expressions of this kind include the kind classifier \textit{fajta} `kind of/type of', and thus can only refer back to a kind-denoting expression.

\ea \label{schv-nem:ex:18}
\gll János három \minsp{(*} darab) újság-ot gyűjt. Ez-ek a fajta kiadás-ok ritká-k. \\
John three {} \textsc{cl\textsubscript{general}} newspaper-\textsc{acc} collect.\textsc{pres}.\textsc{3sg} \textsc{dem}-\textsc{pl} the \textsc{cl\textsubscript{kind}} edition-\textsc{pl} rare-\textsc{pl}\\
\glt `John collects three newspapers. These kinds of editions are rare.'
\z 

\ea \label{schv-nem:ex:19}
\gll Mari három \minsp{(*} tő) virág-gal ültette be a kert-et: nárcisz-szal, tulipán-nal és rózsá-val. Ez-ek a fajta virág-ok csak tavasz-szal ültet-hető-ek. \\
Mari three {} \textsc{cl\textsubscript{root}} flower-\textsc{inst} plant.\textsc{past}.\textsc{3sg} VM the garden-\textsc{acc} daffodil-\textsc{inst} tulip-\textsc{inst} and rose-\textsc{inst}  \textsc{dem}-\textsc{pl} the \textsc{cl\textsubscript{kind}} flower-\textsc{pl} only spring-\textsc{inst} plant-\textsc{pos}-\textsc{pl}\\
\glt ‘Mary filled the garden with three flowers: daffodils, tulips and roses. These kinds of flowers can only be planted in the spring.'
\z 

\noindent The above tests indicate that the insertion of a sortal individuating classifier in NUM+N constructions has an impact on the interpretation, namely: NUM+N can have subkind and existential readings, while NUM+CL+N can only have an existential reading.

This interpretational difference is not expected under a null classifier analysis, such as the one put forward by \citet{csirmaz-dekany-14}, which assigns the same semantics for the null sortal classifier as the one assumed for \textit{darab}. If we assume a one-to-one mapping between the syntactic structure and semantic interpretation, the differences between subkind and plurality of individuals readings observed above remain unexplained. However, theoretically we could assume the existence of a semantically underspecified null classifier which could potentially derive the readings observed in this paper: under the subkind reading, the null classifier could have a semantics similar to the kind-classifier, \textit{fajta}, while under the plurality of individuals reading, the semantics of the null classifier would be equivalent to \textit{darab}. To the best of our knowledge, such null classifiers have not been observed in other languages. 

As we will discuss in the next section, number neutrality does not fully explain the data on kind interpretation nor does it provide a solution for classifier optionality. \citet{erbach-etal-19} does not address the interpretational ambiguity discussed above.  In addition, the kind interpretation of nominals in Hungarian is more complex than assumed in \citet{erbach-etal-19}. Moreover, the role of the optionally used classifier in a framework in which the classifier is required by the numeral remains unsolved. 
In the next section, we argue that the noun flexibility approach is able to better capture the facts discussed above. 

\section{In defense of noun flexibility} \label{schv-nem:sec:3}
 
First, we provide further support to  \citeauthor{schvarcz-rothstein-17}’s (\citeyear{schvarcz-rothstein-17}) claims. Based on syntactic and semantic evidence, we argue against \citeauthor{csirmaz-dekany-14}'s (\citeyear{csirmaz-dekany-14}) `plurality-as-a-classifier' claim, showing that classifiers and the plural neither compete for the same syntactic position nor do they have the same interpretation. In addition, by taking a closer look at kind-readings of number-neutral nominals, we give counterarguments to the number-neutrality analysis. 

\subsection{Plural is not a classifier} \label{schv-nem:sec:3.1}

As mentioned above, \citet{csirmaz-dekany-14} argue in favour of treating Hungarian as a classifier language. In line with this view, \citet{dekany-11} suggests treating the plural in Hungarian as a classifier, whilst maintaining a strict complementarity hypothesis.  However, due to the fact that Hungarian has a productive plural marker, unlike typical classifier languages, which lack such a marker
%\citep{chierchia-98a,Chierchia-2010,cheng-sybesma-99}
(\citealt{chierchia-98a}; \citeyear{Chierchia-2010}; \citealt{cheng-sybesma-99}), Hungarian cannot be considered a classifier language. Moreover, as \citet{schvarcz-14} and \citet{schvarcz-rothstein-17} show, not only does Hungarian have a plural marker, but plurality is also sensitive to the mass/count distinction. We present novel data that support the claim that plurality should not be analysed as a plural sortal classifier.  In this section we raise five issues: frequency of co-occurrence; the impossibility of classifier doubling; differences in the distributions of plurals and classifiers and in agreement phenomena; and interpretational contrasts.

We look first at the frequency of classifiers and plurals co-occurring in the same phrase. A corpus study reveals that plural marking and classifiers co-occur much more frequently than previously thought in contexts that were not discussed before. These include: bare adjectival phrases \REF{schv-nem:ex:20a}, \REF{schv-nem:ex:21a} and \REF{schv-nem:ex:22a}, definite constructions \REF{schv-nem:ex:20b}, \REF{schv-nem:ex:21b} and \REF{schv-nem:ex:22b} and demonstratives \REF{schv-nem:ex:20c}, \REF{schv-nem:ex:21c} and \REF{schv-nem:ex:22c}. The only constructions in which the two cannot co-occur are the ones that contain either a numeral or a quantifier, which cannot combine with plural-marked nouns at any rate.\footnote{Addressing the observations made by an anonymous reviewer regarding the unexpected co-occurrence of plural marking and classifiers, we assume that the classifier first combines with the mass counterpart of a flexible noun deriving a count expression. This expression then can be marked plural. Since numerals combine with singular expressions, it follows that the  that CL+N.\textsc{pl} expressions do not take a numeral and these expressions appear only with adjectives, demonstratives and definite constructions. Deriving the syntax behind constructions involving plurals and classifiers lies beyond the scope of this paper.}

\ea \label{schv-nem:ex:20}
\ea \label{schv-nem:ex:20a}
\gll szép szál virág-ok \\  
pretty \textsc{cl\textsubscript{thread}} flower-\textsc{pl}\\ 
\glt `pretty flowers'
\ex \label{schv-nem:ex:20b}
\gll a \minsp{(} szép) szál virág-ok\\
the {} pretty \textsc{cl\textsubscript{thread}} flower-\textsc{pl}\\
\glt `the pretty flowers'
\ex \label{schv-nem:ex:20c}
\gll \minsp{(} ez-ek) a szál virág-ok \\
{} \textsc{dem}-\textsc{pl} the \textsc{cl\textsubscript{thread}} flower-\textsc{pl}\\
\glt `these threads of flowers'
\z
\ex \label{schv-nem:ex:21}
\ea \label{schv-nem:ex:21a}
\gll  szép darab hús-ok \\  
nice \textsc{cl\textsubscript{piece}}  meat-\textsc{pl}\\ 
\glt `nice pieces of meat'
\ex \label{schv-nem:ex:21b}
\gll a \minsp{(} szép) darab hús-ok \\
the {} nice \textsc{cl\textsubscript{piece}} meat-\textsc{pl}\\
\glt `the nice pieces of meat'
\ex \label{schv-nem:ex:21c}
\gll \minsp{(} az-ok) a szép darab hús-ok \\
{} \textsc{dem}-\textsc{pl} the nice \textsc{cl\textsubscript{piece}} meat-\textsc{pl}\\
\glt `those nice pieces of meat'
\z
\ex \label{schv-nem:ex:22}
\ea \label{schv-nem:ex:22a}
\gll nagy fej káposztá-k \\  
big \textsc{cl\textsubscript{piece}} cabbage-\textsc{pl}\\ 
\glt `big (head of) cabbages'
\ex \label{schv-nem:ex:22b}
\gll a \minsp{(} nagy) fej káposzt-ák     \\
the {} big \textsc{cl\textsubscript{piece}} cabbage-\textsc{pl}\\
\glt `the big (heads of) cabbages'
\ex \label{schv-nem:ex:22c}
\gll \minsp{(} az-ok) a fej káposztá-k \\
{} \textsc{dem}-\textsc{pl} the \textsc{cl\textsubscript{piece}} cabbage-\textsc{pl}\\
\glt `those big (heads of) cabbages'
\z
\z

\noindent A second issue concerning the co-occurrence of plural marking and a classifier is reduplication. If plural were a classifier, then in the above examples, we would assume a double classifier. Yet classifier doubling -- either the reduplication of the same classifier \REF{schv-nem:ex:23a} or the combination of two different classifiers \REF{schv-nem:ex:23b} -- is ruled out in Hungarian. In contrast, in Mandarin Chinese, a true classifier language, reduplicated classifiers serve as unit-plurality markers \REF{schv-nem:ex:24} \citep{zhang-13}. A similar phenomena can be found in Cantonese \REF{schv-nem:ex:25} \citep{wong-98}.

\ea \label{schv-nem:ex:23}
\ea[*]{\gll egy darab szál virág  \\  
one \textsc{cl\textsubscript{piece}} \textsc{cl\textsubscript{thread}} flower\\}  \label{schv-nem:ex:23a}
\ex[*]{\gll egy szál szál virág\\
one \textsc{cl\textsubscript{thread}} \textsc{cl\textsubscript{thread}} flower\\}  \label{schv-nem:ex:23b}
\z
\ex \label{schv-nem:ex:24}
\ea \label{schv-nem:ex:24a}
\gll ge-ge xuesgeng dou you ziji de wangye\\  
\textsc{cl}-\textsc{red} student all have own \textsc{de} website\\
\glt `All of the students have their own webpages.'\\\hfill \citep[Mandarin Chinese;][p. 118, ex. (234a)]{zhang-13}
\ex \label{schv-nem:ex:24b}
\gll he-li piao-zhe \minsp{(} yi) duo-duo lianhua \\
river-in float-\textsc{dur} {} one \textsc{cl}-\textsc{red} lotus\\
\glt `There are many lotuses floating on the river.'\\ \hfill \citep[Mandarin Chinese;][p. 118, ex. (230a)]{zhang-13}
\z

% \ea \label{schv-nem:ex:24}
% \ea \label{schv-nem:ex:24a}
% \gll ge-ge xuesgeng dou you ziji de wangye\\  
% \textsc{cl}-\textsc{red} student all have own \textsc{de} website\\ 
% \glt `All of the students have their own webpages.'\hfill \raisebox{2.4\baselineskip}[0pt][0pt]{[Mandarin Chinese]} \hfill \raisebox{.0\baselineskip}[0pt][0pt]{\citep[p. 118, ex. (234a)]{zhang-13}}
% \ex \label{schv-nem:ex:24b}
% \gll he-li piao-zhe (yi)duo-duo lianhua \\
% river-in float-\textsc{dur} one \textsc{cl}-\textsc{red} lotus\\
% \glt `There are many lotuses floating on the river.' \hfill \raisebox{2.4\baselineskip}[0pt][0pt]{[Mandarin Chinese]} \hfill \raisebox{.0\baselineskip}[0pt][0pt]{\citep[p. 118, ex. (230a)]{zhang-13}}
% \z
% \z

\ex \label{schv-nem:ex:25}
\gll go go hoksaang \\
\textsc{cl} \textsc{cl} student\\
\glt `every student' \hfill \citep[Cantonese;][p. 16]{wong-98}
\z

% \ea \label{schv-nem:ex:25}
% \gll go go hoksaang \\
% \textsc{cl} \textsc{cl} student\\
% \glt `every student' \hfill \citep[p. 16]{wong-98} \hfill \raisebox{2.4\baselineskip}[0pt][0pt]{[Cantonese]} 
% \z

\noindent Third, the distribution of bare classifier and bare plural expressions differ: while bare plurals are allowed in argument positions, bare classifier phrases are not. While speakers of some dialects may find \REF{schv-nem:ex:26b} acceptable, all of our informants rule out \REF{schv-nem:ex:27b}.\footnote{While informants point out that \REF{schv-nem:ex:26b} may be acceptable in a context where more information is provided prior to the utterance, all of them agree that without any context it is ungrammatical.} This difference is due to the position of the bare classifier phrase: as arguments they are unequivocally ungrammatical.  

\ea \label{schv-nem:ex:26}
\ea[]{\gll Rózsá-k-at ültettem a kert-be. \\  
rose-\textsc{pl}-\textsc{acc} plant.\textsc{pres}.\textsc{3sg} the garden-\textsc{ill}\\ 
\glt `I planted roses in the garden.'}\label{schv-nem:ex:26a}
\ex[*]{\gll Tő rózsá-t ültettem a kert-be.\\
\textsc{cl\textsubscript{root}} rose-\textsc{acc} plant.\textsc{pres}.\textsc{3sg} the garden-\textsc{ill}\\} \label{schv-nem:ex:26b}
\z
\ex \label{schv-nem:ex:27}
\ea[]{\gll Újság-ok érkeztek az Amazon csomag-ban.  \\  
newspaper-\textsc{pl} arrive.\textsc{past}.\textsc{3pl} the Amazon package-\textsc{iness}\\ 
\glt `Newspapers arrived in the Amazon package.'}
\ex[*]{\gll Darab újság érkezett az Amazon csomag-ban. \\
\textsc{cl\textsubscript{general}} newspaper arrive.\textsc{past}.\textsc{3sg} the Amazon package-\textsc{iness}\\}  \label{schv-nem:ex:27b}
\z
\z

\noindent Fourth, we look at agreement phenomena. Following the Hungarian patterns of agreement, verbs agree with their subjects in person and number, and external demonstratives agree in number and case \citep{kenesei-etal-98}. The plural marker induces agreement on the verb and on the demonstrative. Classifiers do not. 

\ea \label{schv-nem:ex:28}
\ea \label{schv-nem:ex:28a}
\gll Ez-ek a virág-ok szép-ek. \\  
\textsc{dem}-\textsc{pl} the flower-\textsc{pl} beautiful-\textsc{pl}\\ 
\glt `These flowers are beautiful.'
\ex \label{schv-nem:ex:28b}
\gll Ez a szál virág szép. \\
\textsc{dem} the \textsc{cl\textsubscript{thread}} flower beautiful \\
\glt `This thread of flower is beautiful.'
\z
\ex \label{schv-nem:ex:29}
\gll Ez-ek a káposzták már megértek, de az a fej\hspace{1.5cm} \minsp{(} káposzta) még nem ért meg. \\
\textsc{dem}-\textsc{pl} the cabbage-\textsc{pl} already \textsc{vm}.ripe.\textsc{past}.\textsc{3pl} but that  the \textsc{cl\textsubscript{head}} {} cabbage yet not ripe.\textsc{past}.\textsc{3sg} \textsc{vm}\\
\glt `These cabbages are ripe already, but that head of cabbage has not yet ripened.'
\z

\noindent Lastly, interpretational differences can be observed between the two expressions discussed: constructions containing a classifier cannot receive a subkind interpretation, while plural-marked nouns can have either a kind, subkind or a plurality of individuals reading.  \REF{schv-nem:ex:31a} is modeled on an example from \citet{landman-rothstein-10} and can refer to the guest-kind, and to a plural set of guests. We may also imagine dividing a set of guests into sub-kinds:  invited guests and uninvited guests. \REF{schv-nem:ex:31a} may also be true in this scenario: \textit{Vendégek érkeztek két órán át, meghívottak és hívatlanok} (`Guests arrived for two hours, invited and uninvited ones'). In \REF{schv-nem:ex:31b} we can only get the plurality of guests reading. 

\ea \label{schv-nem:ex:30}
\ea \label{schv-nem:ex:30a}
\gll Újság-ok-at árul ez az újságárus.\\
newspaper-\textsc{pl}-\textsc{acc} sell.\textsc{pres}.\textsc{3sg} \textsc{dem} the  newsvendor\\ \hfill [Sk./Pl.]
\glt `This newsvendor sells newspapers.'
\ex \label{schv-nem:ex:30b}
\gll Három darab újság-ot árul ez az újságárus.\\
three \textsc{cl\textsubscript{general}} newspaper-\textsc{acc} sell.\textsc{pres}.\textsc{3sg} \textsc{dem} the  newsvendor\\
\glt `This newsvendor sells three newspapers.'   \hfill \raisebox{2.3\baselineskip}[0pt][0pt]{[Pl.]}
\z

% \ea \label{schv-nem:ex:30}
% \ea \label{schv-nem:ex:30a}
% \gll Újság-ok-at árul ez az újságárus.\\
% newspaper-\textsc{pl}-\textsc{acc} sell.\textsc{pres}.\textsc{3sg} \textsc{dem} the  newsvendor\\
% \glt `This newsvendor sells newspapers.'  \hfill \raisebox{2.3\baselineskip}[0pt][0pt]{[Sk./Pl.]}
% \ex \label{schv-nem:ex:30b}
% \gll Három darab újság-ot árul ez az újságárus.\\
% three \textsc{cl}\textsubscript{general} newspaper-\textsc{acc} sell.\textsc{pres}.\textsc{3sg} \textsc{dem} the  newsvendor\\
% \glt `This newsvendor sells three newspapers.'   \hfill \raisebox{2.3\baselineskip}[0pt][0pt]{[Pl.]}
% \z
% \z

\ex \label{schv-nem:ex:31}
\ea \label{schv-nem:ex:31a}
\gll Vendég-ek érkeztek két órá-n át.\\
guest-\textsc{pl} arrive.\textsc{past}.\textsc{3pl} two hour-\textsc{sup} for\\ \hfill [K./Sk./Pl.]
\glt `Guests arrived for two hours.'  \hfill \citep[p.188, (14)]{schvarcz-rothstein-17} 
\ex \label{schv-nem:ex:31b}
\gll Három darab vendég érkezett két órá-n át.\\
three \textsc{cl\textsubscript{general}} guest  arrive.\textsc{past}.\textsc{3sg} two  hours-\textsc{sup} for\\
\glt `Three guests arrived for two hours.'   \hfill \raisebox{2.3\baselineskip}[0pt][0pt]{[Pl.]}
\z
\z

% \ea \label{schv-nem:ex:31}
% \ea \label{schv-nem:ex:31a}
% \gll Vendég-ek érkeztek két órá-n át.\\
% guest-\textsc{pl} arrive.\textsc{past}.\textsc{3pl} two hour-\textsc{sup} for/during\\
% \glt `Guests arrived for two hours.'  \hfill \raisebox{2.2\baselineskip}[0pt][0pt]{[K./Sk./Pl.]} \\ \citep[p.188, (14)]{schvarcz-rothstein-17} 
% \ex \label{schv-nem:ex:31b}
% \gll Három darab vendég érkezett két órá-n át.\\
% three \textsc{cl}\textsubscript{general} guest  arrive.\textsc{past}.\textsc{3sg} two  hours-\textsc{sup} for/during\\
% \glt `Three guests arrived for two hours'   \hfill \raisebox{2.3\baselineskip}[0pt][0pt]{[Pl.]}
% \z
% \z

\noindent These data suggest that Hungarian sortal classifiers cannot be syntactically and semantically equated to plural markers. Their distribution and interpretation differ; they exhibit different agreement patterns; and they can in fact co-occur more frequently than previously assumed. We now turn to the number-neutrality analysis proposed by \citet{erbach-etal-19}. 

\subsection{Ruling out number neutrality} \label{schv-nem:sec:3.2}

{\sloppy Based on a cumulativity approach to measurement, i.e. that  measure DPs call upon cumulative predicates (\citealt{krifka-89, filip-92, filip-05, nakanishi-03};\linebreak \citealt{schwarzschild-06}), and on an analysis under which it is the  semantics of numerals requires the use of the classifier rather than that of a noun, \citep{krifka-95, erbach-etal-19, sudo-17} argue that Hungarian notionally count singular nouns are number-neutral.}

At a closer investigation, however, we find that number neutrality cannot accurately account for the data. Some of the phenomena we point out include: the inaccessibility of atoms in pseudo-partitive measure DPs and the availability of mass readings of singular nouns.  For further evidence see \citet{schvarcz-nemes-19}.

First, contra \citet{erbach-etal-19}, our data indicate that in Hungarian measure DPs, atoms are not accessible in the denotation of nouns -- may they be notionally count, dual-life or mass.  One of the major arguments of \citet{erbach-etal-19} relies on atomicity: while in the case of plural count nouns used in measure DPs atoms are accessible to semantic operations making them felicitous in reciprocal contexts, this does not hold of mass nouns. Their examples include \textit{books} -- a plural count noun --, \textit{chocolate(s)} -- a dual-life noun --, and \textit{livestock} -- a naturally atomic mass noun:\footnote{Judgements of English native speakers are divided on the acceptability of \REF{schv-nem:ex:32a}. For discussion, see \citet{erbach-etal-19}. } 

\ea \label{schv-nem:ex:32}
\ea[]{Twenty kilos of books are lying on top of each other.}\label{schv-nem:ex:32a}
\ex[]
{I bought 200gs of chocolates, each of which was filled with a different kind of ganache.}\label{schv-nem:ex:32b}
\ex[*] {I made 1.5 kgs of hummus, each of which was eaten at the party.} \label{schv-nem:ex:32c}
\ex 
[?]{Quite a few livestock/cattle have disappeared today. \\ { } \hfill \citep[p. 93 (13)--(18)]{erbach-etal-19}}\label{schv-nem:ex:32d}
\z
\z

\noindent The equivalents of all of the above sentences are ruled out or have a low degree of acceptability in Hungarian. We can see that \REF{schv-nem:ex:33a} is not unanimously accepted by informants: while it may be interpreted as a plurality of books piled on top of each other having a cumulative weight of 20 kilos, some informants can only interpret it as 20 books each of which weighs one kilo. Moreover, both native speaker authors of this paper consider this sentence slightly infelicitous, yet for different reasons. The first author points out a preference for expressing the situation described in \REF{schv-nem:ex:33a} with a different structure, roughly equivalent to `There are books on top of each other which in total weigh 20 kilos.' The second author finds the combination of the measure phrase \textit{kilo} and the reciprocal phrase \textit{egymás tetején} `on top of each other' unacceptable. As for dual life nouns, chocolate in Hungarian patterns with the mass \textit{hummus}  \REF{schv-nem:ex:33b}--\REF{schv-nem:ex:33c}; and nouns such as \textit{állatállomány} `livestock' are ruled out with count quantifiers such as ‘a few’ \REF{schv-nem:ex:33d}.

\ea \label{schv-nem:ex:33}
\ea[???] {\gll Húsz kiló könyv egymás tetején van a föld-ön. \\
20 kilo book each.other  on.top.of be.\textsc{pres}.\textsc{3sg} the ground-\textsc{sup}\\
\glt `Twenty kilos of books are on top of each other on the ground.'}  \label{schv-nem:ex:33a}
\ex[*] {\gll 200g csokoládé-t vettem, mindegyik más töltelék-kel volt megtöltve.\\
200g  chocolate-\textsc{acc} bought each different   filling-\textsc{inst}
be.\textsc{past}.\textsc{3sg} filled \\
\glt `I bought 200 grams of chocolate, each of which was filled with a different filling.'} \label{schv-nem:ex:33b}  
\ex[*] {\gll Másfél kiló humuszt készítettem, mindegyik-et megették a parti-n.\\
one.and.a.half kilo hummus-\textsc{acc} prepare.\textsc{past}.\textsc{1sg} each-\textsc{acc} \textsc{vm}-eat.\textsc{past}.\textsc{3pl} the party-\textsc{sup}\\
\glt `I prepared one and a half kilo of hummus, each of which was eaten at the party.'} \label{schv-nem:ex:33c}
\ex[*]{\gll Elég kevés állatállomány tűnt el ma.\\
quite few livestock disappear.be.\textsc{past}.\textsc{3sg} \textsc{vm} today   \\
\glt `Quite a few livestock have disappeared today.'}  \label{schv-nem:ex:33d}
\z
\z

\noindent Second, \citeauthor{erbach-etal-19}’s (\citeyear{erbach-etal-19}) assumption that bare singular nouns lack a mass reading in argument position does not hold. Our data indicate that mass readings of singular Ns are available in fact in full argumental positions. In \REF{schv-nem:ex:34} \textit{könyv} is preceded by the definite determiner , while in \REF{schv-nem:ex:35} it appears bare. We assume that bare singular nominals that have a kind interpretation are mass nouns \citep{chierchia-98b}. The bare singular nouns \textit{könyv} `book' and \textit{ima} `prayer' in \REF{schv-nem:ex:35} pattern with the bare mass nouns \textit{homok} `sand' and \textit{vér} `blood' \REF{schv-nem:ex:36}, evincing a mass interpretation to such nouns.\footnote{The kind interpretation in \REF{schv-nem:ex:35} is not due to the conjunction, the same holds for a bare singular: 

\ea
\gll Eminens tanuló-nak könyv fölött a hely-e. \\
eminent student-\textsc{dat} book above the place-\textsc{poss}.\textsc{3sg} \\
\glt `The place of eminent students is above books.'
\z
} The interpretation of nominals in Hungarian will be further discussed in the next section.


\ea \label{schv-nem:ex:34}
\gll A könyv ritka jószág manapság amikor mindenki már Kindle-t használ. \\  
 the book rare stuff nowadays when everyone already Kindle-\textsc{acc} use.\textsc{pres}.\textsc{3sg}\\ 
\glt `Books are rare  nowadays when everybody uses Kindle already.'
\ex \label{schv-nem:ex:35}
\gll Könyv és ima a mindennapi intellektuális táplálék-om.%\footnotemark 
\\  
book   and  prayer  the daily      intellectual    nutrition-\textsc{poss}.\textsc{1sg}\\ 
\glt `Books and prayers are my daily intellectual nutrition.'
\z


\ea \label{schv-nem:ex:36}
\gll Te jól láthatod, amit én érzek, azt kifejezi a képeslap, amelye-n homok és víz egyesül.\\  
 you well see.\textsc{pos}.\textsc{pres}.\textsc{2sg}  what I feel.\textsc{pres}.\textsc{1sg} that express.\textsc{pres}.\textsc{3sg} the postcard which-\textsc{sup} sand and water merge.\textsc{pres}.\textsc{3sg}\\ 
\glt `You may see well what I feel, it is expressed by the postcard on which sand and water merge.' \\\hfill (Source: Hungarian National Corpus, MNSZ 2, \citealt{oravecz-14})
\z

\noindent In sum, the number-neutral analysis may not accurately reflect the empirical facts of Hungarian, as the discussion of the accessibility of atoms in measure DPs is English-based. In addition, as we have shown, the linguistic facts are different in Hungarian. Moreover, the  number-neutral analysis does not account for mass readings of bare singular nouns. Further evidence ruling out the number-neutral approach to the Hungarian nominal system can be found in \citet{schvarcz-nemes-19}.  We now turn to the interpretation of nominals in Hungarian. 

\section{Are Hungarian nouns kind-denoting?} \label{schv-nem:sec:4}

Although the data indicate that Hungarian cannot be considered to be a classifier language, the question remains: Is the function of the sortal classifier in Hungarian the same as in typical classifier languages? In order to provide an answer to this question we will look at the interpretation of bare nominals in Hungarian, as the use of classifiers is closely related to nominal denotation. 

\subsection{Exploring kinds in Hungarian} \label{schv-nem:sec:4.1}

Regarding the basic denotation of nominals, \citet{chierchia-98b} distinguishes two types of languages. On the one hand, in languages like Mandarin Chinese all nouns have a default kind interpretation and can be used as arguments without determiners. On the other hand, in languages like English, count nouns denote properties and since these nouns are of the predicative-type, in order to be used as arguments the use of determiners is required. In contrast, mass nouns in this second type of language are assumed to denote kinds and can be used determinerless in argument positions.  

Focusing on classifier optionality, we contrast Hungarian with the Chinese-type of languages. In these languages, classifiers are obligatorily used in order to retrieve instantiations of a kind, thereby allowing numerical modification. Unlike in typical classifier languages, however, classifiers in Hungarian are optionally used. These facts raise the question about the interpretation of Hungarian nominals: are they kind-denoting as are their the Chinese counterparts, or property denoting as in English?

In Mandarin Chinese, bare nouns can be used as subjects of kind-level predicates \REF{schv-nem:ex:37} \citep{li-13}, while the kind interpretation seems to be much more limited in Hungarian. Kind-level predicates in Hungarian require the definite construction \REF{schv-nem:ex:38}:

\ea \label{schv-nem:ex:37} \gll jing kuai juezhong le.\\  
whale soon be.extinct \textsc{prf}\\
\glt `Whales will soon be extinct.'\hfill \citep[Mandarin Chinese;][p. 90, ex. (4)]{li-13}

% \ea \label{schv-nem:ex:37} \gll jing kuai juezhong le.\\  
% whale soon be.extinct \textsc{prf}\\ 
% \glt `Whales will soon be extinct.'\hfill \raisebox{2.2\baselineskip}[0pt][0pt]{[Mandarin Chinese]} \\ { } \hfill \raisebox{.0\baselineskip}[0pt][0pt]{\citep[p. 90, ex. (4)]{li-13}}
% \z

\ex \gll \minsp{*(} A) bálna a kihalás szélé-n áll.\\  
{} the whale the extinction verge-\textsc{sup} stand.\textsc{pres}.\textsc{3sg}\\ 
\glt `Whales are on the verge of extinction.' \label{schv-nem:ex:38}
\z

\noindent In certain constructions bare nominals can have a kind interpretation. \citet{farkas-deswart-03} suggest that generic interpretations of bare plurals are not usually available in Hungarian, unless they are incorporated. \citet{schvarcz-rothstein-17} show that with kind-level predicates, incorporated bare plurals can be interpreted as kinds. Constructions such as \REF{schv-nem:ex:39} are limited, and their interpretation varies among informants between kind and subkind readings. The incorporated bare plural \textit{bálnák} `whales' in \REF{schv-nem:ex:40} can have both a plurality of individuals as well as a kind interpretation.

\ea \label{schv-nem:ex:39}
\gll Bálná-k állnak a kihalás szélé-n.	\\
Whale-\textsc{pl} stand.\textsc{pres}.\textsc{3pl} the  extinction verge-\textsc{sup}\\ \hfill [K./Sk.]
\glt `Whales (in general) are / the whale is on the verge of extinction.' / `Some kinds of whales are on the verge of extinction.'  \\\hfill  \citep[p. 188, (13)]{schvarcz-rothstein-17}

% \ea \label{schv-nem:ex:39}
% \gll Bálná-k állnak a kihalás szélé-n.	\\
% Whale-\textsc{pl} stand.\textsc{pres}.\textsc{3pl} the  extinction verge-\textsc{sup}\\
% \glt `Whales (in general) are / the whale is on the verge of extinction.' / `Some kinds of whales are on the verge of extinction.'  \hfill \raisebox{3.3\baselineskip}[0pt][0pt]{[K./Sk.]} \\ \citep[p. 188, (13)]{schvarcz-rothstein-17}
% \z

\ex \label{schv-nem:ex:40}
\gll János és Béla bálná-k-at vadásznak az óceán-ban.	\\
John and Bill whale-\textsc{pl}-\textsc{acc} hunt.\textsc{pres}.\textsc{3pl} the ocean-\textsc{iness}\\ \hfill [Pl./K.]
\glt `John and Bill are hunting whales in the ocean.' / `John and Bill are whale hunters (and not dolphin hunters).'
\z

% \ea \label{schv-nem:ex:40}
% \gll János és Béla bálná-k-at vadásznak az óceán-ban.	\\
% John and Bill whale-\textsc{pl}-\textsc{acc} hunt.\textsc{pres}.\textsc{3pl} the ocean-\textsc{iness}\\
% \glt `John and Bill are hunting whales in the ocean.' / `John and Bill are whale\textsubscript{kind}-hunters (and not dolphin-hunters).'  \hfill \raisebox{3.3\baselineskip}[0pt][0pt]{[Pl./K.]}
% \z

\noindent \citet{carlson-77} takes narrow-scope reading of bare plurals as an indication of a kind interpretation. This phenomenon can also be observed in Hungarian:  

\ea \label{schv-nem:ex:41}
\gll János és Béla rózsá-k-at keresnek a piac-on.\\
John and Bill rose-\textsc{pl}-\textsc{acc} look.for.\textsc{pres}.\textsc{3pl} the market-\textsc{sup}\\
\glt `John and Bill are looking for roses on the market.' \\\hfill \citep[p. 203, (13)]{schvarcz-rothstein-17}
\z

\noindent Bare plural subjects of achievement verbs have a kind interpretation, as argued by \citet{landman-rothstein-10}. This has been shown to hold for Hungarian as well (\citealt{schvarcz-rothstein-17} -- see example \REF{schv-nem:ex:31} above). \REF{schv-nem:ex:42}, modelled on their example, shows that this is generally available in Hungarian:

\ea \label{schv-nem:ex:42}
\gll Nagy-ot csalódtunk a delfinfigyelő túrá-n mert két órá-n át báln-ák érkeztek \minsp{(} és nem delfin-ek).\\
big-\textsc{acc} be.disappointed.\textsc{past}.\textsc{3pl} the dolphin.watching tour-\textsc{sup} because two hour-\textsc{sup} for  whale-\textsc{pl}  arrive.\textsc{past}.\textsc{3pl} {} and not dolphin-\textsc{pl}\\
\glt `We were very disappointed by the dolphin-watching tour since whales arrived for two hours and (not dolphins).'  
\z

\noindent \citet{schvarcz-18} shows that in contrastive contexts, a kind interpretation of bare plurals is widely available:

\ea \label{schv-nem:ex:43}
\gll Hód-ok építenek gát-ak-at, nem menyét-ek.\\
beaver-\textsc{pl} build.\textsc{pres}.\textsc{3pl} dam-\textsc{pl}-\textsc{acc} not  weasel-\textsc{pl}\\
\glt `Beavers build dams, not weasels.' \hfill \citep[p. 116, (50a)]{schvarcz-18}
\ex \label{schv-nem:ex:44}
\gll Ember-ek vagyunk, nem állat-ok.\\
people-\textsc{pl} be.\textsc{pres}.\textsc{1pl} not animal-\textsc{pl}\\
\glt `We are people, not animals.'  \hfill \citep[p. 116, (50c)]{schvarcz-18}
\z

\noindent The fact that bare plurals can be interpreted as kinds is not surprising, given the fact that this is also the case in mass/count languages, such as English \citep{carlson-77}. Yet the case of bare singulars in Hungarian remains unexplored. The availability of a kind interpretation with these nouns is of capital importance for determining whether nouns can indeed be seen as kind-denoting. Hungarian bare singulars can get a kind interpretation in negative sentences: when the verb is under negation \REF{schv-nem:ex:45}--\REF{schv-nem:ex:46} as well as in contrastive contexts \REF{schv-nem:ex:47}--\REF{schv-nem:ex:49}:

\ea \label{schv-nem:ex:45}
\gll Ember ilyet nem csinál.\\
man this not do.\textsc{pres}.\textsc{3sg}\\
\glt `Men don’t do this/such a thing.' \hfill \citep[p. 115, (49a)]{schvarcz-18}
\ex \label{schv-nem:ex:46}
\gll Sas nem kapkod legy-ek után.\\
eagle not fluster.\textsc{pres}.\textsc{3sg} fly-\textsc{pl}  after\\
\glt `Eagles do not fluster after flies.' \hfill \citep[p. 115, (49b)]{schvarcz-18}
\ex \label{schv-nem:ex:47}
\gll Nem sas lopkodja a tyúk-ok-at hanem róka.\\
not eagle steal.\textsc{pres}.\textsc{3sg} the hen-\textsc{pl}-\textsc{acc} but fox\\
\glt `It is not the eagles who are stealing hens but foxes.' / `It is not an eagle who is stealing hens but a fox.' 
\ex \label{schv-nem:ex:48}
\gll Nem búzá-t termesztenek Ázsiá-ban hanem rizs-et.\\
not wheat-\textsc{acc} grow.\textsc{pres}.\textsc{3pl} Asia-\textsc{iness} but rice-\textsc{acc}\\
\glt `It is not wheat that they grow in Asia, but rice.' 
\ex \label{schv-nem:ex:49}
\gll  János könyv-et szeret olvasni, nem újság-ot. \\
John book-\textsc{acc} like.\textsc{pres}.\textsc{3sg} read.\textsc{inf} not newspaper-\textsc{acc}\\
\glt `John likes reading books, not newspapers.' 
\z

\noindent As the examples above show, in a limited number of contexts both bare singular and bare plural nouns can get a kind interpretation. Nevertheless, the kind interpretation seems to be significantly more limited in Hungarian than it is in the case of typical classifier languages, like Mandarin Chinese or Japanese. The default choice for expressing a generic is the use of the definite construction \REF{schv-nem:ex:38}. Unlike typical classifier languages which generally lack a definite article, Hungarian has one.

\subsection{A hypothesis for the denotation of Hungarian bare nominals} \label{schv-nem:sec:4.2}

Our hypothesis is that in terms of kind reference, Hungarian count nouns are property-denoting, while mass nouns are kind-denoting. The mass counterpart of a flexible pair has a default kind interpretation, and hence can appear bare in characterizing sentences \REF{schv-nem:ex:45} and generic statements \REF{schv-nem:ex:48}. This is also the reason why it requires a classifier upon combination with numerals -- see \REF{schv-nem:ex:7b} above. We assume that the classifier takes the mass counterpart of a flexible noun, a kind-denoting term, and turns it into a property-denoting one. 

Assuming that in Hungarian the majority of nouns, if not all, are flexible between count and mass versions, which correspond to a property-denoting and to a kind-denoting term respectively, both a definite and a bare construction is available for achieving genericity. Nevertheless, the definite construction is favored, while the bare construction is more marked and is available in contextually and syntactically restricted cases only. While the default argument of kind-taking predicates in generic sentences is a definite phrase, incorporation, negative and contrasting structures seem to override this requirement. Syntactically, we assume that these constructions have a more complex structure which allows bare nominals to receive a kind reading. A more comprehensive account and a formal analysis of this issue is a subject for further study. 
   
\section{The semantics of Hungarian classifiers} \label{schv-nem:sec:5}

We define the meaning of classifiers in a framework in which kinds are perceived to be individual concepts, functions from worlds to pluralities. The newspaper-kind can be thought of as the set of newspapers, the totality of newspapers, the sum of all instances of the newspaper kind \citep{chierchia-98b}. 

Treating mass-counterparts of Hungarian flexible nouns as kind-denoting terms lends itself to an analysis of sortal individuating classifiers under which they are functional operators on kinds, expressions of type  $\stb{k,\stb{e,t}}$. Classifiers serve as functions to access the instantiations of a kind modeled by the \cnst{inst} operation. In other words, classifiers apply to kind denoting terms generating the set of individuals such that they are instantiation of that kind.  From this perspective, the semantics of the general classifier \textit{darab} could be formalised as follows:\footnote{In line with \citet{rothstein17} we assume that numerals in prenominal positions are functions that map entities onto the value true if they have $n$ atomic parts.}
 

\ea \label{schv-nem:ex:50}
\ea
\gll három darab újság \\
three \textsc{cl\textsubscript{general}} newspaper\\
\ex \sib{darab}${} = \lambda k. \lambda x.\cnst{inst}(x,k)$
\ex \sib{darab újság}${} = \lambda x.\cnst{inst}(x,\textsc{newspaper}_{\textsc{kind}})$
\ex \sib{három}${} = \lambda x.  |x| =3$ %\footnotemark
\ex \sib{három darab újság}${} =  \lambda x.\textsc{inst}(x,\textsc{newspaper}_{\textsc{kind}}) \wedge |x| =3$
\z
\z


\noindent  Our semantics of the general classifier can be further extended to those sortal individuating classifiers in Hungarian that select nouns based on size, shape and form:

\ea \label{schv-nem:ex:51}
\ea
\gll két fej hagyma \\
two \textsc{cl\textsubscript{head}} onion\\
\ex \sib{head}${} = \lambda k. \lambda x.\cnst{inst}(x,k) \wedge \textsc{large}(x) \wedge \textsc{round}(x)$
\ex \sib{fej hagyma}${} = \lambda x.\cnst{inst}(x,\textsc{onion}_{\textsc{kind}}) \wedge \textsc{large}(x) \wedge \textsc{round}(x)$
\ex \sib{két}${} = \lambda x.  |x| =2$
\ex \sib{két fej hagyma}\\${} =  \lambda x.\cnst{inst}(x,\textsc{onion}_{\textsc{kind}})\wedge \textsc{large}(x) \wedge \textsc{round}(x) \wedge |x| =2$
\z
\ex \label{schv-nem:ex:52}
\ea
\gll három szál rózsa \\
three \textsc{cl\textsubscript{thread}} rose\\
\ex \sib{szál}${} = \lambda k. \lambda x.\cnst{inst}(x,k) \wedge \textsc{long}(x) \wedge \textsc{thin}(x)$
\ex \sib{szál rózsa}${} = \lambda x.\cnst{inst}(x,\textsc{rose}_{\textsc{kind}}) \wedge \textsc{long}(x) \wedge \textsc{thin}(x)$
\ex \sib{három}${} = \lambda x.  |x| =3$
\ex \sib{három szál rózsa}\\${} =  \lambda x.\cnst{inst}(x,\textsc{rose}_{\textsc{kind}}) \wedge \textsc{long}(x) \wedge \textsc{thin}(x) \wedge |x| =3$
\z
\z

\noindent If we assume that classifiers take kind-denoting expressions, then examples where the plural and the classifier can co-occur \REF{schv-nem:ex:20}--\REF{schv-nem:ex:22} require further explanation. One option to explain such examples is to treat the bare plural noun as denoting a kind. This is further supported by \REF{schv-nem:ex:39}--\REF{schv-nem:ex:44} illustrating the kind interpretation of bare plurals in various contexts. Another option is to compose the structure of classifier-plural nominal co-occurrences in the following way: the classifier combines with the kind-denoting singular noun deriving instantiations of the noun, which is then pluralized. This would assume a syntax in which the plural marker \textit{-k} is higher than the CL+N. Both of these options allow us to maintain the semantics of classifiers proposed in this paper.  

\section{Summary and implications} \label{schv-nem:sec:6}

This paper explored classifier optionality in Hungarian and argued that the phenomena can best be captured in a noun-flexibility approach, while the role of sortal individuating classifiers is to trigger a kind-to-predicate shift in nouns which are born as kind-denoting expressions.  

The foundation of our analysis is a flexibility-based approach to Hungarian mass/count phenomena, according to which most nouns in the language are ambiguous between a mass and count denotation \citep{schvarcz-rothstein-17}. The count and mass versions are derived from the same neutral lexical root of a noun, via the \cnst{count} and \cnst{mass} operations, resulting in  two identical lexical forms. Under this analysis, flexibility is a purely grammatical phenomenon and does not postulate any semantic ambiguity. This approach has numerous advantages over alternative theories of Hungarian nominal semantics. It  helps explain novel data that neither a non-ambiguity \citep{dekany-11, csirmaz-dekany-14} nor an underspecification \citep{erbach-etal-19} approach has discussed. In addition, it accounts for the optionality of sortal individuating classifiers and captures the interpretational differences of Hungarian numerical expressions. 

We first explored the differences in interpretation between NUM+N and \linebreak NUM+CL+N constructions and showed that there is a significant interpretational difference between them: while the former can refer to a plurality of individuals or to a plurality of subkinds, the insertion of the classifier in the latter construction restricts the reading to a plurality of individuals. 

We then provided evidence in defense of the noun-flexibility approach showing that neither a plural-as-a-classifier nor a number-neutrality approach captures the semantic effect induced by the optional classifier.  The distribution, interpretation, and co-occurrence of plurals and classifiers as well as the different agreement patterns induced by the two strongly suggest that the plural cannot be treated as a classifier. Moreover, number-neutral analysis does not account for mass readings of bare singular nouns nor for the semantic input of the classifier observed in our study. 

We claimed that Hungarian nominals are kind-denoting by default and can undergo a kind-to-predicate shift \citep{chierchia-98a} explicitly triggered by a sortal individuating classifier. Hungarian has a unique set of properties, allowing both for bare and for definite constructions to express kind. We have shown that bare singulars with a kind-reading are available both for mass Ns and for the mass counterparts of flexible Ns, indicating that nouns are kind-denoting expressions. 

Hungarian appears to be a ``mixed system'' in terms of the use of a mass/count system and classifiers and has unique properties with regards to the distribution and interpretation of bare nominals, which points to more typological variation between languages than has been suggested before. 

%\input{example-osl.tex}

\section*{Abbreviations}

\begin{tabularx}{.5\textwidth}{@{}lX@{}}
\textsc{1}&first person\\
\textsc{2}&second person\\
\textsc{3}&third person\\
\textsc{$_{\emptyset}$}&null element\\
\textsc{acc}&accusative\\
\textsc{cl}&classifier\\
\textsc{dat}&dative\\
\textsc{de}&associative particle\\
\textsc{dem}&demonstrative\\
\textsc{dur}&durative\\
\textsc{ela}&elative\\
\textsc{ill}&illative\\
\textsc{iness}&inessive\\
\textsc{inf}&infinitive\\
\textsc{inst}&instrumental\\
\end{tabularx}%
\begin{tabularx}{.5\textwidth}{@{}lX@{}}
\textsc{k.}&kind reading\\
\textsc{past}&{past tense}\\
\textsc{pl}&plural\\
Pl.&plurality of individuals reading\\
\textsc{pos}&possibility\\
\textsc{poss}&possessive\\
\textsc{pres}&present tense\\
\textsc{prf}&perfectivity marker\\
\textsc{red}&reduplication\\
\textsc{sbl}&sublative\\
\textsc{sg}&singular\\
Sk.&subkind reading\\
\textsc{sup}&superessive\\
\textsc{vm}&verbal modifier\\
{}&{}\\
\end{tabularx}

\section*{Acknowledgements}
We would like to thank Gabi Danon, Yasutada Sudo, Éva Dékány for their helpful remarks and two anonymous reviewers for comments on an earlier version of the paper. We are grateful to a number of informants for judgments. We are indebted to Susan Rothstein ( ז״ל - may her memory be blessed) for her suggestions on an initial version of the proposal. We owe many thanks to Kata Wohlmuth for her help with LaTeX. We also thank the audiences of the Semantics Research Group of Bar-Ilan University, the SinFonIJA 12, 15.ik Felúton, Constructions of Identity 10 and NaP2019 conferences for useful discussions and feedback on the topic. This work was partially supported by the Azrieli Fellowship to Brigitta R. Schvarcz. 

{\sloppy\printbibliography[heading=subbibliography,notkeyword=this]}

\end{document}
